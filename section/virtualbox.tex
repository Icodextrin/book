\FILENAME

\section{Virtual Box Installation and
Instructions}\label{virtual-box-installation-and-instructions}

For development purposes we recommend tha you use for this class an
ubuntu virtual machine that you set up with the help of virtualbox.

Only after you have successfully used ubuntu in a virtual machine you
will be allowed to use virtual machine son clouds.

A ``cloud drivers license test'' will be conducted to let you gain
access to the cloud infrastructure. We will announce this test. Before
you have not passed the test, you will not be able to use the clouds.
Furthermore, you do not have to ask us for join requests before you have
not passed the test. Please be patient. Only students enrolled in the
class can get access to the cloud.

\subsection{Creation}\label{creation}

First you will need to install virtualbox. It is easy to install and
details can be found at

\begin{itemize}
\tightlist
\item
  \url{https://www.virtualbox.org/wiki/Downloads}
\end{itemize}

After you have installed virtualbox you also need to use an image. For
this class we will be using ubuntu Desktop 16.04 which you can find at:

\begin{itemize}
\tightlist
\item
  \url{http://www.ubuntu.com/download/desktop}
\end{itemize}

Please note the recommended requirements that also apply to a virtual
machine:

\begin{itemize}
\tightlist
\item
  2 GHz dual core processor or better
\item
  2 GB system memory
\item
  25 GB of free hard drive space
\end{itemize}

A video to showcase such an install is available at:

\begin{itemize}
\tightlist
\item
  Video: \url{https://youtu.be/NWibDntN2M4}
\end{itemize}

\begin{description}
\item[If you specify your machien too small you will not be]
able to install the development environment. Gregor used on his machine
8gb of RAM and 20GB diskspace.
\end{description}

Please let us know the smalest configuration that works for you and
share this in Piaza. Only update if yours is smaller and works than a
previous post. If not do not post.

\subsection{Guest additions}\label{guest-additions}

The virtual guest additions allow you to easily do the following tasks:

\begin{itemize}
\tightlist
\item
  Resize the windows of the vm
\item
  Copy and paste content between the Guest operating system and the host
  operating system windows.
\end{itemize}

This way you can use many native programs on you host and copy contents
easily into for example a terminal or an editor that you run in the Vm.

A video is located at

\begin{itemize}
\tightlist
\item
  Video: \url{https://youtu.be/wdCoiNdn2jA}
\end{itemize}

Please reboot the machine after installation and configuration.

On OSX you can once you have enabled bidirectional copying in the Device
tab with

\begin{description}
\item[OSX -\textgreater{} Vbox:]
command c -\textgreater{} shift CONTRL v
\item[Vbox to OSX:]
shift CONTRL v -\textgreater{} shift CONTRL v
\end{description}

On Windows the key combination is naturally different. Please consult
your windows manual.

\subsection{Exercise}\label{exercise}

\begin{description}
\item[Virtualbox.1:]
Install ubuntu desktop on your computer with guest additions.
\item[Virtualbox.2:]
Make sure you know how to paste and copy between your host and guest
operating system
\item[Virtualbox.3:]
Install the programs defined by the development configuration
\end{description}
