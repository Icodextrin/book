

\chapter{Virtual Box}
\label{S:virtual-box}

\FILENAME

For development purposes we recommend tha you use for this class an
Ubuntu virtual machine that you set up with the help of virtualbox. We
recommend that you use the current version of ubuntu and do not
install or reuse a version that you have set up years ago.

As access to cloud resources requires some basic knowledge of linux
and security we will restrict access to our cloud services to thopse
that have demonstrated responsible use on their own
computers. Naturally as it is your own computer you must make sure you
follwo proper security. We have seen in the past students carelessly
working with virtual machines and introducing security vulnerabilities
on our clouds just becasue ``it was not their computer.'' Hence, we
will allow using of cloud resources only if you have demonstrated that
you responsibly use a linux virtual machine on your own computer.
Only after you have successfully used ubuntu in a virtual machine you
will be allowed to use virtual machines on clouds.

A ``cloud drivers license test'' will be conducted. Only after you
pass it we wil let you gain access to the cloud infrastructure. We
will announce this test. Before you have not passed the test, you will
not be able to use the clouds.  Furthermore, you do not have to ask us
for join requests to cloud projects before you have not passed the
test. Please be patient. Only students enrolled in the class can get
access to the cloud. 

If you however have access to other clouds yourself you are welcome to
use the, However, be reminded that projects need to be reproducable,
on our cloud. This will require you to make sure a TA can replicate it.

Let us now focus on using virtual box.

\section{Instalation}\label{creation}

First you will need to install virtualbox. It is easy to install and
details can be found at

  \URL{https://www.virtualbox.org/wiki/Downloads}

After you have installed virtualbox you also need to use an image. For
this class we will be using ubuntu Desktop 16.04 which you can find at:

  \URL{http://www.ubuntu.com/download/desktop}


Please note some hardware you may have may be too old or has too
little resources to be useful. We have heard from students that the
following is a minimal setup for the desktop machine:

\begin{itemize}
\item  multi core processor or better allowing to run hypervisors
\item  8 GB system memory
\item  50 GB of free hard drive space
\end{itemize}

For virtual machines you may need multiple, while the minimal
configuration may not work for all cases.

As configuration we often use

\begin{description}
\item[minimal] 1 core, 2GB Memory, 5 GB disk 
\item[latex] 2 core, 4GB Memory, 25 GB disk 
\end{description}

A video to showcase such an install is available at:

\video{Virtualbox}{seconds}{Video}{https://youtu.be/NWibDntN2M4}

\begin{NOTE}
  If you specify your machine too small you will not be able to
  install the development environment. Gregor used on his machine 8GB
  RAM and 25GB diskspace.
\end{NOTE}

Please let us know the smalest configuration that works.

\section{Guest additions}\label{guest-additions}

The virtual guest additions allow you to easily do the following tasks:

\begin{itemize}
\item  Resize the windows of the vm
\item  Copy and paste content between the Guest operating system and
  the host operating system windows.
\end{itemize}

This way you can use many native programs on you host and copy contents
easily into for example a terminal or an editor that you run in the Vm.

A video is located at

\video{Virtualbox}{4:46}{Video}{https://youtu.be/wdCoiNdn2jA}


Please reboot the machine after installation and configuration.

On OSX you can once you have enabled bidirectional copying in the Device
tab with

\begin{description}
\item[OSX to Vbox:] command c shift CONTRL v
\item[Vbox to OSX:] shift CONTRL v shift CONTRL v
\end{description}

\begin{NOTE}
  On Windows the key combination is naturally different. Please
  consult your windows manual. If you let us know TAs will add the
  information here.
\end{NOTE}




\section{Exercise}

\begin{description}
\item[Virtualbox.1:] Install ubuntu desktop on your computer with
  guest additions.
\item[Virtualbox.2:] Make sure you know how to paste and copy between
  your host and guest operating system.
\item[Virtualbox.3:] Install the programs defined by the development
  configuration.
\item[Virtualbox.4:] Provide us with the key combination to copy and
  paste between Windows and Vbox.
\end{description}

