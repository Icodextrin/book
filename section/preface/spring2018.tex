
\chapter{Course Policies}\label{C:course-2018}

\FILENAME

\begin{WARNING}
  As the class material will evolve during the semester it is obvious
  that some content will be improved and material will be added. This
  benefits all classes. To stay up to date, please, revisit this
  document on weekly basis. This is obvious, as we will adapt content
  based on your feedback. 

  The requirement of the classes to become an expert in cloud and/or
  Big Data applications and technologies stays unchanged.
\end{WARNING}


Residential students are required to attend all Friday classes
including those held in May.  Missing such classes will result in
grade reduction. Based on past class experience we are foced to
implement a strict strict policy. Only for medical reasons a class can
be missed. The reasoning for this is that residential students that
did not attend classes regularly tend to do not as well in this class
as their colleagues. As we want that you achieve your best we strongly
advise to attend and give with this policy an additional motivating
factor.

\section{HID}

You will be assigned an hid (Homework IDentifier) which allows us to
easily communicate with you and doe allow us to not use your
university ID to communicate with you. 

You will receive the HID within the first week of the semester by the
TA's.

\section{Notebook}\label{notebook}

All students are required to maintain a \emph{class notebook} in
github in which they summarize their weekly activities for this course
in bullet form. This includes a self maintained list of which lecture
material they viewed.

The notebook is maintained in the class github.com in your hid project
folder. It is a file called notebook.md that uses markdown as format.
While using md, you can either edit it locally and upload to github, or
directly edit it via the git hub Web editor.

You will be responsible to set up and maintaining the notebook.md and
update it accordingly. We suggest that you prepare sections such as:
Logistic, Theory, Practice, Writing and put in bullet form what you have
done into these sections during the week. We can see from the github
logs when you changed the notbook.md file to monitor progress. The
management of the notebook will be part of your discussion grade.

The format of the notebook is very specific markdown format and must
follow these rules:

\begin{itemize}
\item use headings with the \# character and have a space after the \#
\item use bullets in each topic.
\item each bullet \textbf{must} have an individual date that is of the
  form mm/dd/yy. Please do not lump bullet points under s single
  date. Have each bullet point its own date
\item if you have done the activity in a period than add the second
  date to it mm/dd/yy - mm/dd/yy
\item If you refer to section numbers in your notebook, please aslo
  add the section title as the section numbers may change in case we
  need to add content
\end{itemize}

Please examine carefully the sample note book is available at:

  \URL{https://raw.githubusercontent.com/bigdata-i523/sample-hid000/master/notebook.md}

This will render in git as:

  \URL{https://github.com/bigdata-i523/sample-hid000/blob/master/notebook.md}

The notebook.md is not a blog and should only contain a summary of
what you have done. 

\section{Blog}

Optional: If you like to maintain your own blog, you can create yourself also a
blog.md file. However do not include sensitive information in
there. A blog is not a replacement for the notebook.

\section{Calendar I524, E616, E516}\label{S:calendar}

This class is a full term class of 16 weeks.

\begin{IU}

The semester calendar is posted at 

\URL{http://registrar.indiana.edu/official-calendar/official-calendar-spring.shtml}

The class beginns Mon, Jan 8th and ends Fri, May 4th

\end{IU}

\begin{longtable}{p{3cm}p{11cm}}
  \caption{Calendar} \\   
  \toprule
  Date & Activity \\
  \midrule
  \endfirsthead
  \toprule
  Date & Activity \\
  \endhead
  \hline
  \multicolumn{2}{c}{Continued}\\   \bottomrule
  \endfoot
  \bottomrule
  \endlastfoot

  Jan 8, Mon & Class Begins\\
  
  Jan 15, Mon 9am & Setup communication pathways for the class. (1)
  You must have created a github repository in our class repository.
  (2) You must be in the class Piazza.  (3) Motivation: if we can not
  communicate with you we can not conduct the class. Everyone must be
  in piazza and github timely.  \\

  Weekly & contribution to notebook.md \\
  Weekly & contribution to piazza and/or the Handbook \\

  Jan 15, Mon & MLK Jr. Day. Good day to work on projects, computer setup \\
  Jan 22, Mon 9am & Tutorial 1 \\
  Feb 5,  Mon 9am & Tutorial 2 \\
  Feb 26, Mon 9am & Paper 1 \\
  Mar 5, Mon 9am & Project draft paper due without panelty \\
  \hline
  Spring Break &\\
  Mar 11 - Mar 18.  & This is a good time to work ahead or catch up
  with things. We strongly advise to use this time wisely. \\
  \hline
  Mar 16, Mon 9am & Project reports due without penalty \\
  Mar 23, Mon 9am & Improvments to Projects and documents possible,
  but substential work must have been done before to not encounter a
  grade reduction \\
  May 1 & Any paper submitted after May 1st will get an
  incomplete and a grade reduction. \\
\end{longtable}

\section{Incomplete}\label{incomplete}

Incompletes have to be explicitly requested in piazza. All incompletes
have to be filed by May 1st.

Incomplete's will receive a fractional Grade reduction: A will
become A-, A- will become B+, and so forth. There is enough time in the
course to complete all assignments without getting an incomplete.

Why do we have such a policy? As we teach state-of-the-art software this
software is subject to change, not only within the course, but also
after the course. As we may offer some services and only have access to
the TA's during the semester it is obvious that we like all class
projects and homework assignments to be completed within a semester.
Services that were offered during the semester may no longer be
available after the semester is over and could adversely effect your
planing. It will be in the students responsibility to identify such
services and provide alternatives if they become unavailable. We try
hard to avoid this but we can not guarantee it.

Furthermore, once an incomplete is requested, you will have 10 month to
complete it. We will need 2 month to grade. No grading will be conducted
over breaks. This may effect those that require student loans. Please
plan ahead.

The incomplete request needs to be off the following format in piazza:

\begin{verbatim}
Subject: 
    INCOMPLETE REQUEST: HID000: Lastname, Firstname

Body:
    Firstname: TBD
    Lastname: TBD
    HID: TBD
    Semester: TBD
    Course: TBD
    Online: yes/no

    URL notebook: TBD    
    URL tutorial1: TBD
    URL tutorial2: TBD
    URL paper1: TBD
    URL project: TBD

    URL other1: TBD
\end{verbatim}

Please make sure that the links ar e clickable in piazza.

\section{Registration Information}\label{registration-information}

The folloing course numbers are sections of this class

The official registration information can be found here:

\begin{itemize}
\tightlist
\item
  \url{http://registrar.indiana.edu/official-calendar/official-calendar-fall.shtml}
\end{itemize}

We summarize, but like to point out that the information here may have
changed. We advise to visit the official page. However important to note
is that all residential students meet:

\begin{verbatim}
09:30A-10:45A   Friday      I2 150 
\end{verbatim}

\subsection{E516}

\begin{verbatim}
ENGR-E 516  ENGINEERING CLOUD COMPUTING (3 CR)
              ***** RSTR     ARR             ARR    ARR       Von Laszewski G
                 Above class taught online
                 Above class open to graduates only
                 Discussion (DIS)
              33699 RSTR     11:15A-12:30P   F      I2 150    Von Laszewski G
\end{verbatim}

\begin{verbatim}
        ENGR-E 516  ENGINEERING CLOUD COMPUTING (3 CR)
              33909 RSTR     ARR             ARR    WB WEB    Von Laszewski G
                 This is a 100% online class taught by IU Bloomington. No
                 on-campus class meetings are required. A distance education
                 fee may apply; check your campus bursar website for more
                 information
                 Above class open to graduates only
\end{verbatim}

\subsection{I524 and E616}

These classes are identical. In this courses we will be focussing on
Advanced Cloud COmputing and Big Data Applications and Analytics.  It
covers the applications and technologies needed to process the
application data. It uses Clouds running Data Analytics
Collaboratively processing Big Data to solve real problems.

Intelligent Systems Engeneering:

\begin{verbatim}
	ENGR-E 616  ADVANCED CLOUD COMPUTING (3 CR)
              ***** RSTR     ARR             ARR    ARR       Von Laszewski G
                 Above class taught online
                 Above class open to graduates only
                 Discussion (DIS)
              33697 RSTR     09:30A-10:45A   F      I2 150    Von Laszewski G
\end{verbatim}

Info Residential:



\begin{verbatim}
INFO-I 524  BIG DATA SOFTWARE AND PROJECTS (3 CR) 
              ***** RSTR     ARR             ARR    ARR       Von Laszewski G          
                 Above class open to graduates only
                 Above class taught online
                 Discussion (DIS)
              13053 RSTR     09:30A-10:45A   M      I2 130    Von Laszewski G  
\end{verbatim}

\begin{WARNING}
The location and time of this class is going to change to Friday
09:30A-10:45A   room I2 150.  
\end{WARNING}

Info Online:

\begin{verbatim}        
        INFO-I 524  BIG DATA SOFTWARE AND PROJECTS (3 CR)
              13054 RSTR     ARR             ARR    ARR       Von Laszewski G          
                 Above class open to graduates only
                 This is a 100% online class taught by IU Bloomington. No
                 on-campus class meetings are required. A distance education
                 fee may apply; check your campus bursar website for more
                 information
\end{verbatim}

\subsection{E222}

Undergraduate:

In this course students will be familiarized with different specific 
applications and implementations of intelligent systems and their use 
in desktop and cloud solutions.

\begin{verbatim}
ENGR-E 222  INTELLIGENT SYSTEMS II (3 CR)
              ***** RSTR     02:30P-03:45P   TR     GY 436  Fox
                 Laboratory (LAB)
        E 222 : P - ENGR-E 221
              31434 RSTR     05:45P-06:35P   R      GY 447  Fox
                 Above class for  Intelligent Systems Engineering students

\end{verbatim}        



\section{Waitlist}\label{waitlist}

The waitlist contains students that are unable to enroll in a section of
a course. Students choose to add themselves to the waitlist. They are
not automatically added, but choose to do so intentionally based on the
status of the course. There are two reasons for students to be on the
waitlist. The first, and primary, reason is that the class is already at
the scheduled, maximum capacity. Since there are no seats available, the
student can elect to add themselves to the waitlist. The second reason
is that the students' own schedule has a time conflict. This occurs when
they are trying to enroll in a class that overlaps with the time of a
class they are already enrolled in.

Students are moved from the waitlist to the regular section during a
daily batch process, and not in real time. The process is not in
realtime because the registrar receives many requests to increase
capacity, decrease capacity, and change rooms. If the process were real
time there would be a catastrophe of conflicts.

Students are moved from the waitlist in chronological order that they
added themselves to the waitlist. If you are still on the waitlist there
are no spaces free, the batch process has not run for the day, or the
student in question has a schedule conflict.

Faculty are not able to selectively choose students from the waitlist.

How long does the waitlist process stay active?: The automated
processing of the waitlist ends on Thursday of the first week of class
At this time the waitlist will no longer be processed. 
As the residential class starts on Friday, this may cause
issues. Either talk to the department on Thursday or show up on
Friday. Most likeley there will be spaces left. 
Students on the waitlist at that time will remain on the waitlist, but remain there
until the student decides to change their registration. Students may not
do that, because they get assessed a change schedule fee.

Students tell me they still want to enroll after the first week of
classes. How do they do this?

Beginning Monday, after the first week of class students begin to use
the eAdd process to do a late addition of the course. The request is
routed to the professor of record on an eDoc and the faculty will be
notified via email. Faculty can deny or approve based on whatever
criteria they wish to apply. If the faculty member approves, the eDoc
is electronically forwarded to the Academic Operations office and we
will approve the late add \textbf{if the room capacity} allows the
addition, otherwise we must deny the addition because of fire marshal
regulations. Many times, there are seats in a
classroom/discussion/lab, but because other students have not
\emph{officially} dropped, enrollment is still at capacity.

After everything, a student that was unable to enroll in the class
attended all year and completed all course work as if they had enrolled.
Can the student get credit and can I give the student a grade?

Yes. There is a provision for a late registration - contact our office
if this occurs. Students will be assessed a tuition fee at the time of
late or retroactive registration.

\section{Auditing the class}\label{auditing-the-class}

We no longer allow students to audit I524, E516, and E616. The
motivation to not offer these classes for auditing are:

\begin{itemize}
\item Seating in the lecture room is limited and we want foster
  students that enroll full time first.

\item The best way to take the class is to conduct a project. As this can
not be achieved without taking the class full time and as auditing the
class does not provide the full value of the class, e.g. not more than
10\% of the class, we do not think it is useful to audit the class.

\item  Accounts and services have to be set up and require
  considerable resources that are not accessible to students that
  audit the class.

\end{itemize}


\section{Resource restrictions}

\begin{itemize}
\item It is not allowed to use our services for profit (e.g. just
  enrolling in the class to use our clouds).
\item In case of abuse of available compute time on our clouds the
  student is aware that we will terminate the computer account on our
  clouds and she may have to conduct the project on a public cloud or
  his own computer under her own cost. There will be no guarantee that
  cloud services we offer will be available after the semester is
  over.  Projects can be conducted as part of the class that do not
  require access to the cloud.
\end{itemize}

\section{Meeting Times}\label{meeting-times}

The classes are published online. Residential students at Indiana
University will participate in a discussion taking place at the
following time according to the information provided by the registrar.

\begin{itemize}
\item 09:30A-10:45A Friday I524/E616 residential, I2 150
\item  11:15A-12:30A Friday I516 residential, I2 150
\end{itemize}

The Monday class is moveing to Friday, if this is introducing a
conflict, let us know.

\section{Office Hours}\label{office-hours}

\begin{description}

\item[Online Students:] Online hours are prioritized for online students,
  residential students should attend the residential meetings. 

\item[Residential Students:] Residential students participate in the
  official meeting times. If additional times are required, they have
  to be done by appointment. Office hours will be announced
  publically. All technical office hours are public and can be
  attended by any student.

  Online houres are not an excusenot to come to the residential class.

  However Residential students can in addition to the residential
  class use the online student meeting times.  However, in that case
  online students will be served first. It is probably good to check
  into the zoom meeting and identify if the TA has time. They will be
  in zoom.

\end{description}

We suggest that you let the TA's know in piazza before you come, in order to make
sure they are at the office.

\begin{itemize}
\item Mon 6:00pm-7:00pm, 7:00pm-8:00pm, Gregor (online)
\item Tue TBD, Smith Research Center
\item Wed TBD, Smith Research Center
\item Thu TBD, Smith Research Center
\item Fri TBD, Smith Research Center
\item Sat TBD, Smith Research Center
\end{itemize}


If a meeting is needed with Gregor, this is done upon appointment
Tue-Thu 10am - 2:30pm. However, TA's will figure out if a meeting is needed.
Please prepare your technical questions ahead of time, and place them in Piazza
first. TA's and the class will try to answer them if possible

The link for joining the meeting on Zoom is posted in Piazza.

% \URL{https://iu.zoom.us/j/235405252}

\URL{TBD}

For more up-to-date details, refer to Piazza.

\section{Plagiarizm}

On teh first day of class you will need to read the information about
plagiarizm. If there are any questions about plagiarizm we require you
to take a course offered from the IU educational department.

\begin{WARNING}
  If we find cheating or plagirizm, your assignment will be receiving
  an {\em F}. This especially includes copying text without proper
  attribution. In addition you will be receiving an {\em F} for the
  appropriate time for the discussion points an assignment was issued,
  e.g. If a paper duration assignment is 4 weeks, you get for these
  four weeks no discussion points, meaning an {\em F}. Furthermore, we
  will follow IU policy and report your case to the dean of students
  who may elect to expell you form the university. Please understand
  that it is your doing and the instructors have no choice as to
  follow university policies. Do not blame the instructors. Excuses
  such as ``I missed the lecture on plagiarizm'', ``I forgit to
  include the original refrence as I ran out of time'', ``I did not
  understand what plagiarizm is'' do not count obvioulsy as we
  explicitly make the policies clear. This applies to all material
  prepared for class including assignments, excercises, code,
  tutorials, papers, and projects
\end{WARNING}

For more information on this topic please see:

\URL{https://studentaffairs.indiana.edu/student-conduct/misconduct-charges/academic-misconduct.shtml}