\FILENAME

\section{Course Information and Calendar}\label{course-information-and-calendar}

Information will change. Please, revisit frequently.

This class ends on Dec 15 and not earlier. Residential students are
required to attend all Friday classes including those held in Dec.
Missing such classes will result in grade reduction. We will implement a
strict policy. Only for medical reasons a class can be missed. The
reasoning for this is that residential students that did not attend
classes regularly tend to do not as well in this class as their
colleagues. As we want that you achieve your best we strongly advise to
attend and give with this policy an additional motivating factor.

\subsection{Notebook}\label{notebook}

Students are required to maintain a \emph{class notebook} in github in
which they summarize their weekly activities for this course. This
includes a self maintained list of which lecture material they viewed.

The notebook is maintained in the class github.com in your hid project
folder. It is a file called notebook.md that uses markdown as format.
While using md, you can either edit it locally and upload to github, or
directly edit it via the git hub Web editor.

You will be responsible to set up and maintaining the notebook.md and
update it accordingly. We suggest that you prepare sections such as:
Logistic, Theory, Practice, Writing and put in bullet form what you have
done into these sections during the week. We can see from the github
logs when you changed the notbook.md file to monitor progress. The
management of the notebook will be part of your discussion grade.

The format of the notebook is very specific and must follow these rules:

\begin{itemize}
\tightlist
\item
  use headings with the \# character and have a space after the \#
\item
  use bullets in each topic.
\item
  each bullet \textbf{must} have an individual date that is of the form
  mm/dd/yy. Please do not lump bullet points under s single date. Have
  each bullet point its own date
\item
  if you have done the activity in a period than add the second date to
  it mm/dd/yy - mm/dd/yy
\item
  If you refer to section numbers in your notebook, please aslo add the
  section title as the section numbers may change in case we need to add
  content
\end{itemize}

Please examine carefully the sample note book is available at:

\begin{itemize}
\tightlist
\item
  \url{https://raw.githubusercontent.com/bigdata-i523/sample-hid000/master/notebook.md}
\end{itemize}

This will render inmd as:

\begin{itemize}
\tightlist
\item
  \url{https://github.com/bigdata-i523/sample-hid000/blob/master/notebook.md}
\end{itemize}

\subsection{Calendar}\label{calendar}

This class is a full term class of 16 weeks.

\begin{itemize}
\item
  Aug 21, Mon, Class Begins (see web page)
\item
  Aug 25, Fri, 9am Setup communication pathways for the class (no
  extensions)

  \begin{itemize}
  \tightlist
  \item
    You must have created a github repository in our class repository.
    (Graded homework, no extensions)
  \item
    You must be in the class Piazza (Graded homework, no extensions).
  \item
    Why no extensions: if we can not communicate with you we can not
    conduct the class. Everyone must be in piazza and github timely.
    Instead of giving points we will deduct points from everyone that
    has not done it in time. Those that registered late have one week
    from the time they registered to complete this task.
  \end{itemize}
\item
  Sep 4, Mon, Labor Day

  \begin{itemize}
  \tightlist
  \item
    Good day to work on projects, computer setup
  \end{itemize}
\item
  Oct 9, 9am. Paper 1 due (Graded)
\item
  Oct 2 Computer Setup completed
\item
  Fall Break Oct 6 - Oct 8

  This is a good time to work ahead or catch up with things. We strongly
  advise to use this time wisely.
\item
  Auto W Sun, Oct 22

  \begin{itemize}
  \tightlist
  \item
    Oct 30 General programming assignments due (pass/fail)

    \begin{itemize}
    \tightlist
    \item
      results
    \item
      code quality
    \item
      documentation
    \end{itemize}
  \end{itemize}
\item
  Nov 6, 9am Paper 2 due (Graded)
\item
  Nov 19 - Nov 26, Thanksgiving lecture free time

  This is a good time to work ahead or catch up with things We strongly
  advise to use this time wisely. Projects and paper are due Dec 1. We
  will deduct 10\% of the grade if not completed by Dec 1.

  \begin{itemize}
  \tightlist
  \item
    Dec 4, 9am. Project and Paper due
  \item
    Dec 4, 9am. If not doing a project Extended paper due
  \end{itemize}

  Although the paper is due on Dec 1. we may grant that you continue to
  work on your paper based on a first review (upon approval).
\item
  Dec 1 - 10

  Work in group meetings with TAs on improving your papers and projects
\item
  Dec 11 - 14

  Work in group meetings with TAs on improving your papers and projects
\item
  Ends Fri, Dec 15
\end{itemize}

\subsection{Graded Assignment Due
Dates}\label{graded-assignment-due-dates}

\begin{itemize}
\tightlist
\item
  Paper 1 - Oct 9th 9am
\item
  Paper 2 - Nov 6th 9am
\item
  Final Project - Dec 4th, 9am
\end{itemize}

\subsection{Incomplete}\label{incomplete}

Incomplete's will receive a fractional Grade reduction. Example A will
become A-, A- will become B+, and so forth. There is enough time in the
course to complete all assignments without getting an incomplete.

Why do we have such a policy? As we teach state-of-the-art software this
software is subject to change, not only within the course, but also
after the course. As we may offer some services and only have access to
the TA's during the semester it is obvious that we like all class
projects and homework assignments to be completed within a semester.
Services that were offered during the semester may no longer be
available after the semester is over and could adversely effect your
planing. It will be in the students responsibility to identify such
services and provide alternatives if they become unavailable. We try
hard to avoid this but we can not guarantee it.

Furthermore once an incomplete is requested, you will have 10 month to
complete it. We will need 2 month to grade. No grading will be conducted
over breaks. This may effect those that require student loans. Please
plan ahead.

\subsection{Registration Information}\label{registration-information}

The folloing course numbers are sections of this class

\begin{itemize}
\tightlist
\item
  FA17-BL-ENGR-E534-36123
\item
  FA17-BL-ENGR-E534-36124
\item
  FA17-BL-INFO-I423-13993
\item
  FA17-BL-INFO-I423-13994
\item
  FA17-BL-INFO-I523-13308
\item
  FA17-BL-INFO-I523-13310
\end{itemize}

The official registration information can be found here:

\begin{itemize}
\tightlist
\item
  \url{http://registrar.indiana.edu/official-calendar/official-calendar-fall.shtml}
\end{itemize}

We summarize, but like to point out that the information here may have
changed. We advise to visit the official page. However important to note
is that all residential students meet:

\begin{verbatim}
09:30A-10:45A   Friday      I2 150 
\end{verbatim}

Engineering Residential:

\begin{verbatim}
ENGR-E 534  BIG DATA APPLICATIONS (3 CR)
      36123 RSTR     Von Laszewski G          up to 25
         Above class open to graduate engineering students only
         Above class taught online
         Discussion (DIS)
      36124 RSTR     09:30A-10:45A   F      I2 150    Von Laszewski G
      Above class meets in the Smith Research Center, 151E
\end{verbatim}

Informatics Graduate Residential:

\begin{verbatim}
INFO-I 523  BIG DATA APPLS & ANALYTICS (3 CR)
      *****          Von Laszewski G          up to 50
         Above class open to graduates only
         Above class taught online
         Discussion (DIS)
      13308          09:30A-10:45A   F      I2 150    Von Laszewski G
         Above class meets with INFO-I 423
\end{verbatim}

Informatics Graduate Online:

\begin{verbatim}
INFO-I 523  BIG DATA APPLS & ANALYTICS (3 CR)
I 523 : P - Data Science majors only
      13310 RSTR     Von Laszewski G          up to 90
         This is a 100% online class taught by IU Bloomington. No
         on-campus class meetings are required. A distance education
         fee may apply; check your campus bursar website for more
         information
         Above class for students not in residence on the Bloomington
         campus
\end{verbatim}

Informatics Undergraduate:

\begin{verbatim}
INFO-I 423  BIG DATA APPLS & ANALYTICS (3 CR)
    CLSD ***** RSTR  Von Laszewski G          up to 10
         Above class open to undergraduates only
         Above class taught online
         Discussion (DIS)
    CLSD 13994 RSTR     09:30A-10:45A   F      I2 150    Von Laszewski G
         Above class meets with INFO-I 523
\end{verbatim}

\subsection{Waitlist}\label{waitlist}

The waitlist contains students that are unable to enroll in a section of
a course. Students choose to add themselves to the waitlist. They are
not automatically added, but choose to do so intentionally based on the
status of the course. There are two reasons for students to be on the
waitlist. The first, and primary, reason is that the class is already at
the scheduled, maximum capacity. Since there are no seats available, the
student can elect to add themselves to the waitlist. The second reason
is that the students' own schedule has a time conflict. This occurs when
they are trying to enroll in a class that overlaps with the time of a
class they are already enrolled in.

Students are moved from the waitlist to the regular section during a
daily batch process, and not in real time. The process is not in
realtime because the registrar receives many requests to increase
capacity, decrease capacity, and change rooms. If the process were real
time there would be a catastrophe of conflicts.

Students are moved from the waitlist in chronological order that they
added themselves to the waitlist. If you are still on the waitlist there
are no spaces free, the batch process has not run for the day, or the
student in question has a schedule conflict.

Faculty are not able to selectively choose students from the waitlist.

How long does the waitlist process stay active?: The automated
processing of the waitlist ends on THURSDAY morning, August 24th. At
this time the waitlist will no longer be processed. Students on the
waitlist at that time will remain on the waitlist, but remain there
until the student decides to change their registration. Students may not
do that, because they get assessed a change schedule fee.

Students tell me they still want to enroll after the first week of
classes. How do they do this?

Beginning Monday, August 28th students begin to use the eAdd process to
do a late addition of the course. The request is routed to the professor
of record on an eDoc and the faculty will be notified via email. Faculty
can deny or approve based on whatever criteria they wish to apply. If
the faculty member approves, the eDoc is electronically forwarded to the
Academic Operations office and we will approve the late add \textbf{if
the room capacity} allows the addition, otherwise we must deny the
addition because of fire marshal regulations. Many times, there are
seats in a classroom/discussion/lab, but because other students have not
\emph{officially} dropped, enrollment is still at capacity.

After everything, a student that was unable to enroll in the class
attended all year and completed all course work as if they had enrolled.
Can the student get credit and can I give the student a grade?

Yes. There is a provision for a late registration - contact our office
if this occurs. Students will be assessed a tuition fee at the time of
late or retroactive registration.

\subsection{Auditing the class}\label{auditing-the-class}

\begin{description}
\item[degree seeking students have preference to take this]
class. If the class is full and degree seeking students are on the
waiting list auditing and non-degree students will have to wait till all
others have been able to enroll. IF space permits only than auditing and
non degree students can enroll.
\end{description}

In case you like to audit the class or like to take it as part of a
non-degree program the following applies:

Participation in the class is approved for non degree student and
students that like to audit the class under the following conditions:

\begin{enumerate}
\tightlist
\item
  Due to limited space enrollment in the residential class is not
  allowed. The class must be taken online.
\item
  To assure that the full value of the class is applied all homework
  (graded and ungraded) must be conducted, however we will not grade
  your assignments.
\item
  For non degree students and students that audit the class an
  incomplete will not be allowed. The class homework must be completed
  in the semester as some software and services will only be accessible
  during the semester.
\item
  Accounts and services cannot be shared and will be disabled once the
  class is over.
\item
  It is not allowed to use our services for profit (e.g. just enrolling
  in the class to use our clouds).
\item
  In case of abuse of available compute time on our clouds the student
  is aware that we will terminate the computer account on our clouds and
  she may have to conduct the project on a public cloud or his own
  computer under her own cost. There will be no guarantee that cloud
  services we offer will be available after the semester is over.
  Projects can be conducted as part of the class that do not require
  access to the cloud.
\item
  There will not be any recommendation letter be provided based on
  auditing the class from the instructors. If a certificat of attendance
  is needed please contact the university administration.
\end{enumerate}

\subsection{Meeting Times}\label{meeting-times}

The classes are published online. Residential students at Indiana
University will participate in a discussion taking place at the
following time according to the information provided by the registrar.

\begin{itemize}
\tightlist
\item
  09:30A-10:45A Friday I523/I423/E534 other residential, I2 150
\end{itemize}

The Monday class has been moved to Friday

\subsection{Office Hours}\label{office-hours}

\begin{description}
\item[Residential Students:]
Residential students participate in the official meeting times. If
additional times are required, they have to be done on appointment. As
online hours are reserved for online students, residential students
should not use them till not all questions have been answered by online
students.

\begin{itemize}
\tightlist
\item
  Mon 4-5 PM (Miao in Smith Research Center)
\item
  Tue 3-4 PM (Saber in Smith Research Center)
\item
  Wed 10-11 AM (Juliette in Informatics East Cafe)
\item
  Thu 3-4pm (Juliette in Smith Research Center)
\item
  Fri 3-4pm (Saber, Miao in Smith Research Center)
\end{itemize}

We suggest that you let the TA's know before you come, in order to make
sure there are not too many people coming at the same time.

Residential students can also use the online student meeting times.
However, in that case online students will be served first. It is
probably good to check into the zoom meeting and identify if the TA has
time. They will be in zoom.

If a meeting is needed with Gregor, this is done upon appointment
Tue-Thu 10am - 2:30pm. TAs will figure out if a meeting is needed.
Please prepare your questions ahead of time, and place them in Piazza
first.
\item[Online Students:]
Using a doodle poll from the online students, we have identified the
following times for the online meetings:

\begin{itemize}
\tightlist
\item
  Mon 6-7 PM EST (Gregor, Juliette)
\item
  Mon 7-8 PM EST (Gregor, Juliette)
\item
  Fri 4-5 PM EST (Saber, Miao)
\item
  Sat 10-11 AM EST (Saber, Miao)
\item
  Wed 5-6 PM EST (Juliette, no meeting notes for this meeting time,
\end{itemize}

The link for joining the meeting on Zoom is
\url{https://iu.zoom.us/j/235405252}

If a meeting is needed with Gregor, this is done upon appointment
Tue-Thu 10am - 2:30pm. TAs will figure out if a meeting is needed.
Please prepare your questions ahead of time, and place them in Piazza
first.

Online students can also use the residential student meeting times.
However, in that case residential students will be served first. It is
probably good to check if the TA is free. They will be in zoom.

For more up-to-date details, refer to Piazza.
\end{description}
