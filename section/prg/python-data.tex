\chapter{Data Management}

\FILENAME

Obviously when dealing with big data we may not only be dealing with
data in one format but in many different formats. It is important that
you will be able to master such formats and simlessly integerat in
your analysis. Thus we provide some simple examples on which different
data formats exist and how to use them.

\section{Formats}

\subsection{Pickle}

Python pickle allows you to save data in a python native format into a file
that can later be read in by other programs. However, the data format
may not be portable among different python versions thus the format is
often not suitable to store information. INstead we recommend for
standrad data to use either json or yaml.

\begin{verbatim}
import pickle

flavor = { "small": 100, 
           "medium": 1000,
           "large": 10000 }

pickle.dump( flavor, open( "data.p", "wb" ) )

\end{verbatim}

To read it back in use

\begin{verbatim}
flavor = pickle.load( open( "data.p", "rb" ) )
\end{verbatim}

\subsection{Text Files}

\begin{verbatim}
content = open(“filename.txt”, “r”).read() 
\end{verbatim}

\begin{verbatim}
with open('filename.txt','r') as file:
    output = file.read()
\end{verbatim}

To split up the files into an array you can do

\begin{verbatim}
with open('filename.txt','r') as file:
    lines = file.read().splitlines()
\end{verbatim}


In case the file is too big you will want to read the file line by
line:

\begin{verbatim}
lines = open('filename.txt','r').readlines()
\end{verbatim}


\subsection{CSV Files}

\begin{verbatim}
import csv
with open(‘data.csv’, ‘rb’) as f:
   contents = csv.reader(f)
for row in content:
    print row
\end{verbatim}

using pandas

\begin{verbatim}
import pandas as pd
df = pd.read_csv("example.csv") 
\end{verbatim}

\subsection{Excel spread sheets}

\begin{verbatim}
import pandas as pd
filename = 'data.xlsx'
data = pd.ExcelFile(file)
df = data.parse('Sheet1')
\end{verbatim}

\subsection{YAML}

\begin{verbatim}
import yaml
with open('data.yaml', 'r') as f:
    content = yaml.load(f)
\end{verbatim}

\subsection{JSON}

\begin{verbatim}
import json
with open('strings.json') as f:
    content = json.load(f)
\end{verbatim}

\subsection{XML}

Please contribute a section.

\subsection{RDF}

\begin{verbatim}
from rdflib.graph import Graph
g = Graph()
g.parse("filename.rdf", format="format")
for entry in g:
   print(entry)
\end{verbatim}

\subsection{PDF}

The Portable Document Format (PDF) has been made available by Adobe
Inc. royalty free. This has enabled PDF to become a world wide adopted
format that also has been standardized in 2008 (ISO/IEC 32000-1:2008,
\url{https://www.iso.org/standard/51502.html}).  A lot of research is
published in papers making PDF one of the defacto standards for
publishing. However, PDF is difficult to parse and is focused on high
quality pint putput instead of data representation. Nevertheless,
tools to manipulate PDF exist:

\begin{description}
\item[PDFMiner] \url{https://pypi.python.org/pypi/pdfminer/} allows
  the simple translation of PDF into text that than can be further
  mined. The manual page helps to demonstrate some examples
  \url{http://euske.github.io/pdfminer/index.html}.

\item[pdf-parser.py]
  \url{https://blog.didierstevens.com/programs/pdf-tools/} parses pdf
  documents and identifies some structural elements that can than be
  further processed.

\end{description}

If you know about other tools, let us know.


\subsection{HTML}

Beautiful soup

please contribute a section

\subsection{ConfigParser}

\URL{https://pymotw.com/2/ConfigParser/}

\subsection{ConfigDict}

\URL{https://github.com/cloudmesh/cloudmesh.common/blob/master/cloudmesh/common/ConfigDict.py}

\section{Encryption}

Often we need to protect the information stored in a file. This is
achieved with encryption. There are many methods of supporting
encryption and even if a file is encrypted it may be target to
attacks. Thus it is not only important to encrypt data that you do not
want others to se but also to make sure that the system on which the
data is hosted is secure. This is especially important if we talk
about big data having a potential large effect if it gets into the
wrong hands. 

To illustrate one type of encryption that is non trivial we have
chosen to demonstrate how to encrypt a file with an ssh key. In case
you have openssl installed on your system, this can be achieved as follows.


\begin{verbatim}
#! /bin/sh

# Step 1. Creating a file with data
echo "Big Data is the future." > file.txt

# Step 2. Create the pem 
openssl rsa -in ~/.ssh/id_rsa -pubout  > ~/.ssh/id_rsa.pub.pem

# Step 3. look at the pem file to illustrate how it looks like (optional)
cat ~/.ssh/id_rsa.pub.pem

# Step 4. encrypt the file into secret.txt
openssl rsautl -encrypt -pubin -inkey ~/.ssh/id_rsa.pub.pem -in file.txt -out secret.txt

# Step 5. decrypt the file and print the contents to stdout
openssl rsautl -decrypt -inkey ~/.ssh/id_rsa -in secret.txt
\end{verbatim}

Most important here are Step 4 that encrypts the file and Step 5 that
decrypts the file. Using the Python os module it is straight forward
to implement this. However, we are providing in cloudmesh a convenient
class that makes the use in python very simple.

\begin{verbatim}
from cloudmesh.common.ssh.encrypt import EncryptFile

e = EncryptFile('file.txt', 'secret.txt')
e.encrypt()
e.decrypt()
\end{verbatim}

In our class we initialize it with the locations of the file that is
to be encrypted and decrypted. To initiate that action just call the
methods \verb|encrypt| and \verb|decrypt|.


\subsection{Exercise}

\begin{description}

\item[Encryption.1] Test out the shell script to replicate how this
  example works

\item[Encryption.2] Test out the cloudmesh encryption class

\item[Encryption.3] What other encryption methods exist. Can you
  provide an example and contribute to the section?

\item[Encryption.4] What is the issue of encryption that make it
  challenging for Big Data
 
\item[Encryption.5] Given a test dataset with many files text files,
  how long will it take to encrypt and decrypt them on various
  machines. Write a benchmark that you test. Develop this benchmark as
  a group, test out the time it takes to execute it on a variety of
  platforms.




\end{description}