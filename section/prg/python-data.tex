\FILENAME

\chapter{Data Formats}

Obviously when dealing with big data we may not only be dealing with
data in one format but in many different formats. It is important that
you will be able to master such formats and simlessly integerat in
your analysis. Thus we provide some simple examples on which different
data formats exist and how to use them.

\section{Pickle}

Python pickle allows you to save data in a python native format into a file
that can later be read in by other programs. However, the data format
may not be portable among different python versions thus the format is
often not suitable to store information. INstead we recommend for
standrad data to use either json or yaml.

\begin{verbatim}
import pickle

flavor = { "small": 100, 
           "medium": 1000,
           "large": 10000 }

pickle.dump( flavor, open( "data.p", "wb" ) )

\end{verbatim}

To read it back in use

\begin{verbatim}
flavor = pickle.load( open( "data.p", "rb" ) )
\end{verbatim}

\section{Text Files}

\begin{verbatim}
content = open(“filename.txt”, “r”).read() 
\end{verbatim}

\begin{verbatim}
with open('filename.txt','r') as file:
    output = file.read()
\end{verbatim}

To split up the files into an array you can do

\begin{verbatim}
with open('filename.txt','r') as file:
    lines = file.read().splitlines()
\end{verbatim}


In case the file is too big you will want to read the file line by
line:

\begin{verbatim}
lines = open('filename.txt','r').readlines()
\end{verbatim}


\section{CSV Files}

\begin{verbatim}
import csv
with open(‘data.csv’, ‘rb’) as f:
   contents = csv.reader(f)
for row in content:
    print row
\end{verbatim}

using pandas

\begin{verbatim}
import pandas as pd
df = pd.read_csv("example.csv") 
\end{verbatim}

\section{Excel spread sheets}

\begin{verbatim}
import pandas as pd
filename = 'data.xlsx'
data = pd.ExcelFile(file)
df = data.parse('Sheet1')
\end{verbatim}

\section{YAML}

\begin{verbatim}
import yaml
with open('data.yaml', 'r') as f:
    content = yaml.load(f)
\end{verbatim}

\section{JSON}

\begin{verbatim}
import json
with open('strings.json') as f:
    content = json.load(f)
\end{verbatim}

\section{XML}

Please contribute a section.

\section{RDF}

\begin{verbatim}
from rdflib.graph import Graph
g = Graph()
g.parse("filename.rdf", format="format")
for entry in g:
   print(entry)
\end{verbatim}

\section{PDF}

The Portable Document Format (PDF) has been made available by Adobe
Inc. royalty free. This has enabled PDF to become a world wide adopted
format that also has been standardized in 2008 (ISO/IEC 32000-1:2008,
\url{https://www.iso.org/standard/51502.html}).  A lot of research is
published in papers making PDF one of the defacto standards for
publishing. However, PDF is difficult to parse and is focused on high
quality pint putput instead of data representation. Nevertheless,
tools to manipulate PDF exist:

\begin{description}
\item[PDFMiner] \url{https://pypi.python.org/pypi/pdfminer/} allows
  the simple translation of PDF into text that than can be further
  mined. The manual page helps to demonstrate some examples
  \url{http://euske.github.io/pdfminer/index.html}.

\item[pdf-parser.py]
  \url{https://blog.didierstevens.com/programs/pdf-tools/} parses pdf
  documents and identifies some structural elements that can than be
  further processed.

\end{description}

If you know about other tools, let us know.


\section{HTML}

Beautiful soup

please contribute a section

\subsection{ConfigParser}

\URL{https://pymotw.com/2/ConfigParser/}

\subsection{ConfigDict}

\URL{https://github.com/cloudmesh/cloudmesh.common/blob/master/cloudmesh/common/ConfigDict.py}