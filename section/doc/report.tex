\FILENAME

\section{Report Format}\label{report-format}

Although we provide \textbf{trivial} but detailed report format
requirements, we observed over the years that some students still asked
us can I make my report shorter, or can i use a different format? The
answer to these questions is \textbf{no}. Furthermore, we observed that
the same students than went ahead and played with the formating and
introduced empty lines, increased tables or figures, or worse modified
the fontsize to circumvent the page limit requirement.

Thus we have adopted a much simpler approach that is easy to summarize

\begin{enumerate}
\def\labelenumi{\arabic{enumi}.}
\tightlist
\item
  We provide you with a \textbf{high quality} report template format
  that you must not change and is used by millions of researchers.
\item
  All references must be managed with jabref as reference management
  tool and must be provided in addition to the document.
\item
  If your document does not follow the format or we find that you have
  modified the style of the template we provide will return the document
  without review.
\item
  It is in the students responsibility to use the template format from
  the beginning on. In fact, our assignments will use the template for
  all assignments and not just your term paper or term report.
\end{enumerate}

The template for the report is available from:

\begin{itemize}
\tightlist
\item
  \url{https://github.com/cloudmesh/classes/tree/master/docs/source/format/report}
\end{itemize}

Convenient compressed files are available at

\begin{itemize}
\tightlist
\item
  \url{https://github.com/cloudmesh/classes/tree/master/docs/source/format/report.tar.gz}
\item
  \url{https://github.com/cloudmesh/classes/tree/master/docs/source/format/report.zip}
\end{itemize}

You have two choices. A good one and a bad one.

The good choice is to use the LaTeX template and write your document in
LaTeX. The bad one is to use the Word template and write the document in
Word. Both templates are included in our git repository.

Hence, it is in your best interest to use LaTeX. The good news is that
we have made it simple for you to use it. Furthermore, you are allowed
to use online services. An example report in PDF format is available:

\begin{itemize}
\tightlist
\item
  \href{https://github.com/cloudmesh/classes/blob/master/docs/source/format/report/latex/report.pdf}{report.pdf}
\end{itemize}

We provide a very simple
\href{https://github.com/cloudmesh/classes/blob/master/docs/source/format/report/latex/Makefile}{Makefile}
that allows you to do editing with immediate preview as documented in
the LaTeX lesson. Due to LaTeX being a \textbf{trivial} ASCII based
format and having a superior bibliography management you will save
yourself many hours of work that you will face while fighting with Word.
We got feedback from those that tried it and they thanked us later.
Furthermore, in case you are in a team, you can use either git while
collaboratively developing the LaTeX document, use sharelatex, or
overleaf.

However, we allow you to use word under the following conditions:

\begin{enumerate}
\def\labelenumi{\arabic{enumi}.}
\tightlist
\item
  You accept the risk that Word may crash and you may find yourself in
  last minute in the situation that you lost your work and your document
  is broken. We will not be sympathetic to this situation as we
  recommended that you use LaTeX.
\item
  You must use not only Endnote, but also jabref when managing your
  references, so you have to do the management of references twice. This
  is so that your document could be converted to LaTeX in case we think
  it suitable for publication in a conference or workshop.
\item
  You do not modify the theme.
\item
  All images and tables are placed at the end of the paper.
\item
  Git wil be used to submit all documents with regular updates.
\end{enumerate}

For LaTeX you will encounter a much more smooth experience.

\begin{enumerate}
\def\labelenumi{\arabic{enumi}.}
\tightlist
\item
  Your final document must be committed in git and as LaTeX is ASCII
  based you can do thous throughout the semester and have backups via
  git.
\item
  You will be using jabref to manage your bibliography and as LaTeX has
  build in support for bibliography management there is not much you
  need to pay attention to, all Format of the references is done for you
  in case you entered them correctly
\item
  You do not modify our theme.
\item
  All images and tables are placed at the end of the paper.
\item
  Git wil be used to submit all documents.
\item
  You are allowed to use sharelatex or overleave so you do not have to
  install LaTeX on your computer, but see 5. and the next paragraph.
\end{enumerate}

Whatever format you use, your final submission must be in \textbf{the
class} git. We will not review any documents stored on sharelatex or
overleaf or in any git repository not belonging to the class. Your final
submission will include the bibliography file(s) as a separate
document(s). All images must be placed in an images folder and submitted
in your repository with the originals. When using sharelatex or overleaf
you must replicate the directory layout carefully from our template and
include your final documents in git with a Makefile that can recreate
the document. It is in your responsibility that this works. We will
regenerate the document from source before we grade it. Thus it is not
sufficient to just check in the final PDF. The report must be spell
checked.

\begin{description}
\item[There will be \textbf{NO EXCEPTION} to this. We will not]
review your report if its submission is incomplete.
\end{description}

\subsection{Leverage parallel editing}\label{leverage-parallel-editing}

In most cases you will be able to work in groups on class projects. This
allows you to develop the report collaboratively. Here are some options:

\begin{enumerate}
\tightlist
\item
  LaTeX and git: THis option will likely safe you time as you can use
  jabref also for manageing collaborative bibliographies and
\item
  MS onedrive: It allows you to edit a word document in collaboration.
  We recommend that you use a local installed version of Word and do the
  editiong with that, rather than useing the online verison. The online
  editor has some bugs. See also (untested):
  \url{http://www.paulkiddie.com/2009/07/jabref-exports-to-word-2007-xml/},
  \url{http://usefulcodes.blogspot.com/2015/01/using-jabref-to-import-bib-to-microsoft.html}
\item
  Google Drive: google drive could be used to collaborate on text that
  is than pasted into document. he final document will not accept as
  google document. You must use the 2 column ACM template. We observed
  that students that use google docs lack structure and we no longer
  allow it as final document format. It also does not allow us to
  uniformly compare the documents between each other. It is easy to
  transfer it to LaTeX.
\end{enumerate}

\subsection{Timemanagement Tips}\label{timemanagement-tips}

Obviously taking a class takes time

\begin{enumerate}
\tightlist
\item
  It takes time to read the information
\item
  It takes time understand the information
\item
  It takes time to do the project
\item
  This will get you in trouble: \emph{There are still 10 weeks left till
  the project is due so let me start in 4 weeks \ldots{}}. Postponing
  the project till the last moment
\item
  Do not spend significant time on unimportant documentation and setup.
  Instead spend time to develop cmd5 comamnds and scripts that do these
  things automatically
\end{enumerate}

\subsection{Report Checklist}\label{report-checklist}

This partiald list may serve as a way to check if you follow the rules

\begin{enumerate}
\tightlist
\item
  Have you written the report in the specified format?
\item
  Have you included an acknowledgement section?
\item
  Have you included the report in git?
\item
  Have you specified the HID, names, and e-mails of all team members in
  your report. E.g. the Real Names that are registered in Canvas?
\item
  Have you included the project number in the report?
\item
  Have you included all images in native and PDF format in git in the
  images folder?
\item
  Have you added the bibliography file that you managed with jabref
\item
  In case you used word have you also provided the endnote file
\item
  Have you added an appendix describing who did what in the project or
  report?
\item
  Have you spellchecked the paper?
\item
  Are you useing \textbf{a} and \textbf{the} properly?
\item
  Have you made sure you do not plagiarize?
\item
  Have you not used phrases such as shown in the Figure below, but
  instead used as shown in Figure 3 when referring to the 3rd figure?
\item
  Have you capitalized ``Figure 3'', ``Table 1'', \ldots{} ?
\item
  Any figure that is not referred to explicitly in the text must be
  removed.
\item
  Are the figure captions bellow the figures and not on top. (Do not
  include the titles of the figures in the figure itself but instead use
  the caption or that information?
\item
  When using tables have you put the table caption on top?
\item
  Make the figures large enough so we can read the details. If needed
  make the figure over two columns?
\item
  Do not worry about the figure placement if they are at a different
  location than you think. Figures are allowed to float. If you want you
  can place all figures at the end of the report?
\item
  Are all figures and tables at the end?
\item
  Do not use the word ``I'' instead use we even if you are the sole
  author?
\item
  Do not use the phrase ``In this paper/report we show'' instead use
  ``We show''. It is not important if this is a paper or a report and
  does not need to be mentioned.
\item
  Do not artificially inflate your report if you are bellow the page
  limit and have nothing to say anymore.
\item
  If your paper limit is 12 pages but you want to hand in 120 pages,
  please check first with an instructor ;-)
\item
  Check in your current work of the report on a weekly basis to show
  consistent progress.
\item
  Please use the dedicated report format for class. It may not be the
  ACM or IEEE format, but may have some additions that make management
  of bibliographies easier. Do follow our instructions for
  bibliographies.
\item
  Do not use the characters \& \# \% in the paper if you use LaTeX. If
  you use them you prabably need a in front of them.
\item
  If you want to say and do not use \& but use the word and.
\item
  (I524) Is in your report directory a README.rst file in it as shown in
  the example project that we introduced you to?
\item
  (I523) you do not have to place a readme in your report or paper
  directories. Instead create a README.md in your hid or pid
  directories.
\end{enumerate}

If you observe something missing let us know.

In case you are allowed to use word The following applies in addition

\begin{enumerate}
\tightlist
\item
  Are you manageing your refernces in jabref and endnote (we need both)
\item
  Are you using the right template we have a special 2 column template
  for the class that is a modified version from the 2 column ACM
  template
\item
  Are you using build in numbered section management? MSWord has
  Sections that must be used
\item
  Are you using real bulleted lists in Word and not just a ``*'' or a
  ``-''?
\item
  Have you carelessly pasted and copied into the document without using
  proper formats. E.g. in MSWord this is a problem. You need to fix the
  format and use the build in format. Not that if you paste wrong you
  effect the format styles.
\item
  Have you created not only a docx document but also the PDF.
\item
  Make sure you use .docx and not .doc
\end{enumerate}

If you have other things to add, send them via piazza and we will add
them here.

\subsection{README.md}\label{readme.md}

For I523, Fall 2017, we will manage all papers via github.com. You will
be added to our github at

\begin{itemize}
\tightlist
\item
  \url{https://github.com/bigdata-i523}
\end{itemize}

and assigned an hid (homework index directory) directory with a unique
hid number for you. In addition, once you decide for a project, you will
aslso get a project id (pid) and a directory in which you place the
projects. Projects must not be placed in hid directories as they are
treated differently and a class proceedings is automatically created
based on your submission.

As part of the hid directory, you will need to create a README.md file
in it, that \textbf{must} follow a specific format. The good news is
that we have developed an easy template that with common sense you can
modify easily. The template is located at

\begin{itemize}
\tightlist
\item
  \url{https://raw.githubusercontent.com/bigdata-i523/sample-hid000/master/README.md}
\end{itemize}

As the format may have been updated over time it does not hurt to
revisit it and compare with your README.md and make corrections. It is
important that you follow the format and not eliminate the lines with
the three quotes. The text in the quotes is actually yaml. yaml is a
data format the any data scientist must know. If you do not, you can
look it up. However, if you follow our rules you should be good. If you
find a rule missing for our purpose, let us know. We like to keep it
simple and want you to fill out the \emph{template} with your
information.

Simple rules:

\begin{itemize}
\tightlist
\item
  replace the hid nimber with your hid number.
\item
  naturally if you see sample- in the directory name you need to delete
  that as your directory name does not have sample- in it.
\item
  do not ignore where the author is to be placed, it is in a list
  starting with a -
\item
  there is always a space after a -
\item
  do not introduce empty lines
\item
  do not use TAB and make sure your editor does not bay accident
  automatically creates tabs. This is probably the most frequent error
  we see.
\item
  do not use any : \& \_ in the attribute text including titles
\item
  an object defined in the README.md must have on a single type field.
  for example in the project section. Make sure you select only one type
  and delete the other
\item
  in case you have long paragraphs you can use the \textgreater{} after
  the abstract
\item
  Once you understood how the README.md works, please delete the comment
  section.
\end{itemize}

\subsection{README.rst (for I524, Spring
2017)}\label{readme.rst-for-i524-spring-2017}

In the directory that containes the report, please include the following
README.rst file. Without this file we will not review your document:

\begin{verbatim}
Title: The title of your paper (one line)

Author: The author s of the paper (one line)

HID: The HID of the authors in the order as specified in authors (one line)

PID: The PID of the paper (there will be exactly one)

E-mail: The e-mails of the authors in the order of the author list (one line)

Format: latex or word (specify one)
\end{verbatim}

Please note that all information has an empty line between them and all
information is stored in one line

This information is used to autogenerate the class proceedings.

\subsection{Exercise}\label{exercise}

\begin{description}
\item[Report.1:]
Install latex and jabref on your system
\item[Report.2:]
Check out the report example directory. Create a PDF and view it. Modify
and recompile.
\item[Report.4:]
Learn about the different bibliographic entry formats in bibtex
\item[Report.5:]
What is an article in a magazine? Is it really an Article or a Misc?
\item[Report.6:]
What is an InProceedings and how does it differ from Conference?
\item[Report.7:]
What is a Misc?
\item[Report.8:]
Why are spaces, underscores in directory names problematic and why
should you avoid using them for your projects
\item[Report.9:]
Write an objective report about the advantages and disadvantages of
programs to write reports.
\item[Report.10:]
Why is it advantageous that directories are lowercase have no underscore
or space in the name?
\end{description}
