\FILENAME

\section{Communicating Research in Other Ways}\label{communicating-research}

Naturally, writing papers is not the only way to communicate your
research with others. We find that today we see additional pathways
for cumminication includine blogs, twitter, facebook, e-mail, Web
pages, and electronic notebooks. Let us refisit some of them and
identify when they are helpful.

\subsection{Blogs}\label{blogs}

\begin{description}
\item[blog:]
noun, a regularly updated website or web page, typically one run by an
individual or small group, that is written in an informal or
conversational style.
\end{description}

Advantages:

\begin{itemize}

\item
  encourages spontaneous posts
\item
  encourages small short contributions
\item
  chronologically ordered
\item
  standard software exists to set up blogs
\item
  online services exists to set up blogs
\end{itemize}

Disadvantages:

\begin{itemize}

\item
  structuring data is difficult (some blog software support it)
\item
  not suitable for formal development of a paper
\item
  often lack of sophisticated track change features
\item
  no collaborative editing features
\end{itemize}

\subsection{Sphinx}\label{sphinx}

Sphinx (\url{http://www.sphinx-doc.org/}) is a tool that to create
integrated documentation from a markup language whlie.

Advantages:

\begin{itemize}

\item
  output formats: html, LaTeX, PDF, ePub
\item
  integrates well with directory structure
\item
  powerful markup language (reStructuredText)
\item
  can be hosted on github via github pages
\item
  can integare other renderers such as Markdown
\item
  automatic table of content, tebale of index
\item
  code documentation integration
\item
  search
\item
  written in python and using bash, so extensions and custom automation
  are possible
\end{itemize}

Disadvantage:

\begin{itemize}

\item
  requires compile step
\item
  When using markdown github can render individual page
\end{itemize}

Others:

\begin{itemize}

\item
  Read the Docs (\url{https://readthedocs.org/})
\item
  Doxygen (\url{http://www.stack.nl/~dimitri/doxygen/})
\item
  MkDocs (\url{http://www.mkdocs.org/})
\end{itemize}

\subsection{Notebooks}\label{notebooks}

\subsubsection{Jupyter}\label{jupyter}

The Jupyter Notebook (\url{http://jupyter.org/}) is an open-source web
application allowing users to create and share documents that contain
live code, equations, visualizations and explanatory text. Use cases
include data cleaning and transformation, numerical simulation,
statistical modeling, machine learning.

Advantages:

\begin{itemize}

\item
  Integrates with python
\item
  Recently other programming languages have been integrated
\item
  Allows experimenting with settings
\item
  Allows a form of literate programming while mixing documentation with
  code
\item
  automatically renders on github
\item
  comes with web service that allows hosting
\end{itemize}

Disadvantage:

\begin{itemize}

\item
  mostly encourages short documents
\item
  mark up language is limited
\item
  editing in ASCII is complex and Web editing is prefered
\end{itemize}

\subsubsection{Apache Zeppilin}\label{apache-zeppilin}

A Web-based notebook that enables data-driven, interactive data
analytics and collaborative documents with SQL, Scala and hadoop. It
integrates a web-based notebook with data ingestion, data exploration,
visualization, sharing and collaboration features to Hadoop and Spark.

Advantages:

\begin{itemize}

\item
  integration to various framework
\item
  Web framework
\item
  integration with spark, hadoop
\end{itemize}

Disadvantages:

\begin{itemize}

\item
  larger framework
\item
  must leverages existing deployments of spak, hadoop
\end{itemize}


\begin{comment}
\subsection{References}\label{references}

Collaboratories:

\begin{itemize}

\item
  Myers JD, TC Allison, SJ Bittner, BT Didier, M Frenklach, WH Green, YL
  Ho, J Hewson, WS Koegler, CS Lansing, D Leahy, M Lee, R McCoy, M
  Minkoff, S Nijsure, G von Laszewski, D Montoya, L Oluwole, CM
  Pancerella, R Pinzon, W Pitz, LA Rahn, B Ruscic, KL Schuchardt, EG
  Stephan, A Wagner, TL Windus, and C Yang. 2005. ``A Collaborative
  Informatics Infrastructure for Multi-scale Science.'' Cluster
  Computing 8(4):243-253.
\item
  Metadata in the Collaboratory for Multi-Scale Chemical Science Carmen
  Pancerella, John Hewson, Wendy Koegler, David Leahy, Michael Lee,
  Larry Rahn, Christine Yang, James D. Myers, Brett Didier, Renata
  McCoy, Karen Schuchardt, Eric Stephan, Theresa Windus, Kaizar Amin,
  Sandra Bittner, Carina Lansing, Michael Minkoff, Sandeep Nijsure,
  Gregor von Laszewski, Reinhardt Pinzon, Branko Ruscic, Al Wagner,
  Baoshan Wang, William Pitz, Yen-Ling Ho, David Montoya, Lili Xu,
  Thomas C. Allison, William H. Green, Jr., Michael Frenklach
  \url{http://dcpapers.dublincore.org/pubs/article/view/740/736}
\end{itemize}
\end{comment}