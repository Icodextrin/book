\chapter{Help}

\section{Compiling Cloudmesh Handbook}\label{s:help-compile-handbook}

You must fork the github repository from here :
\URL{https://github.com/cloudmesh/book}

Then you must only work in the master branch.  And you must install
latex in your machine to compile it. We assume you have installed
latex before trying this tutorial.

Our handbook is conveniently designed so that new sections can be
added easily. This includes the support of draft chapters that use the
same style as the handbook.

To facilitate the draft documentation phase, we created a \LaTeX~ file
called single.tex that has also a target in the \verb|Makefile|.

We will walk you through the process of adding a new section to the handbook

Let us assume the section is placed in the directory

\begin{lstlisting}
  section/machine-learning/kmeans.tex 
\end{lstlisting}

First, we must add this folder path to the Makefile. To not effect the
existing makefile and to test it we copy the original makefile to a
convenient one that you can remember such as

\begin{lstlisting}
  cp Makefile Makefile.albert
\end{lstlisting}

Here let us assume that Albert is the author who is going to
collaborate to handbook.

Then open the new Makefile.kmeans and add the following target to the
makefile:

\begin{lstlisting}
albert: dest 
	latexmk -jobname=albert $(FLAGS) -pvc -view=pdf albert
\end{lstlisting}  
%$

As the kmeans.tex file is in a new directory, we need to add the
directory name to the place where the LaTeX output files are placed.
Find in the dest target the alphabetical location where  this folder
belongs and add it.DO not remove the other folder names indicated by
...


\begin{lstlisting}
dest:
    ...
    mkdir -p dest/format
    mkdir -p dest/machine-learning
    mkdir -p dest/notebooks
    ...
\end{lstlisting}

Next, save the Makefile.albert.
Now copy the file called single.tex to albert.tex. 

\begin{lstlisting}
cp single.tex albert.tex
\end{lstlisting}

As we want to review the section we must however  add it to this latex
file with the section name included

\begin{lstlisting}
\include{section/machine-learning/kmeans}
\end{lstlisting}

You may select to outcomment other sections.
Now you can compile the file with 

Sample kmeans.tex file is shown below. And also make sure you include
all the images needed in the image folder inside the repository folder.
Do not create image folders within other folders, pleaese use the given
folder to include images. 

Sample kmeans.tex can be found here :
\URL{https://github.com/cloudmesh/book/blob/master/section/container/docker-kubernetes-installation.tex}


%%%%%%%%%%%%%

Sample albert.tex can be found here :
\URL{https://github.com/cloudmesh/book/blob/master/single.tex}

\begin{lstlisting}
make -f Makefile.albert albert
\end{lstlisting}


The above command will make the file, create a pdf file, and if you
have configured latexmk properly pop up the PDF in your configured PDF
browser. If not, you can find it in the dest folder at for our example
at 

\begin{lstlisting}
dest/kmeans-main.pdf
\end{lstlisting}


To configure your pdf browser you can for example use in ubuntu

\begin{lstlisting}
 sudo apt-get install xpdf
\end{lstlisting}


