\FILENAME

\section{Writing a Scientific Article or Conference Paper}
\label{S:writing}
\index{Writing}

An important part of any scientific research is to document it. This is
often done through scientific conferences or journal articles. Hence it
is important to learn how to prepare and submit such papers. Most
conferences accept typically the papers in PDF format but require the
papers to be prepared on MSWord or in LaTeX. While working with many
students in the past we noticed however that those students using Word
often spend unnecessarily countless hours on trying to make there papers
beautiful while actually violating the template provided by the
conference. Furthermore, we noticed that the same students had issues
with bibliography management. Instead of Word helping the student it
provided the illusion to be easier than LaTeX but when adding up the
time spend on the paper we found that LaTeX actually saved time. This
has been especially true with the advent of collaborative editing
services such as sharelatex \cite{www-sharelatex} and overleaf
\cite{www-overleaf}. 

In this section we provide you with a professional template that is used
for either system based on the ACM standard that you can use to write
papers. Naturally this will be extremely useful if the quality of your
research is strong enough to be submitted to a conference. We structure
this section as follows. Although we do not recommend that you use
MSWord for your editing of a scientific paper, we have included a short
section about it and outline some of its pitfalls that initially you may
not think is problematic, but has proven to be an issue with students.
Next we will focus on introducing you to LaTeX and showcasing you the
advantages and disadvantages. We will dedicate an entire section on
bibliography management and teach you jhow to use jabref which clearly
has advantages for us.

Having a uniform report format not only helps the students but allow
allows the comparison of paper length and effort as part of teaching a
course. We have added an entire section to this chapter that discusses
how we can manage a \emph{Class Proceedings} form papers that are
contributed by teams in the class.

\subsection{Professional Paper Format}\label{professional-paper-format}

The report format we suggest here is based on the standard ACM
proccedings format. It is of very high quality and can be adapted for
your own activities. Moreover, it is possible to use most of teh text to
adapt to other formats in case the conference you intend to submit your
paper to has a different format. The ACM format is always a good start.

Important is that you do not need to change the template but you can
change some parameters in case you are not submitting the paper to a
conference but use it for class papers. Certainly you should not change
the spacing or the layout and instead focus on writing content. As for
bibliography management we recommend you use jabref which we will
introduce in Section \ref{??}.

We recommend that you carefully study the requirements for the report
format. We would nat want that your paper gets rejected by a journal,
conference or the class just because you try to modify the format or do
not follow the established publication guidlines.

The template we are providing is available from:

\URL{https://github.com/cloudmesh/classes/tree/master/docs/source/format/report}

Convenient compressed files are available at

\URL{https://github.com/cloudmesh/classes/tree/master/docs/source/format/report.tar.gz}
\URL{https://github.com/cloudmesh/classes/tree/master/docs/source/format/report.zip}

You will find in it a modified ACM proceedings templates for Word and
for LaTeX that has an identification box removed on the lower left hand
side of the first page. This is done for classes so that you have more
space to write. In case you must submit to a conference you can use the
original ACM template. This template can be found at

\subsection{Submission Requirements}\label{submission-requirements}

Although the initial requirement for some conferences or journals is the
document PDF, in many cases you must be prepared to provide the source
when submitting to the conference. This includes the submission of the
original images in an images foder. You may ba asked to package the
document into a folder with all of its sources and submit to the
conference for professional publication.

\subsection{Microsoft Word vs. \LaTeX}\label{microsoft-word}

Microsoft word will provide you with the initial impression that you
will safe lots of time writing in it while you see the layout of the
document. This will be initially true, but once you progress to the
more challenging parts and later pages such as image menagement and
bibliography management you will see some issues. Thes include that
figure placement in Word need sto be done just right in order for images
to be where they need. We have seen students spending hours with the
placement of figures in a paper but when they did additional changes the
images jumped around and were not at the place where teh students
expected them to be. So if you work with images, make sure you
understand how to place them. Also always use relative caption counters
so that if an image gets placed elsewhere the counter stays consistent.
So nefer use just the number, but a reference to the figure when referring
to it. Recently a new bibliography management system was added to Word.
However, however it is not well documented and the references are placed
in the system bibliography rather than a local managed bibliography.
This mah have severe consequences when working with many authors on a
paper. The same is true when using Endnote. We have heard in many
occasions that the combination of endnote and Word destroyed documents.
You certainly do not want that to happen the day before your deadline.
Also in classes we observed that those using LaTeX deliver better
structured and written papers as the focus is on text and not beautiful
layout.

For all these reasons we do not recommend that you use Word.

In LaTeX where we have an easier time with this as we can just ignore
all of these issues due to relative good image placement and excellent
support for academic reference management. Hence, it is in your best
interest to use LaTeX. The information we provide here will make it easy
for you to get started and write a paper in no time as it is just like
filling out a form.

\subsection{Working in a Team}\label{working-in-a-team}

Today research is done in potentially large research teams. This also
include the writing of a document. There are multiple ways this is done
these days and depends on the system you chose.

In MSWord you can use skydrive, while for LaTeX you can use sharelatex
and overleaf. However, in many cases the use of github is possible as
the same groups that develop the code are also familiar with github.
Thus we provide you here also with the introduction on how to write a
document in github while group members can contribute.

Here are the options:

\begin{description}

\item [LaTeX and git:] This option will likely safe you time as you can use
  jabref also for managing collaborative bibliographies and
\item [sharelatex:] an online tool to write latex documents
\item [overleaf:] an online tool to write latex documents
\item [MS onedrive:] It allows you to edit a word document in collaboration.
  We recommend that you use a local installed version of Word and do the
  editing with that, rather than using the online version. The online
  editor has some bugs. See also (untested):
  \url{http://www.paulkiddie.com/2009/07/jabref-exports-to-word-2007-xml/},
  \url{http://usefulcodes.blogspot.com/2015/01/using-jabref-to-import-bib-to-microsoft.html}
\item [Google Drive:] google drive could be used to collaborate on text that
  is than pasted into document. However it is just a starting point as
  it does not support typically the format required by the publisher.
  Hence at one point you need to swithc to one of the other systems.
\end{description}

\subsection{Timemanagement}\label{timemanagement}

Obviously writing a paper takes time and you need to car-fully make sure
you devote enough time to it. The important part is that the paper
should not be an after thought but should be the initial activity to
conduct and execute your research. Remember that

\begin{enumerate}

\item  It takes time to read the information
\item  It takes time understand the information
\item  It takes time to do the research

\end{enumerate}

For deadlines the following will get you in trouble:

\begin{enumerate}

\item
  \emph{There are still 10 weeks left till the deadline, so let me start
  in 4 week \ldots{}}. Procrastination is your worst enemy.
\item
  If you work in a team that has time management issues address them
  immediately
\item
  Do not underestimate the time it takes to prepare the final submission
  into the submission system. Prepare automated scripts that can deliver
  the package for submission in minutes rather than hours by hand.
\end{enumerate}

\subsection{Paper and Report Checklist}\label{paper-checklist}
\index{Writing!Checklist}

In this section we summarize a number of checks that you may perform to
make sure your paper is properly formatted and in excellent shape.
Naturally this list is just a partial list and if you find things we
should add here, let us know.

\begin{itemize}[label=$\Box$]

\item Have you written the report in the specified format?

\item Have you included an acknowledgement section?

\item Have you included the paper in the submission system (In our
  case git). This includes all images, bibliography files and other
  material that is needed to build the paper from scratch?

\item Have you added the bibliography file that you managed (In our
  case jabref to make it simple for you)?

\item Have you specified proper identification in the submission
  system for your submission.  This is typically a form or ASCII text
  that needs to be filled out and follows a very particular format.
  In our case it is a README.md file that ncludes a homework ID, names
  of the authors, and e-mails)?

\item In case you used word have you also provided the jabref?

\item In case of a class and if you do a multi-author paper, have you
  added a work-breakdown section in the appendix describing who did
  what in the paper?

\item IN case you have an appenix it is included fater the bibliography

\item Have you spellchecked the paper?

\item Have you grammar chacked the paper?

\item Are you using \textbf{a} and \textbf{the} properly?

\item Have you made sure you do not plagiarize?

\item Is the title properly capitalized?

\item Have you not used phrases such as shown in the Figure below, but
  instead used as shown in Figure 3 when referring to the 3rd figure?
  Numbers in LaTeX are done with \verb|\label{}| after captions and
  \verb|\ref{}| in the text (See examples in the LaTeX section). In
  word you must use relative numbering.

\item Have you capitalized ``Figure 3'', ``Table 1''?

\item Have you removed any figure that is not referred explicitly in
  the text. E. ech figure needs a text such as  {\em As shown in
    Figure ..} or similar.

\item Are the figure captions bellow the figures and not on top. (Do
  not include the titles of the figures in the figure itself but
  instead use the caption or that information?

\item When using tables have you put the table caption above the table?

\item When using image have you put the table caption bellow the image?

\item Make the figures large enough so we can read the details. If
  needed make the figure over two columns? 

\item Do not worry about the figure placement if they are at a
  different location than you think. Figures are allowed to float. In
  many submissions you may have to place all figures at the end of the
  paper so you can focus on content rather than placing figures. In
  addition it will help you to refer to the figures by index.

\item Are all figures and tables at the end?

\item In case you copied a figure from another paper you need to ask
  for copyright permission. In case of a class paper you \textbf{must}
  include a reference to the original at the end of the figure caption.

\item Do not use the word ``I'' instead use ``we'' even if you are the
  sole author.

\item Do not use the phrase ``In this paper/report we show'' instead
  use ``We show''. It is not important if this is a paper or a report
  and does not need to be mentioned.

\item Do not artificially inflate your paper if you are bellow the
  page limit and have nothing to say anymore.

\item If your paper limit is 12 pages but you want to hand in 120
  pages, please check first ;-) If your page limoit is 2 pages but you
  hand in 4 thats is no issue.

\item Do not use the characters \& \# \% \_  in the paper if you use
  LaTeX. If you use them you probably need a bakslash in front of them.

\item Latex uses double single open quotes and double single closed
  quotes for quotes. Have you made sure you replaced them?

  When using quotes in LaTeX, do not use the double quote but instead
  use two single quotes such as \verb|``This is a quote''|. THis will
  place the proper quotes in the text. To only use the quotes when you
  literraly quote from other papers. Never use a quote to emphasize a
  thext. For that you use \verb|{\em this is emphasize}| resulting in
  {\em This is emphasized}.

\item If you want to say and do not use \& but use the word {\em
    and}. If you need tou use it be reminded to write it as \verb|\&|


\item Pasting and copying from the Web often results in non ascii
  characters to be used in your text, please remove them and replace
  accordingly. This includes some form of -- that you may see showing
  up as fi in pdf

\item Is your Abstract not a proposal? Abstracts are no proposals, e.g
  This paper intends to show .... If the paper intends to show you are
  still in the draft phase of the paper. However, if you say We sow
  ... That would be good. Let us just assume you intended to show
  something but did not achieve then you can say We intended to show
  this but we showed it was not possible. As you can see not only the
  intention is communicated, but the result. If you just focuss on the
  intent that's just a proposal and is not a proper abstract.

\item Are you not using the word paper in your writeup?  Abstracts and
  the entire paper should not have the word paper in it.

\item If your paper is an introduction or overview paper, please do
  not assume the reader to be an expert. Provide enough material for
  the paper to be useful for an introduction into the topic. 

\item Are you refernces correct? References to a paper are no
  afterthought, they should be properly cited. Use jabref and make
  sure the citation type of the refernce is correct and fill out as
  many fields as you can. Some jouranls and conferences have for
  example special requirements that go beyond the requirements of for
  example jabref. One example is that maky conferences require you
  that wne you cite papers form another conference to augment the
  conferences not only with the location where the conference took
  place, but also with the dates the conference took
  place. Unfortunately, this is information that is often only
  avalable through additional google quesies and many refernce entries
  you find in the internet do not have this information readily
  available.

\end{itemize}

In case of a class

\begin{itemize}[label=$\Box$] 

\item Check in your current work of the paper on a weekly basis to
  show consistent progress.

\item Please use the dedicated report format for class. It may not be
  the ACM or IEEE format, but may have some additions that make
  management of bibliographies easier. Do follow our instructions for
  bibliographies.

\end{itemize}

In case you are allowed to use word in class, such as the one we teach
at IU, the following applies in addition:

\begin{itemize}[label=$\Box$] 

\item Are you managing your references in jabref and endnote (we need
  both)
\item Are you using the right template we have a special 2 column
  template for the class that is a modified version from the 2 column
  ACM template
\item Are you using build in numbered section management? MSWord has
  Sections that must be used
\item Are you using real bulleted lists in Word and not just a ``*''
  or a ``-''?
\item Have you carelessly pasted and copied into the document without
  using proper formats. E.g. in MSWord this is a problem. You need to
  fix the format and use the build in format. Not that if you paste
  wrong you effect the format styles.
\item Have you created not only a docx document but also the PDF.

\item Make sure you use .docx and not .doc

\end{itemize}

If you observe something missing let us know.

\subsection{Example Paper}\label{example-paper}

An example report in PDF format is available:

\begin{itemize}

\item
  \href{https://github.com/cloudmesh/classes/blob/master/docs/source/format/report/latex/report.pdf}{report.pdf}
\end{itemize}

\subsection{Creating the PDF from LaTeX on your
Computer}\label{creating-the-pdf-from-latex-on-your-computer}

Latex can be easily installed on any computer as long as you have
enough space. Furthermore if your machine can execute the make command
we have provided in the standard report format a simple
\href{https://github.com/cloudmesh/classes/blob/master/docs/source/format/report/latex/Makefile}{Makefile}
that allows you to do editing with immediate preview as documented in
the LaTeX lesson.

\subsection{Class Specific README.md}\label{class-specific-readme.md}

For the class we will manage all papers via github.com. You will be
added to our github at

\URL{https://github.com/bigdata-i523}

and assigned an hid (homework index directory) directory with a unique
hid number for you. In addition, once you decide for a project, you will
aslso get a project id (pid) and a directory in which you place the
projects. Projects must not be placed in hid directories as they are
treated differently and a class proceedings is automatically created
based on your submission.

As part of the hid directory, you will need to create a README.md file
in it, that \textbf{must} follow a specific format. The good news is
that we have developed an easy template that with common sense you can
modify easily. The template is located at

\URL{https://raw.githubusercontent.com/bigdata-i523/sample-hid000/master/README.md}

As the format may have been updated over time it does not hurt to
revisit it and compare with your README.md and make corrections. It is
important that you follow the format and not eliminate the lines with
the three quotes. The text in the quotes is actually yaml. yaml is a
data format the any data scientist must know. If you do not, you can
look it up. However, if you follow our rules you should be good. If you
find a rule missing for our purpose, let us know. We like to keep it
simple and want you to fill out the \emph{template} with your
information.

Simple rules:

\begin{itemize}

\item
  replace the hid nimber with your hid number.
\item naturally if you see sample- in the directory name you need to

  delete that as your directory name does not have sample- in it.
\item do not ignore where the author is to be placed, it is in a list

  starting with a -
\item there is always a space after a -

\item do not introduce empty lines

\item do not use TAB and make sure your editor does not bay accident

  automatically creates tabs. This is probably the most frequent error
  we see.
\item do not use any : \& \_ in the attribute text including titles

\item an object defined in the README.md must have on a single type

  field.  for example in the project section. Make sure you select
  only one type and delete the other
\item in case you have long paragraphs you can use the \textgreater{}

  after the abstract
\item Once you understood how the README.md works, please delete the

  comment section.
\item Add a chapter topic that your paper belongs to

\end{itemize}

\subsection{Exercise}\label{exercise}

\begin{description}
\item[Report.1:]
Install latex and jabref on your system
\item[Report.2:]
Check out the report example directory. Create a PDF and view it. Modify
and recompile.
\item[Report.4:]
Learn about the different bibliographic entry formats in bibtex
\item[Report.5:]
What is an article in a magazine? Is it really an Article or a Misc?
\item[Report.6:]
What is an InProceedings and how does it differ from Conference?
\item[Report.7:]
What is a Misc?
\item[Report.8:]
Why are spaces, underscores in directory names problematic and why
should you avoid using them for your projects
\item[Report.9:]
Write an objective report about the advantages and disadvantages of
programs to write reports.
\item[Report.10:]
Why is it advantageous that directories are lowercase have no underscore
or space in the name?
\end{description}
