\chapter{Assignments}\label{c:222-assignments}

\section{Assignments E222}
\label{s:e222-assignment}
\label{s:e222-assignments}
\index{Assignments!E222}

\subsection{Bio Post}
\label{E:e222-bio}

\begin{exercise}\label{E:e222-bio-piazza}
{\bf Bio Post on Piazza.} Please post a formal bio to piazza
\end{exercise}

\begin{exercise} \label{E:e222-bio-googledocs}

  {\bf Bio Post in Google doc.} After you have posted it to piazza
  copy your updated formal bios into the following document.  Make
  sure you use 3rd person and stay formal. This is a formal
  bio. Comment on the effectiveness of using the cloud service for
  this task. A the end of the document. This assignment does not
  replace the post of the bio to piazza, it is used to gather all bios
  in one document and to evaluate if google docs is a good tool for
  this kind of task. Remember we have lots of students and google is
  used often just with small groups.
 
 \smallskip

 {\hfill \href{https://docs.google.com/document/d/1ejzlKYqC3dLac8WXVpcPQsJh1j4BDqRxxgGg1cFQbeQ/edit?usp=sharing}{E222 Link to google doc $\mapsto$}}

 \end{exercise}

\subsection{IU Google Services}
\label{E:e222-iu-google-services}

\begin{exercise}\label{E:e222-iu-google}

  {\bf IU Google Services:} This assignment is only for those that do
  not yet have access to our google documents This assignment does not
  have to be conducted for anyone that has access to our google
  documents for bios, and the technology list

  \begin{itemize}
 
  \item What is the difference between umail.iu.edu and iu.edu? Tip:
    the answer is provided in the IU knowledge base

  \item Login via the iu.edu account and not the umail.iu.edu account
    to google and open the document for the bio. Paste the bio into
    the document.

  \item Explain why IU has two different google services and
    logins. As we use cloud in this class, it is important to
    understand this and what implication this has. This is not just an
    assignment to give you access to the service, but to make you
    think why this works like this.

  \item Can you imagine a different way this ought to work?

  \end{itemize}

\end{exercise}

\section{Assignments E516, I524, E616}
\label{s:616-assignments}
\index{Assignments!E516}
\index{Assignments!E616}
\index{Assignments!I524}

\subsection{Bio Post}\label{a:616-bio}

\begin{exercise}\label{E:616-bio-piazza}
{\bf Bio Post on Piazza.} Please post a formal bio to piazza
\end{exercise}

\begin{exercise} \label{E:616-bio-googledocs}

  {\bf Bio Post in Google doc.} After you have posted it to piazza
  copy your updated formal bios into the following document.  Make
  sure you use 3rd person and stay formal. This is a formal
  bio. Comment on the effectiveness of using the cloud service for
  this task. A the end of the document. This assignment does not
  replace the post of the bio to piazza, it is used to gather all bios
  in one document and to evaluate if google docs is a good tool for
  this kind of task. Remember we have lots of students and google is
  used often just with small groups.
 
 \smallskip

 {\hfill \href{https://docs.google.com/document/d/1ejzlKYqC3dLac8WXVpcPQsJh1j4BDqRxxgGg1cFQbeQ/edit?usp=sharing}{E516 Link to google doc $\mapsto$}}

 \end{exercise}

\subsection{IU Google Services}
\label{E:e616-iu-google-services}

\begin{exercise}\label{E:616-iu-google}

  {\bf IU Google Services:} This assignment is only for those that do
  not yet have access to our google documents This assignment does not
  have to be conducted for anyone that has access to our google
  documents for bios, and the technology list

  \begin{itemize}
 
  \item What is the difference between umail.iu.edu and iu.edu? Tip:
    the answer is provided in the IU knowledge base

  \item Login via the iu.edu account and not the umail.iu.edu account
    to google and open the document for the bio. Paste the bio into
    the document.

  \item Explain why IU has two different google services and
    logins. As we use cloud in this class, it is important to
    understand this and what implication this has. This is not just an
    assignment to give you access to the service, but to make you
    think why this works like this.

  \item Can you imagine a different way this ought to work?

  \end{itemize}

\end{exercise}


\subsection{Big Data Collaboration}
\label{E:616-bigdata-collab}

\begin{exercise} \label{E:616-big-data-and-collaboration} {\bf Big
    data and collaboration.}The purpose of this assignment is
  multifold; test the ability of Google docs to be used in
  collaborative fashion by more than a small group and report on the
  experience. Good Things and bad things, learn on how to use Google
  docs with headings and table of contents learn how to gather
  resources quickly with hyperlinks to web resources or articles and
  translate them into formal academic references. Most importantly
  convey some very important feature of big data.Contribute this into
  the handbook for everyone's benefit (done by TAs).  \smallskip

  \noindent {\bf Task:} Your task is to identify Big Data size related
  articles and Web resources and produce a historical development of
  the growth of this data

  {\hfill \href{https://docs.google.com/document/d/1ZHNdhX_Jx7uBQo0kthSYQ6TQR8_KNbgOwH2EuqBQcjY/edit?usp=sharing}{E516 Link to google doc $\mapsto$}}



\end{exercise}



\subsection{New Technology List}
\label{E:616-new-tech}

\begin{exercise} 
Due: Jan 29

The handbook contains a large number of technologies to which an
abstract is provided.

Your task is to identify FIRST not to do an abstract but to
collaboratively gather a LIST of new technologies that are important
in Cloud and Big Data. We suggest doing this in a google docs document
first. Write Lastname, Firstname, class id behind the technology so we
know who contributed it. Indicate also if commercial, or open source,
We are mostly interested in open source activities. Keep the list
sorted by alphabet. Use a bullet so formatting is preserved

\url{https://docs.google.com/document/d/1LeHGHTSBbaPXYVor0efhmi5W7JJjS7EQHABHqgRAPuU/edit?usp=sharing}

Example: 

OpenWhisk, \url{https://openwhisk.apache.org/}, open source, Gregor von Laszewski, e616

\end{exercise}


\subsection{New Technology Abstract}
\label{E:616-new-tech-abstract}

\begin{exercise} 
 
Due date: Feb 5th

We have gathered with the technology list 

\url{https://piazza.com/class/jbkvbp3ed3m2ez?cid=50}

a number of technologies that are not yet covered in the handbook or
need improvement in the handbook.

The TAs will be selecting about 5 technologies for each student. Each
student will write high-quality non-plagiarized abstracts which bibtex
references.
 
Learning outcomes:

\begin{itemize}

\item Identify how to not plagiarize
\item Work in a large team (with coordination by TAs)
\item Use bibtex and jabref for reference management which you will be using for your final paper
\item Find new trends in big data and cloud computing

\end{itemize}

\end{exercise}

