\section{Software Projects}\label{software-projects}

\textbf{Contents}

Please read the information in the overview page at

\begin{itemize}
\tightlist
\item
  \url{http://bdaafall2016.readthedocs.io/en/latest/overview.html\#software-project}
\end{itemize}

After doing so please return to this page. Identify a project suitable
for this class, propose it and work on it.

There are several categories of software projects, which are detailed in
lower sections:

\begin{enumerate}
\tightlist
\item
  Deployment
\item
  Analytics
\end{enumerate}

You may propose a project in one of these categories, if you are doing a
software projects.

These are non-trivial project and involve substantial work. Many
students vastly underestimate the difficulty and the amount of time
required. This is the reason why the project assignment is early on in
the semester so you have ample time to propose and work on it. If you
start the project 2 weeks before December (Note the early due data) We
assume you may not finish.

\subsection{Common Requirements}\label{common-requirements}

All software projects must:

\begin{enumerate}
\item
  Be submitted via gitlab (a repository will be created for you)
\item
  Be reproducibly deployed

  Assume you are given a username and a set of IP addresses. From this
  starting point, you should be able to deploy everything in a single
  command line invocation.

  Do not assume that the username or IP address will be the ones you use
  during development and testing.
\item
  Provide a report in the \texttt{docs/report} directory

  LaTeX or Word may be used. Include the original sources as well as a
  PDF called \texttt{report.pdf} (See overview-software-project for
  additional details on the report format. You will be using 2 column
  ACM format we have used before.)
\item
  Provide a properly formatted \texttt{README.rst} or \texttt{README.md}
  in the root directory

  The README should have the following sections:

  \begin{itemize}
  \tightlist
  \item
    Authors: list the authors
  \item
    Project Type: one of ``Deployment'', ``Analytics''
  \item
    Problem: describe the task and/or problem
  \item
    Requirements: describe your assumptions and requirements for
    deployment/running. This should include any software requirements
    with a link to their webpage. Also indicate which versions you have
    developed/tested with.
  \item
    Running: describe the steps needed to deploy and run
  \item
    Acknowledgements: provide proper attribution to any websites, or
    code you may have used or adapted
  \end{itemize}

  \begin{description}
  \item[in the past we got projects that had 10 pages]
  installation instructions. Certainly that is not good and you will get
  point deductions. The installation should be possible in a couple of
  lines. A nice example is the installation of the development software
  in the ubuntu vm. Naturally you can use other technologies, other than
  ansible. Shell scrips, makefiles, python scripts are all acceptable.
  \end{description}
\item
  A \texttt{LICENSE} file (this should be the \texttt{LICENSE} for
  Apache License Version 2.0)
\item
  All figures should include labels with the following format:
  \texttt{label\ (units)}.

  For example:

  \begin{itemize}
  \tightlist
  \item
    \texttt{distance\ (meters)}
  \item
    \texttt{volume\ (liters)}
  \item
    \texttt{cost\ (USD)}
  \end{itemize}
\item
  All figures should have a caption describing what the measurement is,
  and a summary of the conclusions drawn.

  For example:

  \begin{quote}
  This shows how A changes with regards to B, indicating that under
  conditions X, Y, Z, Alpha is 42 times better than otherwise.
  \end{quote}
\end{enumerate}

\subsection{Deployment Projects}\label{deployment-projects}

Deployment projects focuses on automated software deployments on
multiple nodes using automation tools such as Ansible, Chef, Puppet,
Salt, or Juju. You are also allowed to use shell scripts, pdsh, vagrant,
or fabric. For example, you could work on deploying Hadoop to a cluster
of several machines. Use of Ansible is recommended and supported. Other
tools such as Chef, Puppet, etc, will not be supported.

Note that it is not sufficient to merely deploy the software on the
cluster. You must also demonstrate the use of the cluster by running
some program on it and show the utilization of your entire cluster. You
should also benchmark the deployment and running of your demonstration
on several sizes of a cluster (eg 1, 3, 6, 10 nodes) (Note that these
numbers are for example only).

We expect to see figures showing times for each (deployment, running)
pair on for each cluster size, with error bars. This means that you need
to run each benchmark multiple times (at least three times) in order to
get the error bars. You should also demonstrate cluster utilization for
each cluster size.

The program used for demonstration can be simple and straightforward.
This is not the focus of this type of project.

\subsection{IaaS}\label{iaas}

It is allowable to use

\begin{itemize}
\tightlist
\item
  virtualbox
\item
  chameleon cloud
\item
  futuresystems
\item
  AWS (your own cost)
\item
  Azure (your own cost)
\end{itemize}

for your projects. Note that on powerful desktop machines even
virtualbox can run multiple vms. Use of docker is allowed, but you must
make sure to use docker properly. In the past we had students that used
docker but did not use it in the way it was designed for. Use of docker
swarm is allowed.

\subsubsection{Requirements}\label{requirements}

Deployment projects must include a repeatable deployment framework that
uses cmd5 and ansible. When using ansible it should be called from a
custoom cmd5 program.

\subsubsection{Example projects}\label{example-projects}

See also
\url{https://docs.google.com/document/d/1KylDsRBmVbCZSqGpRbzYwdzUGKFi92bkATwU03of5gw}

\begin{itemize}
\tightlist
\item
  deploy Apache Spark on top of Hadoop
\item
  deploy Apache Pig on top of Hadoop
\item
  deploy Apache Storm
\item
  deploy Apache Flink
\item
  deploy a Tensorflow cluster
\item
  deploy a PostgreSQL cluster
\item
  deploy a MongoDB cluster
\item
  deploy a CouchDB cluster
\item
  deploy a Memcached cluster
\item
  deploy a MySQL cluster
\item
  deploy a Redis cluster
\item
  deploy a Mesos cluster
\item
  deploy a Hadoop cluster
\item
  deploy a docker swarm cluster
\item
  deploy NIST Fingerprint Matching
\item
  deploy NIST Human Detection and Face Detection
\item
  deploy NIST Live Twitter Analysis
\item
  deploy NIST Big Data Analytics for Healthcare Data and Health
  Informatics
\item
  deploy NIST Data Warehousing and Data mining
\end{itemize}

Deployment projects must have EASY installation setup just as we
demonstrated in the ubuntu image.

A command to manage the deployment must be written using python docopts
that than starts your deployment and allows management of it. You can
than from within this command call whatever other framework you use to
manage it. The docopts manual page should be designed first and
discussed in the team for completeness.

Using argparse and other python commandline interface environments is
not allowed.

Deployment project will not only deply the farmewor, but either provide
a sophisticated benchmark while doing a simple analysis using the
deployed software.

\subsection{Analytics Projects}\label{analytics-projects}

Analytics projects focus on data exploration. For this type of projects,
you should focus on analysis of a dataset (see datasets for starting
points). The key here is to take a dataset and extract some meaningful
information from in using tools such as \texttt{scikit-learn},
\texttt{mllib}, or others. You should be able to provide graphs,
descriptions for your graphs, and argue for conclusions drawn from your
analysis.

Your deployment should handle the process of downloading and installing
the required datasets and pushing the analysis code to the remote node.
You should provide instructions on how to run and interpret your
analysis code in your README.

\subsubsection{Requirements}\label{requirements-1}

An analytocs project may focus on a sophisticated and academically
correct usage of an analytics of data. It must be significant and not
just a simple replication of what others have done before.

\subsubsection{Example projects}\label{example-projects-1}

\begin{itemize}
\tightlist
\item
  analysis of US Census data
\item
  analysis of Uber ride sharing GPS data
\item
  analysis of Health Care data
\item
  analysis of images for Human Face detection
\item
  analysis of streaming Twitter data
\item
  analysis of airline prices, flights, etc
\item
  analysis of network graphs (social networks, disease networks, protein
  networks, etc)
\item
  analysis of music files for recommender engines
\item
  analysis of NIST Fingerprint Matching
\item
  analysis of NIST Human Detection and Face Detection
\item
  analysis of NIST Live Twitter Analysis
\item
  analysis of NIST Big Data Analytics for Healthcare Data and Health
  Informatics
\item
  analysis of NIST Data Warehousing and Data mining
\item
  author disambiguity problem in academic papers
\item
  application of a k-means algorithm
\item
  application of a MDS
\end{itemize}

\subsection{Project Idea: World wide road
kill}\label{project-idea-world-wide-road-kill}

This project can also be executed as bonus project to gather information
about the feasability of existing databases.

It would be important to identify also how to potentially merge these
databases into a single world map and derive statistics from them. This
project can be done on your local machines. Not more than 6 people can
work on this.

Identify someone that has experience with android and/or iphone
programming Design an application that preferably works on iphone and
android that allows a user while driving to

\begin{itemize}
\tightlist
\item
  call a number to report roadkill via voice and submitting the gps
  coordinates
\item
  have a button on the phone that allows the gps coordinates to be
  collected and allow upload either live, or when the user presses
  another butten.
\item
  have provisions in the application that allow you to augment the data
\item
  have an html page that displays the data
\item
  test it out within users of this class (remember we have world wide
  audience)
\end{itemize}

Make sure the app is ready early so others can test and use it and you
can collect data.

Before starting the project identify if such an application already
exists.

If more than 6 people sign up we may build a second group doing
something similar, maybe potholes ..

Gregor would like to get this project or at least the database search
query staffed.

\subsection{Project Idea: Author disambiguty
problem}\label{project-idea-author-disambiguty-problem}

Given millions of publications how do we identify if an author of paper
a with the name Will Smith is the sam as the author of paper 2 with the
name Will Smith, or William Smith, or W. Smith. AUthor databases are
either provided in bibtex format, or a database that can not be shared
outside of this class. YOu may have to add additional information from
IEEE explorer, rsearch gate, ISI, or other online databases.

Identify further issues and discuss solutions to them. Example, an
author name changes, the author changes the institution.

Do a comprehensive literature review

Some ideas:

\begin{itemize}
\tightlist
\item
  Develop a graph view application in JS that showcases dependencies
  between coauthors, institutions
\item
  Derive probabilities for the publications written by an auther given
  they are the same
\item
  Utilize dependency graphs as given by online databases
\item
  Utilize the and or topic/abstarct/full text to identify similarity
\item
  Utilize keywords in the title
\item
  Utilize refernces of the paper
\item
  Prepare some vizualization of your result
\item
  Prepare som interactive vizualization
\end{itemize}

A possible good start is a previous project published at

\begin{itemize}
\tightlist
\item
  \url{https://github.com/scienceimpact/bibliometric}
\end{itemize}

There are also some screenshots available:

\begin{itemize}
\tightlist
\item
  \url{https://github.com/scienceimpact/bibliometric/blob/master/Project\%20Screenshots/Relationship_Authors_Publications.PNG}
\item
  \url{https://github.com/scienceimpact/bibliometric/blob/master/Project\%20Screenshots/Relationship_Authors_Publications2_Clusters.PNG}
\end{itemize}
