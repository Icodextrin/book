
\chapterimage{rest.png} 

\chapter{REST} 
\label{c:rest}
\index{REST}


\section{Overview of REST}
\label{s:rest-intro}
\index{REST!Overview}

This section is accompanied by a video about REST.

\video{REST}{36:02}{REST}{https://youtu.be/xjFuA6q5N_U}

REST stands for {\bf RE}presentational {\bf S}tate {\bf
  T}ransfer. REST is an architecture style for designing networked
applications. It is based on stateless, client-server, cacheable
communications protocol. In contrast to what some others write or say,
REST is not a \emph{standard}. Although not based on http, in most
cases, the HTTP protocol is used. In that case, RESTful applications
use HTTP requests to (a) post data while creating and/or updating it,
(b) read data while making queries, and (c) delete data.

REST was first introduced in a thesis from Fielding:

\URL{https://www.ics.uci.edu/~fielding/pubs/dissertation/top.htm}

Hence REST can use HTTP for the four CRUD operations:

\begin{itemize}
\item  {\bf C}reate resources
\item  {\bf R}ead resources
\item  {\bf U}pdate resources
\item  {\bf D}elete resources
\end{itemize}

As part of the HTTP protocol we have methods such as GET, PUT, POST,
and DELETE. These methods can than be used to implement a REST
service. This is not surprising as the HTTP protocol was explicitly
designed to support these operations. As REST introduces collections
and items we need to implement the CRUD functions for them.  We
distinguish single resources  and collection of resources.  The
semantics for accessing them is explained next illustrating how to
implement them with HTTP methods (see
\url{https://en.wikipedia.org/wiki/Representational_state_transfer}).


\subsection{Collection of Resources}

Let us assume the following URI identifies a collection of resources

\verb|http://.../resources/|

than we need to implement the following CRUD methods:

\begin{description} 
    \item[GET] List the URIs and perhaps other details of the
      collections members
    \item[PUT] Replace the entire collection with another collection.
    \item[POST] Create a new entry in the collection. The new entry’s
      URI is assigned automatically and is usually returned by the
      operation.
    \item[DELETE] Delete the entire collection.
\end{description} 


\subsection{Single Resource}

Let us assume the following URI identifies a single resource in a
collection of resources

\verb|http://.../resources/item42|

than we need to implement the following CRUD methods:

    \begin{description} 

    \item[GET] Retrieve a representation of the addressed member of
      the collection, expressed in an appropriate internet media type.

    \item[PUST] Replace the addressed member of the collection, or if
      it does not exist, create it.

    \item[POST] Not generally used. Treat the addressed member as a
      collection in its own right and create a new entry within it.
                               
    \item[DELETE] Delete the addressed member of the collection. 
\end{description} 

 
\subsection{REST Tool Classification}

Due to the well defined structure that REST provides a number of tools
have been created that manage the creation of the specification for
rest services and their programming. We distinguish several different
categories:

\begin{description}
\item[REST programming language support:] These tools and services are
targeting a particular programming language. Such tools include Eve
which we will explore in more detail.

\item[REST documentation based tools:] These tools are primarily
  focussing on documenting REST specifications. Such tools include
  Swagger, which we will explore in more detail.

\item[REST design support tools:] These tools are used to support the
  design process of developing REST services while abstracting on top
  of the programming languages and define reusable specifications that
  can be used to create clients and servers for particular technology
  targets. Such tools include also swagger as additional tools are
  available that can generate code from swagger specifications, which
  we will explore in more detail.
\end{description}


\section{Flask RESTful Services}
\label{s:rest-flask}
\index{REST}
\index{REST!Flask RESTful}


Flask is a mocro services framework allowing to write web services in
python quickly. One of its extensions is Flask-RESTful. It adds for 
building REST APIs based on a class definition making it relatively
simple. Through tis interface we can than integrate with
your existing Object Relational Modles and libraries. As Flask-RESTful
leverages the main features from Flask an extensive set of
documentation is available allowing you to get started quickly and
thouroghly. The Web page contains extensive documentation:

\URL{https://flask-restful.readthedocs.io/en/latest/}

We will provide a simple example that showcases some {\em hard coded}
data to be served as a rest service. It will be easy to replace this
for example with functions and methods that obtain such information
dynamically from the operationg system. 

\begin{WARNING}
This example has not been tested., We like that the E222 class defines
a beautiful example to contribute to this section. and explains what
happens in this example.
\end{WARNING}


\begin{lstlisting}
from flask import Flask
from flask_restful import reqparse, abort, Api, Resource

app = Flask(__name__)
api = Api(app)

COMPUTERS = {
    'computer1': {
      'processor': 'iCore7'
    },
    'computer2': {
      'processor': 'iCore5'
    },
    'computer3': {
      'processor': 'iCore3'
    },
}

def abort_if_cluster_doesnt_exist(computer_id):
    if todo_id not in TODOS:
        abort(404, message="Computer {} doesn't exist".format(computer_id))

parser = reqparse.RequestParser()
parser.add_argument('processor')

class Computer(Resource):
    ''' shows a single computer item and lets you delete a computer
        item.'''

    def get(self, computer_id):
        abort_if_computer_doesnt_exist(computer_id)
        return COMPUTERS[computer_id]

    def delete(self, computer_id):
        abort_if_computer_doesnt_exist(computer_id)
        del COMPUTERS[computer_id]
        return '', 204

    def put(self, computer_id):
        args = parser.parse_args()
        processor = {'processor': args['processor']}
        COMPUTERS[computer_id] = processor
        return processor, 201


# ComputerList
class ComputerList(Resource):
    ''' shows a list of all computers, and lets you POST to add new computers'''

    def get(self):
        return COMPUTERS

    def post(self):
        args = parser.parse_args()
        computer_id = int(max(COMPUTERS.keys()).lstrip('computer')) + 1
        computer_id = 'computer%i' % computer_id
        COMPUTERS[computer_id] = {'processor': args['processor']}
        return COMPUTERS[computer_id], 201

##
## Setup the Api resource routing here
##
api.add_resource(ComputerList, '/computers')
api.add_resource(Computer, '/computers/<computer_id>')


if __name__ == '__main__':
    app.run(debug=True)
\end{lstlisting}


\section{Rest Services with Eve}
\label{s:eve-intro}
\index{REST}
\index{REST!Eve}

Next, we will focus on how to make a RESTful web service with Python
Eve. Eve makes the creation of a REST implementation in python
easy. More information about Eve can be found at:

\URL{http://python-eve.org/}

\begin{WARNING}

Although we do recommend Ubuntu 17.04, at this time there is a bug
that forces us to use 16.04. Furthermore, we require you to follow the
instructions on how to install pyenv and use it to set up your python
environment. We recommend that you use either python 2.7.14 or 3.6.4.
We do not recommend you to use anaconda as it is not suited for cloud
computing but targets desktop computing. If you use pyenv you also
avoid the issue of interfering with your system wide python
install. We do recommend pyenv regardless if you use a virtual machine
or are working directly on your operating system. After you have set
up a proper python environment, make sure you have the newest version
of pip installed with 

\smallskip

\begin{lstlisting}{language=bash}
$ pip install pip -U
\end{lstlisting}
\end{WARNING}

To install Eve, you can say 

\begin{lstlisting}{language=bash}
$ pip install eve
\end{lstlisting}

As Eve also needs a backend database, and as MongoDB is an obvious
choice for this, we will have to first install MongoDB.  MongoDB is a Non-SQL
database which helps to store light weight data easily.

\subsection{Ubuntu install of MongoDB}
On Ubuntu you can install MongoDB as follows

\begin{lstlisting}{language=bash}
$ sudo apt-key adv --keyserver hkp://keyserver.ubuntu.com:80 --recv 2930ADAE8CAF5059EE73BB4B58712A2291FA4AD5
$ echo "deb [ arch=amd64,arm64 ] https://repo.mongodb.org/apt/ubuntu xenial/
mongodb-org/3.6 multiverse" | sudo tee /etc/apt/sources.list.d/mongodb-org-3.6.list
$ sudo apt-get update
$ sudo apt-get install -y mongodb-org
\end{lstlisting}
% $

\subsection{OSX install of MongoDB}

On OSX you can use the command

\begin{lstlisting}{language=bash}
brew update.
brew install mongodb
\end{lstlisting}

\subsection{Windows 10 Installation of MongoDB}

\begin{exercise}

A student or student group of this class are invited to discuss on
piazza on how to install mongoDB on Windows 10 and come up with an
easy installation solution. Naturally we have the same 2 different ways
on how to run mongo. In user space or in the system. As we want to
make sure your computer stays secure. the solution must have an easy
way on how to shut down the Mongo services.

An enhancement of this task would be to integrate this function into
cloudmesh cmd5 with a command {\em mongo} that allows for easily
starting and stopping the service from {\em cms}.
\end{exercise}


\subsection{Database Location}
After downloading Mongo, create the {\em db} directory. This is where
the Mongo data files will live. You can create the directory in the
default location and assure it has the right permissions. Make sure
that the /data/db directory has the right permissions by running

\subsection{Verification}

In order to check the MongoDB installation, please run the following
commands in one terminal:

\begin{lstlisting}{language=bash}
$ mkdir -p ~/cloudmesh/data/db
$ mongod --dbpath ~/cloudmesh/data/db
\end{lstlisting}

In another terminal we try to connect to mongo and issue a mongo
command to show the databases:

\begin{lstlisting}{language=bash}
$ mongo --host 127.0.0.1:27017
$ show databases
\end{lstlisting}

If they execute without errors, you have successfully installed
MongoDB. In order to stop the running database instance run the
following command. simply CTRL-C the running mongod process

\subsection{Building a simple REST Service}

In this section we will focus on creating a simple rest service. To
organize our work we will create the following directory:


\begin{lstlisting}{language=bash}
$ mkdir -p ~/cloudmesh/eve
$ cd ~/cloudmesh/eve
\end{lstlisting}

As Eve needs a configuration and it is read in by default from the
file \verb|settings.py| we place the following content in the file
\verb|~/cloudmesh/eve/settings.py|:

\begin{lstlisting}
MONGO_HOST = 'localhost'
MONGO_PORT = 27017
MONGO_DBNAME = 'student_db'
DOMAIN = {
    'student': {
        'schema': {
            'firstname': {
                'type': 'string'
            },
            'lastname': {
                'type': 'string'
            },
            'school': {
                'type': 'string'
            },
            'university': {
                'type': 'string'
            },
            'email': {
                'type': 'string',
                 'unique': True
            }
        }
    }
}
RESOURCE_METHODS = ['GET', 'POST']
\end{lstlisting}

The DOMAIN object specifies the format of a \verb|student| object that
we are using as part of our REST service.  In addition we can specify
\verb|RESOURCE_METHODS| which methods are activated for the REST
service. This way the developer can restrict the available methods for
a REST service. To pass along the specification for mongoDB, we simply
specify the hostname, the port, as well as the database name.

Now that we have defined the settings for our example service, we need
to start it with a simple python program. We could name that program
anything we like, but often it is called simply \verb|run.py|. This
file is placed in the same directory where you placed the
\textbf{settings.py}. In our case it is in the file
\verb|~/cloudmesh/eve/run.py| and contains the following python
program:

\begin{lstlisting}
from eve import Eve
app = Eve()

if __name__ == '__main__':
    app.run()
\end{lstlisting}

This is the most minimal application for Eve, that uses the
settings.py file for its configuration. Naturally, if we were to
change the configuration file and for example change the DOMAIN and
its schema, we would naturally have to remove the database previously
created and start the service new. This is especially important as
during the development phase we may frequently change the schema and
the database. Thus it is convenient to develop necessary cleaning
actions as part of a Makefile which we leave as easy exercise for the
students.

Next, we need to start the services which can easily be achieved in a
terminal while running the commands:

Previously we started the mongoDB service as follows:

\begin{lstlisting}{language=bash}
$ mongod --dbpath ~/cloudmesh/data/db/
\end{lstlisting}
%$

This is done in its own terminal, so we can observe the log messages easily.
Next we start in another window the Eve service with 

\begin{lstlisting}{language=bash}
$ cd ~/cloudmesh/eve
$ python run.py
\end{lstlisting}

You can find the codes and commands up to this point in the following document.
\URL{https://github.com/cloudmesh/book/blob/dev/section/rest/rest-code.md}

\subsection{Interacting with the REST service}

Yet in another window, we can now interact with the REST service.
We can use the commandline to save the data in the database using the
REST api. The data can be retrieved in XML or in json format. Json is
often more convenient for debugging as it is easier to read than XML.

Naturally, we need first to put some data into the server. Let us assume
we add the user Albert Zweistein.

\begin{lstlisting}{language=bash}
$ curl -H "Content-Type: application/json" -X POST \
       -d '{"firstname":"Albert","lastname":"Zweistein", \
       "school":"ISE","university":"Indiana University", \
       "email":"albert@iu.edu"}' http://127.0.0.1:5000/student/
\end{lstlisting}
%$

To achieve this, we need to specify the header using \textbf{H} tag
saying we need the data to be saved using json format. And \textbf{X}
tag says the HTTP protocol and here we use POST method. And the tag
\textbf{d} specifies the data and make sure you use json format to
enter the data. Finally, the REST api endpoint to which we must save
data. This allows us to save the data in a table called
\textbf{student} in MongoDB within a database called \textbf{eve}.

In order to check if the entry was accepted in mongo and included in the
server issue the following command sequence in another terminal:

\begin{lstlisting}
$ mongo
\end{lstlisting}
% $ 

Now you can query mongo directly with its shell interface

\begin{lstlisting}
> show databases
> use student_db  
> show tables # query the table names
> db.student.find().pretty()  # pretty will show the json in a clear way
\end{lstlisting}
%$

Naturally this is not really necessary for A REST service such as eve
as we show you next how to gain access to the data via mongo while
using REST calls. We can simply retrieve
the information with the help of a simple URI:


\begin{lstlisting}{language=bash}
$ curl http://127.0.0.1:5000/student?firstname=Albert
\end{lstlisting}
%$

Naturally, you can formulate other URLs and query attributes that are
passed along after the ? 

This will now allow you to develop sophisticated REST services. We
encourage you to inspect the documentation provided by Eve to showcase
additional features that you could be using as part of your efforts.

Let us explore how to properly use additional REST API calls. We
assume you have MongoDB up and running. To query the service itself we
can use the URI on the Eve port


\begin{lstlisting}{language=bash}
$ curl -i http://127.0.0.1:5000
\end{lstlisting}
%$

Your payload should look like the one below, if your output is not
formatted like below try adding \verb|?pretty=1|

\begin{lstlisting}{language=bash}
$ curl -i http://127.0.0.1:5000?pretty=1
\end{lstlisting}
%$

\begin{lstlisting}
HTTP/1.0 200 OK
Content-Type: application/json
Content-Length: 150
Server: Eve/0.7.6 Werkzeug/0.11.15 Python/2.7.5
Date: Wed, 17 Jan 2018 18:34:07 GMT

{
    "_links": {
        "child": [
            {
                "href": "student",
                "title": "student"
            }
        ]
    }
\end{lstlisting}

Remember that the API entry points include additional information such
as links and a child, and href.



\begin{exercise}
Set up a python environment that works for your platform. Provide
explicit reasons why anaconda and other prepackaged python versions
have issues for cloud related activities. When may you use anaconda
and when should you not use anaconda. Why would you want to use pyenv?
\end{exercise}

\begin{exercise}
What is the meaning and purpose of links, child, and href
\end{exercise}

\begin{exercise}
In this case how many child resources are available through
our API?
\end{exercise}

\begin{exercise}
Develop a REST service with Eve and start and stop it
\end{exercise}

\begin{exercise}
Define curl calls to store data into the service and retrive it.
\end{exercise}

\begin{exercise}
Write a Makefile and in it a target clean that cleans the data
base. Develop additional targets such as start and stop, that start
and stop the mongoDB but also the Eve REST service
\end{exercise}


\begin{exercise}
Issue the command

\begin{lstlisting}{language=bash}
$ curl -i http://127.0.0.1:5000/people
\end{lstlisting}
%$

What does the \verb|_links| section describe?

What does the \verb|_items| section describe?

\begin{lstlisting}
{
    "_items": [],
    "_links": {
        "self": {
            "href": "people",
            "title": "people"
        },
        "parent": {
            "href": "/",
            "title": "home"
        }
    },
    "_meta": {
        "max_results": 25,
        "total": 0,
        "page": 1
    }
}
\end{lstlisting} 
\end{exercise}

\clearpage

\subsection{Creating REST API Endpoints}\label{s:rest-api-endpoints}

Next we wont to enhance our example a bit. First, let us get back to
the eve working directory with

\begin{lstlisting}
 $ cd ~/cloudmesh/eve
\end{lstlisting}
%$

Add the following content to a file called \textbf{run2.py}

\begin{lstlisting}
from eve import Eve
app = Eve()

@app.route('/student/john')
def theStudentJohn():
    return "My name is John"

if __name__ == '__main__':
    app.run()

\end{lstlisting}

After creating and saving the file. Run the following command to start
the service


\begin{lstlisting}
 $ python run2.py
\end{lstlisting}
%$

After running the command, you can interact with the service while
entering the  following url in the web browser:

\begin{lstlisting}
http://127.0.0.1:5000/student/john
\end{lstlisting}

The browser will show the following message:

\begin{lstlisting}
My name is John
\end{lstlisting}

This example illustrates how easy it is to create REST services in
python.  The important part is to understand the decorator
\textbf{app.route} provided the route of the API endpoint which will
be the address appended to the base path,
\textbf{http://127.0.0.1:5000}. And in this example we just returned a
simple string using the annotated method. It is not compulsory
for the function to have the same name as the API endpoint, but the
best practice is to have a relationship between endpoint name and
function name. For instance you can use \textbf{/student/john} and
use a function called \textbf{theStudentJohn}.

\subsection{REST API Output Formats and Request Processing}

In Eve, by default the data return type is JSON, in this case if we have a
result as a string, there has to be a way to formulate a standard JSON. In order
to do this, we can include a better formatting in the code as follows. 

If we want to create an object like Student, we can first define a python class.
Create a file called \textbf{student.py}.

\begin{lstlisting}
class Student(object):
    def __init__(self, firstname, lastname, age, school, email):
        self.firstname = firstname
        self.lastname = lastname
        self.age = age
        self.school = school
        self.email = email

\end{lstlisting}

This is how we create a class in python. Next we can reformulate our API coding
we earlier did to return information about a particular student. Open up the 
same \textbf{run2.py} file and add the following lines and the final
file will look like as shown below. (You can replace the content in the previous
file).

\begin{lstlisting}
from eve import Eve
from student import Student
import platform
import psutil
import json
from flask import Response

app = Eve()


@app.route('/student/john', methods=['GET'])
def processor():
    studentInfo = Student("John", "Doe", "23", "ISE", "john@iu.edu")

    sdata = json.dumps(studentInfo.__dict__)

    response = Response()
    response.headers["status"] = 200
    response.headers["Content-Type"] = "application/json; charset=utf-8"
    response.data = sdata

    return response


if __name__ == '__main__':
    app.run()


\end{lstlisting}


\subsection{Consuming REST API Using a Client Application}

In the Section \ref{s:rest-api-endpoints} we created our own REST API application
using Python Eve. Now once the service running, a we need to learn how to
interact with it through clients.

First go back to the working folder:

\begin{lstlisting}
 $ cd ~/cloudmesh/eve
\end{lstlisting}
%$

Here we create a new python file called \textbf{client.py}. The file include the
following content.

\begin{lstlisting}
import requests
import json

def get_all():
    response = requests.get("http://127.0.0.1:5000/student")
    print(json.dumps(response.json(), indent=4, sort_keys=True))


def save_record():
    headers = {
        'Content-Type': 'application/json'
    }

    data = '{"firstname":"Jane","lastname":"Doe","school":"ISE","university":"Indiana University","email":"jane@iu.edu"}'

    response = requests.post('http://localhost:5000/student/',
                              headers=headers,
                              data=data)

    print(response.json())

save_record()
#get_all()
\end{lstlisting}

Run the following command in a new terminal to execute the simple client by 

\begin{lstlisting}
 $ python client.py
\end{lstlisting}
%$

Here when you run this class for the first time, it will run successfully, but 
if you tried it for the second time, it will give you an error. Because we did 
set the email to be a unique field in the schema when we designed the settings.py
file in the beginning. So if you want to save another record you must have entries
with unique emails. In order to make this dynamic you can include a input reading
by using the terminal to get the student data first and instead of the static
data you can use the user input data from the terminal to get dynamic data. 
But for this exercise we don't expect that or any other form data functionality.

In order to get the saved data, you can comment the record saving function and
uncoment the get all function. In python commenting is done by using \verb|#|.

This client is using the \textbf{requests} python
library to send GET, POST and other HTTP requests to the server so you
can leverage build in methods to simplify your work.

The $get_all$ function provides a way to get the output to the
console with all the data in the student database. The \verb|save_record|
function provides a way to save data in the database. You can create
dynamic functions in order to save dynamic data. However it may take
some time for you to apply as exercise.

\begin{exercise}
Write a RESTful service to determine a useful piece of information off
of your computer i.e. disk space, memory, RAM, etc. In this exercise what
you need to do is use a python library to extract data about computer
information mentioned above and send these information to the user once
the user calls an API endpoint like http://localhost:5000/performance/ram,
it must return the RAM value of the given machine. For each information like
disk space, RAM, etc you can use an endpoint per each feature needed. As a tip 
for this exercise, use the psutil library in python to retrieve the data, and
then get these information into an string then populate a class called Computer 
and try to save the object like wise.
\end{exercise}

\subsection{HATEOAS}\label{s:hateoas}
\index{REST!HATEOS}

In the previous section we discussed the basic concepts about RESTful
web service. Next we introduce you to the concept of HATEOAS

HATEOAS stands for Hypermedia as the Engine of Application State and
this is enabled by the default configuration within Eve. It is useful
to review the terminology and attributes used as part of this
configuration. HATEOS explains how REST API endpoints are defined and it
provides a clear description on how the API can be consumed through
these terms:

\begin{description} 

\item [\_links]. Links describe the relation of current resource being
  accessed to the rest of the resources. It is like if we have a set
  of links to the set of objects or service endpoints that we are
  referring in the RESTful web service. Here an endpoint refers to a
  service call which is responsible for executing one of the CRUD
  operations on a particular object or set of objects. More on the
  links, the links object contains the list of serviceable API endpoints
  or list of services.  When we are calling a GET request or any other
  request, we can use these service endpoints to execute different
  queries based on the user purpose.  For instance, a service call can
  be used to insert data or retrieve data from a remote database using
  a REST API call. About databases we will discuss in detail in another
  chapter.

\item [title]. The title in the rest endpoint is the name or topic
  that we are trying to address. It describes the nature of the object
  by a single word. For instance student, bank-statement, salary,etc
  can be a title.

\item [parent]. The term parent referes to the very initial link or an
  API endpoint in a particular RESTful web service. Generally this is
  denoted with the primary address like http://example.com/api/v1/.

\item [href]. The term href refers to the url segment that we use to
  access the a particular REST API endpoint. For instance
  ``student?page=1'' will return the first page of student list by
  retrieving a particular number of items from a remote database or a
  remote data source. The full url will look like this,
  ``http://www.exampleapi.com/student?page=1''.

\end{description}  

In addition to these fields eve will automatically create the
follwoing information when resources are created as showcased ot 

\URL{http://python-eve.org/features.html}


\begin{tabular}{ll}
Field	& Description \\
\hline
\verb|_created|	& item creation date.\\
\verb|_updated|	& item last updated on.\\
\verb|_etag|	& ETag, to be used for concurrency control and conditional requests.\\
\verb|_id|	& unique item key, also needed to access the individual item
  endpoint.\\
\end{tabular}

Pagenation information can be included in the \verb|_meta| field.


\subsubsection{Filtering}


CLients can submit query strings to the rest service to retrieve
resources based on a filter. This also allows sorting of the results
queried. One nice feature about using mongo as a backend database is
that Eve not only allows python conditional expressions, but also
mongo queries.

A number of examples to conduct such queries include:


\begin{lstlisting}
curl -i -g http://eve-demo.herokuapp.com/people?where={%22lastname%22:%20%22Doe%22}
\end{lstlisting}

A python expression

\begin{lstlisting}
curl -i http://eve-demo.herokuapp.com/people?where=lastname=="Doe"
\end{lstlisting}

\begin{comment}
\subsection{Rendering Data in UI}

\TODO{seems incomplete}

In this section, we are talking about the way of using HATEOAS
practically in user interface development in web applications.

\begin{lstlisting}
  <resource>
  <link rel='child' href='student' title='student'/>
  </resource>
\end{lstlisting}

This is the practical representation of the HATEOAS in real world
applications.
\end{comment}

\subsection{Pretty Printing}

Pretty printing is typically supported by adding the parameter
\verb|?pretty| or \verb|?pretty=1|

If this does not work you can always use python to beautify a json
output with 

\begin{lstlisting}
curl -i http://localhost/people?pretty
\end{lstlisting}

or

\begin{lstlisting}
curl -i http://localhost/people | python -m json.tool
\end{lstlisting}

\subsection{XML}

If for some reason you like to retrieve the information in XML you can
specify this for example through curl with an Accept header

\begin{lstlisting}
curl -H "Accept: application/xml" -i http://localhost
\end{lstlisting}


\subsection{Extensions to Eve}
\index{REST!Eve!Extensions}

A number of extensions have been developed by the community. This
includes
eve-swagger,
eve-sqlalchemy,
eve-elastic,
eve-mongoengine,
eve-neo4j,
eve.net,
eve-auth-jwt, and
flask-sentinel.

Naturally there are many more.

Students have the opportunity to pic one of the community extensions
and provide a section for the handbook.

\begin{exercise}
Pick one of the extension, research it and provide a small section for
the handbook so we add it.
\end{exercise}

\section{Object Management with Eve and Evegenie}
\index{REST!Evegenie}
\index{Evegenie}


\url{http://python-eve.org/}

Eve makes the creation of a REST implementation in python easy.  We
will provide you with an implementation example that showcases that we
can create REST services without writing a single line of code. The
code for this is located at \url{https://github.com/cloudmesh/rest}

This code will have a master branch but will also have a dev branch in
which we will add gradually more objects. Objects in the dev branch will
include:

\begin{itemize}
\item
 virtual directories
\item
 virtual clusters
\item
 job sequences
\item
 inventories
\end{itemize}

You may want to check our active development work in the dev branch.
However for the purpose of this class the master branch will be
sufficient.

\subsection{Installation}\label{installation}

First we have to install mongodb. The installation will depend on your
operating system. For the use of the rest service it is not important to
integrate mongodb into the system upon reboot, which is focus of many
online documents. However, for us it is better if we can start and stop
the services explicitly for now.

On ubuntu, you need to do the following steps:

\TODO{TO BE CONTRIBUTED BY THE STUDENTS OF THE CLASS AS HOMEWORK}

On windows 10, you need to do the following steps:

\TODO{TO BE CONTRIBUTED BY THE STUDENTS OF THE CLASS AS HOMEWORK. If
  you elect Windows 10. You could be using the online documentation
  provided by starting it on Windows, or running it in a docker
  container.}

On OSX you can use home-brew and install it with:

\begin{lstlisting}{language=bash}
brew update
brew install mongodb
\end{lstlisting}

In future we may want to add ssl authentication in which case you
may
need to install it as follows:

\begin{lstlisting}{language=bash}
brew install mongodb --with-openssl
\end{lstlisting}

\subsection{Starting the service}\label{starting-the-service}

We have provided a convenient Makefile that currently only works for
OSX. It will be easy for you to adapt it to Linux. Certainly you can
look at the targes in the makefile and replicate them one by one.
Important targets are deploy and test.

When using the makefile you can start the services with:

\begin{lstlisting}{language=bash}
make deploy
\end{lstlisting}

IT will start two terminals. IN one you will see the mongo service, in
the other you will see the eve service. The eve service will take a file
called sample.settings.py that is base on sample.json for the start of
the eve service. The mongo servide is configured in such a way that it
only accepts incoming connections from the local host which will be
sufficient for our case. The mongo data is written into the
\verb|$USER/.cloudmesh| directory, so make sure it exists.

To test the services you can say:

\begin{lstlisting}{language=bash}
make test
\end{lstlisting}

YOu will se a number of json text been written to the screen.

\subsection{Creating your own objects}\label{creating-your-own-objects}

The example demonstrated how easy it is to create a mongodb and an eve
rest service. Now let us use this example to create your own. For
this we have modified a tool called evegenie to install it onto your
system.

The original documentation for evegenie is located at:

\begin{itemize}
\item
 \url{http://evegenie.readthedocs.io/en/latest/}
\end{itemize}

However, we have improved evegenie while providing a commandline tool
based on it. The improved code is located at:

\begin{itemize}
\tightlist
\item
 \url{https://github.com/cloudmesh/evegenie}
\end{itemize}

You clone it and install on your system as follows:

\begin{lstlisting}{language=bash}
cd ~/github
git clone https://github.com/cloudmesh/evegenie
cd evegenie
python setup.py install
pip install .
\end{lstlisting}

This should install in your system evegenie. YOu can verify this by
typing:

\begin{lstlisting}
which evegenie
\end{lstlisting}

If you see the path evegenie is installed. With evegenie installed its
usaage is simple:

\begin{lstlisting}
$ evegenie

Usage:
  evegenie --help
  evegenie FILENAME
\end{lstlisting}
%$

It takes a json file as input and writes out a settings file for the use
in eve. Lets assume the file is called sample.json, than the settings
file will be called sample.settings.py. Having the evegenie programm
will allow us to generate the settings files easily. You can include
them into your project and leverage the Makefile targets to start the
services in your project. In case you generate new objects, make sure
you rerun evegenie, kill all previous windows in which you run eve and
mongo and restart. In case of changes to objects that you have designed
and run previously, you need to also delete the mongod database.

\section{Towards cmd5 extensions to manage eve and
mongo}\label{towards-cmd5-extensions-to-manage-eve-and-mongo}


Naturally it is of advantage to have in cms administration commands to
manage mongo and eve from cmd instead of targets in the Makefile. Hence,
we \textbf{propose} that the class develops such an extension. We will
create in the repository the extension called admin and hobe that
students through collaborative work and pull requests complete such an
admin command.

The proposed command is located at:

 \URL{https://github.com/cloudmesh/rest/blob/master/cloudmesh/ext/command/admin.py}


It will be up to the class to implement such a command. Please
coordinate with each other.

The implementation based on what we provided in the Make file seems
straight forward. A great extension is to load the objects definitions
or eve e.g. settings.py not from the class, but forma place in
.cloudmesh. I propose to place the file at:

\begin{lstlisting}
.cloudmesh/db/settings.py
\end{lstlisting}

the location of this file is used when the Service class is initialized
with None. Prior to starting the service the file needs to be copied
there. This could be achieved with a set command.

\section{Responses}
\index{Responses}
\URL{https://dzone.com/refcardz/rest-foundations-restful}

\begin{comment}


\begin{lstlisting}
Code  Description

200 OK. The request has successfully executed. Response depends upon the verb invoked.
201 Created. The request has successfully executed and a new resource has been created in the process. The response body is either empty or contains a representation containing URIs for the resource created. The Location header in the response should point to the URI as well.
202 Accepted. The request was valid and has been accepted but has not yet been processed. The response should include a URI to poll for status updates on the request. This allows asynchronous REST requests
204 No Content. The request was successfully processed but the server did not have any response. The client should not update its display.
Table 1 - Successful Client Requests


Redirected Client Requests

CODE  DESCRIPTION
301 Moved Permanently. The requested resource is no longer located at the specified URL. The new Location should be returned in the response header. Only GET or HEAD requests should redirect to the new location. The client should update its bookmark if possible.
302 Found. The requested resource has temporarily been found somewhere else. The temporary Location should be returned in the response header. Only GET or HEAD requests should redirect to the new location. The client need not update its bookmark as the resource may return to this URL.
303 See Other. This response code has been reinterpreted by the W3C Technical Architecture Group (TAG) as a way of responding to a valid request for a non-network addressable resource. This is an important concept in the Semantic Web when we give URIs to people, concepts, organizations, etc. There is a distinction between resources that can be found on the Web and those that cannot. Clients can tell this difference if they get a 303 instead of 200. The redirected location will be reflected in the Location header of the response. This header will contain a reference to a document about the resource or perhaps some metadata about it.




Invalid Client Requests

Code  Description
405 Method Not Allowed.
406 Not Acceptable.
410 Gone.
411 Length Required.
412 Precondition Failed.
413 Entity Too Large.
414 URI Too Long.
415 Unsupported Media Type.
417 Expectation Failed.


Code  Description
500 Internal Server Error.
501 Not Implemented.
503 Service Unavailable.

\end{lstlisting}

\end{comment}




\section{Django REST Framework}
\index{Django!REST}
\URL{http://www.django-rest-framework.org/}

Django REST framework is a large toolkit to develop Web APIs. The
developers of the framework provide the following reasons for using it:

``(1) The Web browsable API is a huge usability win for your
developers.  (2) Authentication policies including packages for
OAuth1a and OAuth2.  (3) Serialization that supports both ORM and
non-ORM data sources.  (4) Customizable all the way down - just use
regular function-based views if you do not need the more powerful
features.  (5) Extensive documentation, and great community support.
(6) Used and trusted by internationally recognised companies including
Mozilla, Red Hat, Heroku, and Eventbrite.''

