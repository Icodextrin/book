
\chapter{REST}

\section{Overview}

This section is accompanied by a video about REST.

\video{REST}{36:02}{REST}{https://youtu.be/xjFuA6q5N_U}

REST stands for {\bf RE}presentational {\bf S}tate {\bf
  T}ransfer. REST is an architecture style for designing networked
applications. It is based on stateless, client-server, cacheable
communications protocol. Although not based on http, in most cases,
the HTTP protocol is used. In contrast to what some others write or
say, REST is not a \emph{standard}. RESTful applications use HTTP
requests to (a) post data while creating and/or updating it, (b) read
data while making queries, and (c) delete data.

\url{https://www.ics.uci.edu/~fielding/pubs/dissertation/top.htm}

Hence REST uses HTTP for the four CRUD operations:

\begin{itemize}
\item  {\bf C}reate 
\item  {\bf R}ead
\item  {\bf U}pdate
\item  {\bf D}elete
\end{itemize}

As part of the HTTP protocol we have methods such as GET, PUT, POST, and
DELETE. These methods can than be used to implement a REST service. As
REST introduces collections and items we need to implement the CRUD
functions for them. The semantics is explained in the Table
illustrating how to implement them with HTTP methods.

\begin{description}

\item[http://.../resources/] ~\\

    \begin{description} 
    \item[GET] List the URIs and perhaps other details of the
      collection’s members
    \item[PUT] Replace the entire collection with another collection.
    \item[POST] Create a new entry in the collection. The new entry’s
      URI is assigned automatically and is usually returned by the
      operation.
    \item[DELETE] Delete the entire collection.
\end{description} 

\item[http://.../resources/item42] ~\\

    \begin{description} 

    \item[GET] Retrieve a representation of the addressed member of
      the collection, expressed in an appropriate Internet media type.

    \item[PUST] Replace the addressed member of the collection, or if
      it does not exist, create it.

    \item[POST] Not generally used. Treat the addressed member as a
      collection in its own right and create a new entry within it
                               
\item[DELETE] Delete the addressed member of the collection. 
\end{description} 
\end{description}

Source:
\URL{https://en.wikipedia.org/wiki/Representational_state_transfer}


Due to the structure that REST provides a number of tools have been
created that manage the creation of the specification for rest
services and their programming. We distinguish several different
categories:

\begin{description}
\item[REST programming language support:] These tools and services are
targeting a particular programming language. Such tools include Eve
which we will explore in more detail.

\item[REST documentation based tools:] These tools are primarily
  focussing on documenting REST specifications. Such tools include Swagger.

\item[REST design support tools:] These tools are used to support the
  design process of developing REST services while abstracting on top
  of the programming languages and define reusable specifications that
  can be used to create clients and servers for particular
  technology targets. Such tools include also swagger as additional
  tools are available that can generate code from swagger specifications
\end{description}

\section{Eve}

Next, we will focus on how to make a RESTful web service with Python
Eve. Eve makes the creation of a REST implementation in python easy. 


\URL{http://python-eve.org/}

\begin{WARNING}

Although we do recommend Ubuntu 17.04, at this time there is a bug
that forces us to use 16.04. Furthermore, we require you to follow the
instructions on how to install pyenv and use it to set up your python
environment. We recommend that you use either python 2.7.14 or 3.6.4.
We do not recommend you to use anaconda as it is not suited for cloud
computing but targets desktop computing. If you use pyenv you also
avoid the issue of interfering with your system wide python
install. We do recommend pyenv regardless if you use a virtual machine
or are working directly on your operating system. After you have set
up a proper python environment, make sure you have the newest version
of pip installed with 

\smallskip

\begin{lstlisting}
pip install pip -U
\end{lstlisting}
\end{WARNING}

To install Eve, you can say 

\begin{lstlisting}
  $ pip install eve
\end{lstlisting}

As Eve also needs a backend database and as MongoDB is an obvious
choice for this, we will also install MongoDB.  MongoDB is a Non-SQL
database which helps to store light weight data easily.

\subsection{Ubuntu}
On Ubuntu you can install MongoDB as follows

\begin{lstlisting}
$ sudo apt-key adv --keyserver hkp://keyserver.ubuntu.com:80 --recv 2930ADAE8CAF5059EE73BB4B58712A2291FA4AD5
$ echo "deb [ arch=amd64,arm64 ] https://repo.mongodb.org/apt/ubuntu xenial/
mongodb-org/3.6 multiverse" | sudo tee /etc/apt/sources.list.d/mongodb-org-3.6.list
$ sudo apt-get update
$ sudo apt-get install -y mongodb-org
\end{lstlisting}
% $

\subsection{OSX}

On OSX you can use the command

\begin{lstlisting}
brew update.
brew install mongodb
\end{lstlisting}

After downloading Mongo, create the {\em db} directory. This is where
the Mongo data files will live. You can create the directory in the
default location and assure it has the right permissions. Make sure
that the /data/db directory has the right permissions by running

\subsection{Verification}

In order to check the MongoDB installation, please run the following
commands in one terminal:

\begin{lstlisting}
$ mkdir -p ~/data/db
$ mongod --dbpath ~/data/db
\end{lstlisting}

In another terminal we try to connect to mongo and issue a mongo
command to show the databases:

\begin{lstlisting}
$ mongo --host 127.0.0.1:27017
$ show databases
\end{lstlisting}

If they execute without errors, you have successfully installed
MongoDB. In order to stop the running database instance run the
following command. simply CTRL-C the running mongod process

\subsection{Building a simple REST Service}

In this section we will focus on creating a simple rest service. To
organize our work we will create the following directory:


\begin{lstlisting}
$ mkdir ~/cloudmesh/eve
$ cd ~/cloudmesh/eve
\end{lstlisting}

As Eve needs a configuration and it is read in by default from the
file \verb|settings.py| we place the following content in the file
\verb|~/cloudmesh/eve/settings.py|:

\begin{lstlisting}
MONGO_HOST = 'localhost'
MONGO_PORT = 27017
MONGO_DBNAME = 'student_db'
DOMAIN = {
    'student': {
        'schema': {
            'firstname': {
                'type': 'string'
            },
            'lastname': {
                'type': 'string'
            },
            'school': {
                'type': 'string'
            },
            'university': {
                'type': 'string'
            },
            'email': {
                'type': 'string',
                 'unique': True
            }
        }
    }
}
RESOURCE_METHODS = ['GET', 'POST']
\end{lstlisting}

The DOMAIN object specifies the format of a \verb|student| object that
we are using as part of our REST service.  In addition we can specify
\verb|RESOURCE_METHODS| which methods are activated for the REST
service. This way the developer can restrict the available methods for
a REST service. To pass along the specification for mongoDB, we simply
specify the hostname, the port, as well as the database name.




Thus our object will be

Initially we need to create the python application to host the REST
API and we will use CURL to do the GET and POST requests in this
tutorial.

\subsection{Creating REST Service}

Create a file called \textbf{run.py} in the same directory where you
placed the \textbf{settings.py} and add the following code.

\begin{lstlisting}
from eve import Eve
app = Eve()

if __name__ == '__main__':
    app.run()
\end{lstlisting}

This is the minimal application which can host a REST
application. Let's start our database server and the REST service.

\subsection{Running REST Service}

Let's say we'll use terminal 1 to run the python REST API.

To start the rest service :
\begin{lstlisting}
$python run.py
\end{lstlisting}

\subsection{Running Database Server}

Next in terminal 2, run the following command to start the database server
\begin{lstlisting}
$ mongod --dbpath=<BASE-PATH>/data/db/
\end{lstlisting}

For example :
BASE-PATH can be replaced by any path you want. 
\begin{lstlisting}
$ mongod --dbpath=/home/john/e222/labs/lab1/data/db/
\end{lstlisting}

\subsection{Saving Data Using Curl}

We can use the commandline to save the data in the database using the
REST api. Here we use the endpoint we created as student in the schema
as the entry point to the rest service. The data is being saved using
JSON format.

Run the following command in the terminal 3, this will save data in
the database server.

\begin{lstlisting}
$ curl -H "Content-Type: application/json" -X POST -d 
'{"firtname":"John","lastname":"Doe", "school" : "ISE",
"university":"IUB", "email":"johndoe@iu.edu"}' http://127.0.0.1:5000/student/
\end{lstlisting}

\newline

In order to check the database please run the following commands in Terminal 4,

\begin{lstlisting}
Run mongo client
$ mongo
Query the databse list
$>show databases; (will display eve)  
Select database
$use eve  
$ show tables # query the table names
$ db.student.find().pretty()  # pretty will show the json in a clear way
\end{lstlisting}

\newline

Here we specify the header using \textbf{H} tag saying we need the
data to be saved using json format. And \textbf{X} tag says the HTTP
protocol and here we use POST method. And the tag \textbf{d} specifies
the data and make sure you use json format to enter the data. Finally,
the REST api endpoint to which we must save data. This allows us to
save the data in a table called \textbf{student} in MongoDB within a
database called \textbf{eve}.

\subsection{Retrieving Data Using Curl}

Run the following command to retrieve the saved data :

\begin{lstlisting}
$ curl http://127.0.0.1:5000/student?firstname=John
\end{lstlisting}



\newline

Here you can use any of the attributes in the saved object to call the
value. If you just used the endpoint itself without specifying the
attribute name, it will load all the saved objects.

\section{Exercises}
\subsection{Payloads}
For this section we will ensure that the proper payloads are received
upon a GET and DELETE request. For this to work ensure that you have a
MongoDB up and running.
\begin{lstlisting}
$ sudo service mongod start
\end{lstlisting}

We will now create the proper script to launch the API and an
appropriate configuration file. \\ The configuration file
(settings.py) should have the following line.
\begin{lstlisting}
DOMAIN = {'people': {}}
\end{lstlisting}

In the same directory as the the configuration file you need to create
a launch script i.e. run.py with the following lines.

\begin{lstlisting}
from eve import Eve
app = Eve()

if __name__ == '__main__':
    app.run()
\end{lstlisting}

Now that you have an instance of MongoDB running and the two newly
created .py files located in the same directory we can launch the
API. In a new terminal navigate to the directory containing the
configuration and launch files and execute the launch script. Again
make sure this is in a seperate terminal so you can see what's
happening as you make requests.

\begin{lstlisting}
$ python run.py
* Running on http://127.0.0.1:5000/
\end{lstlisting}

Now consume the API with the following command and inspect the payload
and identify what request was made.

\begin{lstlisting}
$ curl -i http://127.0.0.1:5000
\end{lstlisting}

Your payload should look like the one below, if your output isn't
formatted like below try adding ?pretty=1 (curl -i
http://127.0.0.1:5000?pretty=1) to the end of the curl command.
\begin{lstlisting}
HTTP/1.0 200 OK
Content-Type: application/json
Content-Length: 150
Server: Eve/0.7.6 Werkzeug/0.11.15 Python/2.7.5
Date: Wed, 17 Jan 2018 18:34:07 GMT

{
    "_links": {
        "child": [
            {
                "href": "people",
                "title": "people"
            }
        ]
    }
\end{lstlisting}

Remember that the API entry points adhere to the HATEOAS principle,
thus provide information regarding the resources accessible through
the API. In this case how many child resources are available through
our API?

\vspace{5mm}

If you said (1) you're correct! It's the one resource we defined in
the configuration file 'people'. Now lets try to request people.

\begin{lstlisting}
$ curl -i http://127.0.0.1:5000/people
\end{lstlisting}

How does the payload change? What does the \_links section describe?
Notice the difference between the payload when people is requested.

\begin{lstlisting}
{
    "_items": [],
    "_links": {
        "self": {
            "href": "people",
            "title": "people"
        },
        "parent": {
            "href": "/",
            "title": "home"
        }
    },
    "_meta": {
        "max_results": 25,
        "total": 0,
        "page": 1
    }
}
\end{lstlisting} 


\begin{exercise}
Write a RESTful service to determine a useful piece of information off
of your computer i.e. disk space, memory, RAM, etc.
\end{exercise}




\section{Object Management with Eve and Evegenie}

\url{http://python-eve.org/}

Eve makes the creation of a REST implementation in python easy.  We
will provide you with an implementation example that showcases that we
can create REST services without writing a single line of code. The
code for this is located at \url{https://github.com/cloudmesh/rest}

This code will have a master branch but will also have a dev branch in
which we will add gradually more objects. Objects in the dev branch will
include:

\begin{itemize}
\item
 virtual directories
\item
 virtual clusters
\item
 job sequences
\item
 inventories
\end{itemize}

You may want to check our active development work in the dev branch.
However for the purpose of this class the master branch will be
sufficient.

\subsection{Installation}\label{installation}

First we have to install mongodb. The installation will depend on your
operating system. For the use of the rest service it is not important to
integrate mongodb into the system upon reboot, which is focus of many
online documents. However, for us it is better if we can start and stop
the services explicitly for now.

On ubuntu, you need to do the following steps:

\begin{lstlisting}
TO BE CONTRIBUTED BY THE STUDENTS OF THE CLASS AS HOMEWORK
\end{lstlisting}

On windows 10, you need to do the following steps:

\begin{lstlisting}
TO BE CONTRIBUTED BY THE STUDENTS OF THE CLASS AS HOMEWORK. If you
elect Windows 10. YOu could be using the online documentation
provided by starting it on Windows, or running it in a docker container.
\end{lstlisting}

On OSX you can use home-brew and install it with:

\begin{lstlisting}
brew update
brew install mongodb
\end{lstlisting}

In future we may want to add ssl authentication in which case you
may
need to install it as follows:

\begin{lstlisting}
brew install mongodb --with-openssl
\end{lstlisting}

\subsection{Starting the service}\label{starting-the-service}

We have provided a convenient Makefile that currently only works for
OSX. It will be easy for you to adapt it to Linux. Certainly you can
look at the targes in the makefile and replicate them one by one.
Important targets are deploy and test.

When using the makefile you can start the services with:

\begin{lstlisting}
make deploy
\end{lstlisting}

IT will start two terminals. IN one you will see the mongo service, in
the other you will see the eve service. The eve service will take a file
called sample.settings.py that is base on sample.json for the start of
the eve service. The mongo servide is configured in sucj a way that it
only accepts incoming connections from the local host which will be
sufficient for our case. The mongo data is written into the
\verb|$USER/.cloudmesh| directory, so make sure it exists.

To test the services you can say:

\begin{lstlisting}
make test
\end{lstlisting}

YOu will se a number of json text been written to the screen.

\section{Creating your own objects}\label{creating-your-own-objects}

The example demonstrated how easy it is to create a mongodb and an eve
rest service. Now lets use this example to create your own. FOr this we
have modified a tool called evegenie to install it onto your system.

The original documentation for evegenie is located at:

\begin{itemize}
\tightlist
\item
 \url{http://evegenie.readthedocs.io/en/latest/}
\end{itemize}

However, we have improved evegenie while providing a commandline tool
based on it. The improved code is located at:

\begin{itemize}
\tightlist
\item
 \url{https://github.com/cloudmesh/evegenie}
\end{itemize}

You clone it and install on your system as follows:

\begin{lstlisting}
cd ~/github
git clone https://github.com/cloudmesh/evegenie
cd evegenie
python setup.py install
pip install .
\end{lstlisting}

This should install in your system evegenie. YOu can verify this by
typing:

\begin{lstlisting}
which evegenie
\end{lstlisting}

If you see the path evegenie is installed. With evegenie installed its
usaage is simple:

\begin{lstlisting}
$ evegenie

Usage:
  evegenie --help
  evegenie FILENAME
\end{lstlisting}

It takes a json file as input and writes out a settings file for the use
in eve. Lets assume the file is called sample.json, than the settings
file will be called sample.settings.py. Having the evegenie programm
will allow us to generate the settings files easily. You can include
them into your project and leverage the Makefile targets to start the
services in your project. In case you generate new objects, make sure
you rerun evegenie, kill all previous windows in which you run eve and
mongo and restart. In case of changes to objects that you have designed
and run previously, you need to also delete the mongod database.

\section{Towards cmd5 extensions to manage eve and
mongo}\label{towards-cmd5-extensions-to-manage-eve-and-mongo}

Naturally it is of advantage to have in cms administration commands to
manage mongo and eve from cmd instead of targets in the Makefile. Hence,
we \textbf{propose} that the class develops such an extension. We will
create in the repository the extension called admin and hobe that
students through collaborative work and pull requests complete such an
admin command.

The proposed command is located at:

 \URL{https://github.com/cloudmesh/rest/blob/master/cloudmesh/ext/command/admin.py}


It will be up to the class to implement such a command. Please
coordinate with each other.

The implementation based on what we provided in the Make file seems
straight forward. A great extension is to load the objects definitions
or eve e.g. settings.py not from the class, but forma place in
.cloudmesh. I propose to place the file at:

\begin{lstlisting}
.cloudmesh/db/settings.py
\end{lstlisting}

the location of this file is used when the Service class is initialized
with None. Prior to starting the service the file needs to be copied
there. This could be achieved with a set command.

\section{Responses}

\URL{https://dzone.com/refcardz/rest-foundations-restful}


\begin{comment}


\begin{lstlisting}
Code  Description

200 OK. The request has successfully executed. Response depends upon the verb invoked.
201 Created. The request has successfully executed and a new resource has been created in the process. The response body is either empty or contains a representation containing URIs for the resource created. The Location header in the response should point to the URI as well.
202 Accepted. The request was valid and has been accepted but has not yet been processed. The response should include a URI to poll for status updates on the request. This allows asynchronous REST requests
204 No Content. The request was successfully processed but the server did not have any response. The client should not update its display.
Table 1 - Successful Client Requests


Redirected Client Requests

CODE  DESCRIPTION
301 Moved Permanently. The requested resource is no longer located at the specified URL. The new Location should be returned in the response header. Only GET or HEAD requests should redirect to the new location. The client should update its bookmark if possible.
302 Found. The requested resource has temporarily been found somewhere else. The temporary Location should be returned in the response header. Only GET or HEAD requests should redirect to the new location. The client need not update its bookmark as the resource may return to this URL.
303 See Other. This response code has been reinterpreted by the W3C Technical Architecture Group (TAG) as a way of responding to a valid request for a non-network addressable resource. This is an important concept in the Semantic Web when we give URIs to people, concepts, organizations, etc. There is a distinction between resources that can be found on the Web and those that cannot. Clients can tell this difference if they get a 303 instead of 200. The redirected location will be reflected in the Location header of the response. This header will contain a reference to a document about the resource or perhaps some metadata about it.




Invalid Client Requests

Code  Description
405 Method Not Allowed.
406 Not Acceptable.
410 Gone.
411 Length Required.
412 Precondition Failed.
413 Entity Too Large.
414 URI Too Long.
415 Unsupported Media Type.
417 Expectation Failed.


Code  Description
500 Internal Server Error.
501 Not Implemented.
503 Service Unavailable.

\end{lstlisting}

\end{comment}



\section{Swagger}

Swagger \url{https://swagger.io/} is a tool for developing API
specifications based on the OpenAPI Specification (OAS). It allows not
only the specification, but the generation of code based on the
specification in a variety of languages.

Swagger itself has a number of tools which together build a framework
for developing REST services for a variety of languages.


\section{Swagger Tools}

The major Swagger tools of interest are:

\begin{description}

\item[Swagger Core] includes libraries for working with Swagger
 specifications \url{https://github.com/swagger-api/swagger-core}.

\item[Swagger Codegen] allows to generate code from the specifications
 to develop Client SDKs, servers, and documentation. \url{https://github.com/swagger-api/swagger-codegen}

\item[Swagger UI] is an HTML5 based UI for exploring and interacting
 with the specified APIs \url{https://github.com/swagger-api/swagger-ui}

\item[Swagger Editor] is a Web-browser based editor for composing 
 specifications using YAML \url{https://github.com/swagger-api/swagger-editor}

\end{description}

The developed APIs can be hosted and further developed on an
online repository named SwaggerHub \url{https://app.swaggerhub.com/home}
The convenient online editor is available which also can be installed
locally on a variety of operating systems including OSX, Linux, and
Windows. 


\section{FlaskRESTful}

Flask-RESTful provides a framework for developing REST APIs on top of
Flask. The abstraction introduce tha ability to use traditional existing
ORM libraries. The goal is to enable a minimal setup and allow
developers to leverage the familiarity they have with Flask and
develop Flask-RESTful services quickly.

\URL{https://flask-restful.readthedocs.io/en/latest/}

\section{Django REST Framework}

\URL{http://www.django-rest-framework.org/}

Django REST framework is a large toolkit to develop Web APIs. The
developers of the framework provide the following reasons for using it:

``(1) The Web browsable API is a huge usability win for your
developers.  (2) Authentication policies including packages for
OAuth1a and OAuth2.  (3) Serialization that supports both ORM and
non-ORM data sources.  (4) Customizable all the way down - just use
regular function-based views if you don't need the more powerful
features.  (5) Extensive documentation, and great community support.
(6) Used and trusted by internationally recognised companies including
Mozilla, Red Hat, Heroku, and Eventbrite.''

