\chapterimage{rest.png} 

\section{REST Service Generation with Swagger}
\label{c:swagger-codegen}

\video{REST}{36:02}{Swagger}{https://youtu.be/0_Ub13py_K8}

In this subsection we are discussing how to use OpenAPI 2.0 and Swagger
Codegen to define and develop a REST Service.

We assume you have been familiar with the concept of REST service,
OpenAPI and Swagger as discussed in Sections
\ref{s:rest-intro} and \ref{S:swagger-specification}.

We will use a simple example to demonstrate the process of developing a
REST service with Swagger/OpenAPI 2.0 specification and the tools
related to is. The general steps are:

\begin{itemize}
\item Step 1 (Section \ref{s:step-1-define-your-rest-service}). Define
  the REST service conforming to Swagger/OpenAPI 2.0 specification. It
  is a YAML document file with the basics of the REST service defined,
  e.g., what resources it has and what actions are supported.
\item Step 2 (Section \ref{s:step-2-swagger-code-gen}). Use Swagger
  Codegen to generate the server side stub code.  Fill in the actual
  implementation of the business logic portion in the code.
\item Step 3 (Section \ref{s:step-3-swagger-codegen}). Install the
  server side code and run it. The service will then be available.
\item Step 4 (Section \ref{s:step-4-swagger-codegen}). Generate client
  side code. Develop code to call the REST service. Install and run to
  verify.
\end{itemize}

\subsection{Step 1: Define Your REST  Service}
\label{s:step-1-define-your-rest-service}

In this example we define a simple REST service that returns the hosting
server's basic CPU info. The example specification in yaml is as
follows:

\begin{lstlisting}
swagger: "2.0"
info: 
  version: "0.0.1"
  title: "cpuinfo"
  description: "A simple service to get cpuinfo as an example of using swagger-2.0 specification and codegen"
  termsOfService: "http://swagger.io/terms/"
  contact: 
    name: "Cloudmesh REST Service Example"
  license: 
    name: "Apache"
host: "localhost:8080"
basePath: "/api"
schemes: 
  - "http"
consumes: 
  - "application/json"
produces: 
  - "application/json"
paths: 
  /cpu:
    get: 
      description: "Returns cpu information of the hosting server"
      produces: 
        - "application/json"
      responses: 
        "200":
          description: "CPU info"
          schema: 
            $ref: "#/definitions/CPU"
definitions:
  CPU:
    type: "object"
    required: 
      - "model"
    properties: 
      model:
        type: "string"
\end{lstlisting}
% $


\subsection{Step 2: Server Side Stub Code Generation and
Implementation}\label{s:step-2-swagger-code-gen}

With the REST service having been defined, we can now generate the
server side stub code easily.

\subsubsection{Setup the Codegen
Environment}\label{setup-the-codegen-environment}

You will need to \href{https://swagger.io/docs/swagger-tools/}{install
the Swagger Codegen tool} if not yet done so. For OSX we recommend
that you use the homebrew install via 

\begin{lstlisting}
brew install swagger-codegen
\end{lstlisting}

On Ubuntu you can install swagger as follows (update the version as needed):

\begin{lstlisting}
 mkdir ~/swagger
cd ~/swagger
wget https://oss.sonatype.org/content/repositories/releases/io/swagger/swagger-codegen-cli/2.3.1/swagger-codegen-cli-2.3.1.jar
alias swagger-codegen="java -jar ~/swagger/swagger-codegen-cli-2.3.1.jar"
\end{lstlisting}

Add the alias to your \verb|.bashrc| or \verb|.bash_profile| file. After you start a
new terminal you can use in that terminal now the command 

\begin{lstlisting}
swagger-codegen
\end{lstlisting}


For other platforms you can just use the \texttt{.jar} file, which can be downloaded
from
\href{https://oss.sonatype.org/content/repositories/releases/io/swagger/swagger-codegen-cli/2.3.1/swagger-codegen-cli-2.3.1.jar}{this
link}.

Also make sure Java 7 or 8 is installed in your system. To have a well
defined location we recommend that you place the code in the directory
\texttt{\textasciitilde{}/cloudmesh}. In this directory you will also
place the \texttt{cpu.yaml} file.

\subsubsection{Generate Server Stub
Code}\label{generate-server-stub-code}

After you have the codegen tool ready, and with Java 7 or 8 installed in
your system, you can run the following to generate the server side stub
code:

\begin{lstlisting}
swagger-codegen generate \
    -i ~/cloudmesh/cpu.yaml \
    -l python-flask \
    -o ~/cloudmesh/swagger_example/server/cpu/flaskConnexion \
    -D supportPython2=true
\end{lstlisting}

or if you have not created an alias 

\begin{lstlisting}
java -jar swagger-codegen-cli.jar generate \
    -i ~/cloudmesh/cpu.yaml \
    -l python-flask \
    -o ~/cloudmesh/swagger_example/server/cpu/flaskConnexion \
    -D supportPython2=true
\end{lstlisting}



In the specified directory under \emph{flaskConnexion} you will find the
generated python flask code, with python 2 compatibility. It is best to
place the swagger code under the directory \verb|~/cloudmesh|
to have a location where you can easily find it. If you want to use
python 3 make sure to chose the appropriate option. To switch between
python 2 and python 3 we recommend that you use pyenv as discussed in
our python section.

\subsubsection{Fill in the actual
implementation}\label{s:swagger-codegen-implementation}

Under the \emph{flaskConnecxion} directory, you will find a
\emph{swagger\_server} directory, under which you will find directories
with \emph{models} defined and \emph{controllers} code stub resides. The
models code are generated from the definition in Step 1. On the
controller code though, we will need to fill in the actual
implementation. You may see a \texttt{default\_controller.py} file under
the \emph{controllers} directory in which the resource and action is
defined but yet to be implemented. In our example, you will find such a
function definition which we list next:

\begin{lstlisting}
def cpu_get():  # noqa: E501
    """cpu_get

    Returns cpu info of the hosting server # noqa: E501


    :rtype: CPU
    """
    return 'do some magic!'
\end{lstlisting}

Please note the \texttt{do\ some\ magic!} string at the return of the
function. This ought to be replaced with actual implementation what you
would like your REST call to be really doing. In reality this could be
some call to a backend database or datastore; a call to another API; or
even calling another REST service from another location. In this example
we simply retrieve the cpu model information from the hosting server
through a simple python call to illustrate this principle. Thus you can
define the following code:

\begin{lstlisting}
import os, platform, subprocess, re

def get_processor_name():
    if platform.system() == "Windows":
        return platform.processor()
    elif platform.system() == "Darwin":
        command = "/usr/sbin/sysctl -n machdep.cpu.brand_string"
        return subprocess.check_output(command, shell=True).strip()
    elif platform.system() == "Linux":
        command = "cat /proc/cpuinfo"
        all_info = subprocess.check_output(command, shell=True).strip()
        for line in all_info.split("\n"):
            if "model name" in line:
                return re.sub(".*model name.*:", "", line, 1)
    return "cannot find cpuinfo"
\end{lstlisting}

And then change the \textbf{cpu\_get()} function to the following:

\begin{lstlisting}
def cpu_get():  # noqa: E501
    """cpu_get

    Returns cpu info of the hosting server # noqa: E501


    :rtype: CPU
    """
    return CPU(get_processor_name())
\end{lstlisting}

Plese note we are returning a CPU object as defined in the API and later
generated by the codegen tool in the \emph{models} directory.

It is best \emph{not} to include the definition of
\texttt{get\_processor\_name()} in the same file as you see the
definition of \texttt{cpu\_get()}. The reason for this is that in case
you need to regenerate the code, your modified code will naturally be
overwritten. Thus, to minimize the changes, we do recommend to maintain
that portion in a different filename and import the function as to keep
the modifications small.

At this step we have completed the server side code development.


\subsection{Step 3: Install and Run the REST
Service:}\label{s:step-3-swagger-codegen}

Now we can install and run the REST service. It is strongly recommended
that you run this in a pyenv or a virtualenv environment.

\subsubsection{Start a virtualenv:}\label{start-a-virtualenv}

In case you are not using pyenv, please use virtual env as follows:

\begin{lstlisting}
virtualenv RESTServer
source RESTServer/bin/activate
\end{lstlisting}

\subsubsection{Make sure you have the latest
pip:}\label{make-sure-you-have-the-latest-pip}

\begin{lstlisting}
pip install -U pip
\end{lstlisting}

\subsubsection{Install the requirements of the server side
code:}\label{install-the-requirements-of-the-server-side-code}

\begin{lstlisting}
cd ~/cloudmesh/swagger_example/server/cpu/flaskConnexion
pip install -r requirements.txt
\end{lstlisting}

\subsubsection{Install the server side code
package:}\label{install-the-server-side-code-package}

Under the same directory, run:

\begin{lstlisting}
python setup.py install
\end{lstlisting}

\subsubsection{Run the service}\label{run-the-service}

Under the same directory:

\begin{lstlisting}
python -m swagger_server
\end{lstlisting}

You should see a message like this:

\begin{lstlisting}
* Running on http://0.0.0.0:8080/ (Press CTRL+C to quit)
\end{lstlisting}

\subsubsection{Verify the service using a web
browser:}\label{verify-the-service-using-a-web-browser}

Open a web browser and visit:

\begin{itemize}
\tightlist
\item
  http://localhost:8080/api/cpu
\end{itemize}

to see if it returns a json object with cpu model info in it.

Assignment: How would you verify that your service works with a
\texttt{curl} call?

\subsection{Step 4: Generate Client Side Code and
Verify}\label{s:step-4-swagger-codegen}

In addition to the server side code swagger can also create a client
side code.

\subsubsection{Client side code
generation:}\label{client-side-code-generation}

Generate the client side code in a similar fashion as we did for the
server side code:

\begin{lstlisting}
java -jar swagger-codegen-cli.jar generate \
    -i ~/cloudmesh/cpu.yaml \
    -l python \
    -o ~/cloudmesh/swagger_example/client/cpu \
    -D supportPython2=true
\end{lstlisting}

\subsubsection{Install the client side code
package:}\label{install-the-client-side-code-package}

Although we could have installed the client in the same python pyenv or
virtualenv, we showcase here that it can be installed in a completely
different environment. That would make it even posible to use a python 3
based client and a python 2 based server showcasing interoperability
between python versions (although we just use python 2 here). Thus we
create ane new python virtual environment and conduct our install.

\begin{lstlisting}
virtualenv RESTClient
source RESTClient/bin/activate
pip install -U pip
cd swagger_example/client/cpu
pip install -r requirements.txt
python setup.py install
\end{lstlisting}

\subsubsection{Using the client API to interact with the REST
service}\label{using-the-client-api-to-interact-with-the-rest-service}

Under the directory \emph{swagger\_example/client/cpu} you will find a
README.md file which serves as an API documentation with example client
code in it. E.g., if we save the following code into a .py file:

\begin{lstlisting}
from __future__ import print_function
import time
import swagger_client
from swagger_client.rest import ApiException
from pprint import pprint
# create an instance of the API class
api_instance = swagger_client.DefaultApi()

try:
    api_response = api_instance.cpu_get()
    pprint(api_response)
except ApiException as e:
    print("Exception when calling DefaultApi->cpu_get: %s\n" % e)
\end{lstlisting}

We can then run this code to verify the calling to the REST service
actually works. We are expecting to see a return similar to this:

\begin{lstlisting}
{'model': 'Intel(R) Core(TM)2 Quad CPU    Q9550  @ 2.83GHz'}
\end{lstlisting}

Obviosly, we could have applied additional cleanup of the information
returned by the python code, such as removing duplicated spaces.

\subsection{Towards a Distributed Client
Server}\label{towards-a-distributed-client-server}

Although we develop and run the example on one localhost machine, you
can separate the process into two separate machines. E.g., on a server
with external IP or even DNS name to deploy the server side code, and on
a local laptop or workstation to deploy the client side code. In this
case please make changes on the API definition accordingly, e.g., the
\textbf{host} value.

\subsection{Exercises}\label{s:swagger-exercises}

\begin{exercise}
  In Section \ref{s:step-1-define-your-rest-service}, we introduced a
  schema. The question relates to termsOfService: Investigate what the
  termOfService attribut is and suggest a better value. Discuss on
  piazza.
\end{exercise}

\begin{exercise}
  In Section \ref{s:step-1-define-your-rest-service}, we introduced a
  schema. The question relates to model: What is the meaning of model
  under the definitions
\end{exercise}

\begin{exercise}
  In Section \ref{s:step-1-define-your-rest-service}, we introduced a
  schema. The question relates to \$ref: what is the meaning of the
  \$ref. Discuss on piazza, come up with a student answer in class.
\end{exercise}

\begin{exercise}
  In Section \ref{s:step-1-define-your-rest-service}, we introduced a
  schema.  What does the response 200 mean. Do you need other
  responses?
\end{exercise}

\begin{exercise}
  After you have gone through the entire section and verified it works
  for you add create a more sophisticated schema and add more attributes
  exposing more information from your system.
\end{exercise}

\begin{exercise}
  How can you for example develop a rest service that exposes portions
  of your file system serving large files, e.g.~their filenames and
  their size? How would you download these files? Would you use a rest
  service, or would you register an alternative service such as ftp,
  DAV, or others? Please discuss in piazza. Note this will be a helping
  you to prepare a larger assignment. Think abou this first before you
  implement.
\end{exercise}


\begin{exercise}
You can try expand the API definition with more resources and
actions included. E.g., to include more detailed attributes in the CPU
object and to have those information provided in the actual
implementation as well. Or you could try defining totally different
resources.
\end{exercise}

\begin{exercise}
The codegen tool provides a convenient way to have the code
stubs ready, which frees the developers to focus more on the API
definition and the real implementation of the business logics. Try with
complex implementation on the back end server side code to interact with
a database/datastore or a 3rd party REST service.
\end{exercise}

\begin{exercise}
For advanced python users, you can naturally use function assignments to
replace the \texttt{cpu\_get()} entirely even after loading the
instantiation of the server. However, this is not needed. If you are an
advanced python developer, please feel free to experiment and let us
know how you suggest to integrate things easily.
\end{exercise}