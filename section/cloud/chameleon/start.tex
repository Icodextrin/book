\FILENAME

\chapter{Getting Started}\label{getting-started}

Here we describe how you can get access to chameleon clude under the
assumption that you are a student or a researcher that joins an
existing project on Chameleon cloud. You will need to follow the
following steps:

\section{Step 1: Create~a Chameleon~account}\label{accounts}

To get started using Chameleon you will need
to\textbf{\href{https://www.chameleoncloud.org/register}{~create a
user~account}.}

You will be asked to agree to the
\href{https://www.chameleoncloud.org/terms/view/chameleon-user-terms/}{Chameleon
terms and conditions} which, among others, ask you to acknowledge the
use of Chameleon in your publications.

As part of creating an account you may request PI status, which means
that you will be able to create and lead Chameleon projects (see Step 2
below). 

\section{Step 2: Create or join a project}\label{allocations}

To use Chameleon, you will need to be associated with a
\href{https://www.chameleoncloud.org/docs/user-faq/\#toc-how-do-i-apply-for-a-chameleon-project-}{project}
that is assigned an
\href{https://www.chameleoncloud.org/docs/user-faq/\#toc-what-are-the-project-allocation-sizes-and-limits-}{allocation}.
This means that you either need to
(1) \textbf{\href{https://www.chameleoncloud.org/user/projects/new/}{apply
for a new project}} or (2)
\textbf{\href{https://www.chameleoncloud.org/docs/user-faq/\#toc-my-pi-professor-colleague-already-has-a-chameleon-project-how-do-i-get-added-as-a-user-on-the-project-}{ask
the PI of an existing Chameleon project to add you}.}

A project is headed by a project PI, typically
\href{https://www.chameleoncloud.org/docs/user-faq/\#toc-who-is-eligible-to-be-chameleon-pi-and-how-do-i-make-sure-that-my-pi-status-is-reflected-in-my-profile-}{a
faculty member or researcher scientist at a scientific institution}. If
you are a student we recommend that you ask your professor to work with
you on creating a project. Please note taht you must not create a
project by yourself and that you indeed need to work with your
proferrsor. 

A project application typically consists of about one paragraph
description of the intended research and takes one buisness day to
process. 

Enrolling you into an existing research or class project depends on
the time availability of the project lead or professor of your
class. It is important that you communicate your chameleon cloud
account name to the project lead so they can easily add you. Make sure
you realy give them only your chameleon coount name and potentially
your organizational e-mail, Firstname, and Lastname so they can check
you are realy eligible to get access.

\section{Step 3: Start using Chameleon!}\label{using-chameleon}

Now that you have enrolled nad are added to the project by your
project lead you cans tart using chameleon cloud. However be minded
that you ought to shut down the resources/VMs whenever they are not in
use to avoid unnecessary charging. Remember the project has imited
time on chameleon and any unused time will be charged against the project.

Chameleon provides two types of resources with links to their respective
users guides below:

\textbf{\href{https://www.chameleoncloud.org/docs/bare-metal-user-guide-old/}{Bare
Metal User Guide}} will tell you how to use Chameleon bare metal
resources which provide strong isolation and allow you maximum control
(reboot to new operating system, reboot the kernel, etc.)

\textbf{\href{https://www.chameleoncloud.org/docs/user-guides/openstack-kvm-user-guide/}{OpenStack
KVM User Guide}} will tell you how to get started with Chamemeleon's
OpenStack KVM cloud which is a multi-tenant environment providing weak
performance isolation. 

If you have any questions or encounter any problems, you can check out
our \href{https://www.chameleoncloud.org/docs/user-faq/}{User FAQ},
or \href{https://www.chameleoncloud.org/user/help/}{submit a ticket}.

