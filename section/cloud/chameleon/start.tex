
\chapter{Getting Started}\label{getting-started}

Welcome! We are glad you want to use Chameleon for your Computer Science
experiments. We will get you on your way in these 3 easy steps.~

\section{Step 1: Create~a Chameleon~account}\label{accounts}

To get started using Chameleon you will need
to\textbf{\href{https://www.chameleoncloud.org/register}{~create a
user~account}.}~

You will be asked to agree to the
\href{https://www.chameleoncloud.org/terms/view/chameleon-user-terms/}{Chameleon
terms and conditions}~which, among others, ask you to acknowledge the
use of Chameleon in your publications.~

As part of creating an account you may request~PI status, which means
that you~will be able to create and lead~Chameleon projects (see Step 2
below).~

\section{Step 2: Create or join a project}\label{allocations}

To use Chameleon, you will need to be associated with a
\href{https://www.chameleoncloud.org/docs/user-faq/\#toc-how-do-i-apply-for-a-chameleon-project-}{project}
that is assigned an
\href{https://www.chameleoncloud.org/docs/user-faq/\#toc-what-are-the-project-allocation-sizes-and-limits-}{allocation}.
This means that you either need to
(1)~\textbf{\href{https://www.chameleoncloud.org/user/projects/new/}{apply
for a new project}~}or (2)
\textbf{\href{https://www.chameleoncloud.org/docs/user-faq/\#toc-my-pi-professor-colleague-already-has-a-chameleon-project-how-do-i-get-added-as-a-user-on-the-project-}{ask
the PI of an existing Chameleon project to add you}.}

A project~is headed by a project PI, typically
\href{https://www.chameleoncloud.org/docs/user-faq/\#toc-who-is-eligible-to-be-chameleon-pi-and-how-do-i-make-sure-that-my-pi-status-is-reflected-in-my-profile-}{a
faculty member or researcher scientist at a scientific institution}.~If
you are a student we recommend that you ask your professor to work with
you on creating a project.~

A project application typically consists of about one paragraph
description of the intended research and takes one buisness day to
process.~

\section{Step 3: Start using Chameleon!}\label{using-chameleon}

Congratulations, you are now ready to go!

Chameleon provides two types of resources with links to their respective
users guides below:

\textbf{\href{https://www.chameleoncloud.org/docs/bare-metal-user-guide-old/}{Bare
Metal User Guide}~}will tell you how to use Chameleon bare metal
resources which provide strong isolation and allow you maximum control
(reboot to new operating system, reboot the kernel, etc.)

\textbf{\href{https://www.chameleoncloud.org/docs/user-guides/openstack-kvm-user-guide/}{OpenStack
KVM User Guide}~}will tell you how to get started with~Chamemeleon's
OpenStack KVM cloud which is a multi-tenant environment providing weak
performance isolation.~

If you~have any questions or encounter any problems, you can check out
our \href{https://www.chameleoncloud.org/docs/user-faq/}{User FAQ},
or~\href{https://www.chameleoncloud.org/user/help/}{submit a ticket}.

