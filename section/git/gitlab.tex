\section{Gitlab}\label{gitlab}

In case your course the use of \href{https://gitlab.com/}{GitLab.com}
for your homework submissions this information will help you to get
stated.

Once you have completed the entry survey you will be granted access to a
gitlab repository in which to develop your homework submissions.

The repository organization is determined by the class and specific
instructions will be given to you. This may include:

\begin{verbatim}
report
proposal
report
code
paper1
paper2
paper3
bib
\end{verbatim}

Important is that the the use of \textbf{lower case} directories, for
the main directories in the reporsitory

\begin{description}
\item[Please use only \textbf{lowercase} characters in the directory]
names and no special characters such as @ ; / \_ and spaces. In general
we recommend that you avoid using directoru names with capital letters
spaces and \_ in them. This will simplify your documentation efforts and
make the URLs from git more readable. Also while on some OS's the
directories ``MyDirectory'' is different from ``mydirectory'' gitlab
considers it teh same. Furthermore our automated scripts that check your
submission are case sensitive. We only will read the lower case
directories. It is in your responsibility to assure proper spelling.
\end{description}

\subsection{Getting an account}\label{getting-an-account}

Please go to gitlab and create an account. Use a nice account name that
only includes characters in {[}a-zA-Z0-9{]}.

  \URL{http://gitlab.com}


\subsection{Getting a repository}\label{getting-a-repository}

Once you submitted the account to us you will recieve within a week an
e-mail with your repository. It will be based on a Homework-ID (HID)
that will be assigned to you by us. The HID will be the name of the
directory in gitlab that you will be using to submit your homework.

\subsection{Upload your public key}\label{upload-your-public-key}

Please upload your public key to the repository as documented in gitlab.

\subsection{How to configure Git and Gitlab for your
computer}\label{how-to-configure-git-and-gitlab-for-your-computer}

The proper way to use git is to install a client on your computer. Once
you have done so, make sure to configure git to use your name and email
address label your commits.:

\begin{verbatim}
$ git config --global user.name "Albert Einstein"
$ git config --global user.email albert@iu.edu
\end{verbatim}

Do this on any computer you want to make direct checkins into gitlab.

Make sure to substitute in your name and email address in the commands
above.

You should also configure the push behavior to push only matching
branches. See the \href{https://git-scm.com/docs/git-config}{git
documentation} for more details on what this means.:

\begin{verbatim}
$ git config --global push.default matching
\end{verbatim}

\subsection{Using Web browsers to
upload}\label{using-web-browsers-to-upload}

Although we do not recommend using the Web browser to add or modify
files, it is possible. This means you could operate it without
installing anything. This will work, but it is not very convenient. To
conduct your project efficiently you certainly wil want to use the git
command line inteface.

\subsection{Using Git GUI tools}\label{using-git-gui-tools}

There are many git GUI tools available that directly integrate into your
operating system finders, windows, \ldots{}, or PyCharm. It is up to you
to identify such tools and see if they are useful for you. Most of the
people we work with us git from the command line, even if they use
PyCharm or other tools that have build in git support. We find the
commandline tools sufficient.

\subsection{Submission of homework}\label{submission-of-homework}

You will have a HID given to you. Let us assume the id is:

\begin{verbatim}
S17-DG-9999
\end{verbatim}

When you log into gitlab, you will find a directory with that name.
Please substitute the HID that we gave above as an example with your
own. We refer to this ID as \textless{}HID\textgreater{} in these
instructions.

THis section will be updated

Now you can go to your web browser and past the following URL into it,
where you replace the \textless{}HID\textgreater{} with your HID that
you can find in Canvas.:

\begin{verbatim}
https://gitlab.com/cloudmesh_spring2017/<HID>
\end{verbatim}

For our example this would result in:

\begin{verbatim}
https://gitlab.com/cloudmesh_spring2017/S16-DG-9999
\end{verbatim}

You will be responsible to create the directory structure in git
following the guidelines of your class.

To submit the homework you need to first clone the repository (read the
git manual about what cloning means):

\begin{verbatim}
git clone https://gitlab.com/cloudmesh/fall2016/HID
\end{verbatim}

Your homework for submission should be organized according to folders in
your clone repository. To submit a particular assignment, you must first
add it using:

\begin{verbatim}
git add <name of the file you are adding>
\end{verbatim}

Afterwards, commit it using:

\begin{verbatim}
git commit -m "message describing your submission"
\end{verbatim}

Then push it to your remote repository using:

\begin{verbatim}
git push
\end{verbatim}

If you want to modify your submission, you only need to:

\begin{verbatim}
git commit -m "message relating to updated file"
\end{verbatim}

afterwards:

\begin{verbatim}
git push
\end{verbatim}

\begin{description}
\item[If you lose any documents locally, you can retrieve them from
your]
remote repository using:

\begin{verbatim}
git pull
\end{verbatim}
\end{description}

If you have any issues, please post your question in the folder gitlab.
Our TAs will answer them.

\subsection{README.md}\label{readme.md}

You will have to create a README.md file in the top most directory of
your repository It serves the purpose of identifying your submission for
homework and information about yourself.

It is important to follow the format precisely. Any derivation from this
format will not allow us to see your homework as our automated scripts
will use the README.rst to detect them. Please also mind that all
filenames of all homework and the main directory must be
\textbf{lowercase} as explained above.

Naturally you \textbf{MUST} use the firstname and lastname that you used
in CANVAS so we can identify you in CANVAS properly. If you use an alias
that is not used in CANVAS we can naturally not identify you. Each
homework will have a single line in the readme. Once you have completed
a homework, please check it of with an {[}x{]}. There is no need to ask
us if your submisison was received, we will delete such request and not
answer them. A program will inspect your submission and a list will be
produced with all submissions included. THis list will be updated on
regular baseis and published for the class. If we require additional
fields we will announce this and you will need to add them. When we
request to update the README.md is a must and can not be delayed.

\begin{verbatim}
- [ ] author: Firtsname, Lastname
- [ ] hid: TBD
- [ ] github: githubusername (if used for class)
- [ ] gitlab: gitlabusername (if used for class)
- [ ] paper1: paper1/paper.pdf, not stated, date of submission
- [ ] paper2: paper2/paper.pdf, not stated, date of submission
- [ ] paper3: paper2/paper.pdf, not stated, date of submission
- [ ] proposal: report/proposal.pdf, not started, date of submission
- [ ] midterm: report/midterm.pdf, not started, date of submission
- [ ] report: report/report.pdf, not started, date of submission
\end{verbatim}

\subsection{Git Resources}\label{git-resources}

If you are unfamiliar with git you may find these resources useful:

\begin{itemize}
\tightlist
\item
  \href{https://git-scm.com/book/en/v2}{Pro Git book}
\item
  \href{https://git-scm.com/docs/gittutorial}{Official tutorial}
\item
  \href{https://git-scm.com/doc}{Official documentation}
\item
  \href{http://www.tutorialspoint.com/git/}{TutorialsPoint on git}
\item
  \href{https://try.github.io}{Try git online}
\item
  \href{https://help.github.com/articles/good-resources-for-learning-git-and-github/}{GitHub
  resources for learning git} Note: this is for github and not for
  gitlab. However as it is for gt the only thing you have to do is
  replace hihub, for gitlab.
\item
  \href{https://www.atlassian.com/git/tutorials/}{Atlassian tutorials
  for git}
\end{itemize}

\subsection{Exercise}\label{exercise}

Gitlab.1: Create a gitlab account

Gitlab.2: Create a README.md in your gitlab account with your
information
