\FILENAME

\chapter{Linux Shell}
\label{C:linux-shell}

There are many good tutorials out there that explain why one needs a
linux shell and not just a GUI. Randomly we picked the firts one that
came up with a google query (This is not an endorsement for the material
we point to, but could be a worth while read for someone that has no
experience in Shell programming:

\URL{http://linuxcommand.org/lc3_learning_the_shell.php}

Certainly you are welcome to use other resources that may suite you
best. We will however summarize in table form a number of useful
commands that you may als find in a link to a RefCard.

\URL{http://www.cheat-sheets.org/\#Linux}


\section{File commands}\label{file-commands}

Find included a number of commands related to file manipulation.

\begin{tabular}{ll}
Command & Description \\
\hline
ls & Directory listing\\
ls -lisa & list details \\
cd \emph{dirname} & Change directory to \emph{dirname} \\
mkdir \emph{dirname} & create the directory \\
pwd & print working directory \\
rm \emph{file} & remove the file \\
cp \emph{a} \emph{b} & copy file \emph{a} to \emph{b} \\
mv \emph{a} \emph{b} & move/rename file \emph{a} to \emph{b}\\
cat \emph{a} & print content of file\emph{a}\\
less \emph{a} & print paged content of file \emph{a}\\
head -5 \emph{a} & Display first 5 lines of file \emph{a}\\
tail -5 \emph{a} & Display last 5 lines of file \emph{a}
\end{tabular}

\section{Search commands}\label{search-commands}

Find included a number of commands related to seraching.

\begin{tabular}{ll}
Command & Description \\
\hline
fgrep &  TBD \\
grep -R ``xyz'' . & TBD \\
find . -name ``*.py'' &  TBD \\
\end{tabular}

\section{Help}\label{help}

Find included a number of commands related to manual pages.

\begin{tabular}{ll}
Command & Description \\
\hline
man \emph{command} & manual page for the \emph{command} \\
\end{tabular}

\section{Keyboard Shortcuts}\label{keyboard-shortcuts}

These shortcuts will come in handy. Note that many overlap with emacs
short cuts.

\begin{tabular}{ll}
Keys     & Description  \\
\hline
Up Arrow & Show the previous command\\
Ctrl + z & Stops the current command  \\
         & Resume with fg in the foreground \\
         & Resume with bg in the background \\
Ctrl + c & Halts the current command\\
Ctrl + l & Clear the screen\\
Ctrl + a & Return to the start of the command you're typing\\
Ctrl + e & Go to the end of the command you're typing\\
Ctrl + k & Cut everything after the cursor to a special clipboard\\
Ctrl + y & Paste from the special clipboard \\
Ctrl + d & Log out of current session, similar to exit \\
\end{tabular}

\section{.bashrc and .bash\_profile}\label{bashrc-and-.bash_profile}

Usage of a particular command and all the attributes associated with it,
use `man' command. Avoid using `rm -r' command to delete files
recursively. A good way to avoid accidental deletion is to include the
following in your .bash\_profile file:

\begin{verbatim}
alias e=open_emacs
alias rm='rm -i'
alias mv='mv -i' 
alias h='history'
\end{verbatim}

More Information

\url{https://cloudmesh.github.io/classes/lesson/linux/refcards.html}

\section{Exercise}\label{exercise}

\begin{description}
\item[Linux.1:]
Familiarize yourself with the commands
\item[Linux.2:]
Find more commands that you find useful and add them to this page.
\item[Linux.3:]
Use the sort command to sort all lines of a file while removing
duplicates.
\end{description}
