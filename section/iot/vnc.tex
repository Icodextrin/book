

\chapter{VNC}\label{vnc}

\FILENAME

\textbf{Note:} \emph{If you like to connect to your Raspberry from your
laptop, we recommend to use VNC. If you rather like to connect a monitor
and keyboard as well as a mouse to the Raspberry, you can skip the steps
with the VNC update.}

\section{Setting up VNC}\label{setting-up-vnc}

We had some issues with the installed version of VNC that is customized
for connecting a Laptop via the ethernet cable to the PI. However as we
connect wirelessly, our setup is slightly diffrent. The easiset way that
we found is to update the Raspbian OS as follows. In a terminal type

\begin{verbatim}
sudo apt-get update
sudo apt-get install realvnc-vnc-server 
sudo apt-get install realvnc-vnc-viewer
\end{verbatim}

Next you enable the VNC server in the configuration panel via the
Rasbian GUI by selecting

\begin{verbatim}
 Menu > 
    Preferences > 
       Raspberry Pi Configuration > 
          Interfaces.
\end{verbatim}

Here you toggle the VNC service to enabled. As we are already at it in
our setup we enabled all other services, especially those that deal with
Grove sensor related bins and wires.

Next reboot and double check if the settings are preseved after the
reboot

\subsection{Install VNC on OSX}\label{install-vnc-on-osx}

To install a vnc server of your liking on your Mac. You find one at

\begin{itemize}

\item
  \href{http://www.realvnc.com/download/vnc/latest/\%5D}{http://www.realvnc.com/download/vnc/latest/}
\end{itemize}

Be sure to download the version of the VNC Viewer for the computer you
are going to use to virtually control the Pi (there is a version listed
for Raspberry Pi-- don't download this one. For us this is the Mac
version.)

\subsection{Run VNC Viewer on OSX}\label{run-vnc-viewer-on-osx}

Once you have downloaded the VNC viewer installed it you can open the
program. Next you can start vnc viewer and enter the ip address of your
raspberry. Make sure you are on the same network. You can find the
address by using ifconfig.
