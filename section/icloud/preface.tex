\part{Preface}

\chapter{Overview}

\FILENAME

\section{Registrar}

Intelligent Systems Engineering
E516 Engineering Cloud Computing

Credits: 3

Prerequisite(s): Experience with Windows or Linux using Java and scripts.

This course covers basic concepts on programming models and tools of
cloud computing to support data intensive science
applications. Students will get to know the latest research topics of
cloud platforms, parallel algorithms, storage and high level language
for proficiency with a complex ecosystem of tools that span many
disciplines.  


\section{Cloud Computing}

The course covers all aspects of the cloud architecture stack, from
Software as a Service (large-scale biology and graphics applications),
Platform as a Service (MapReduce (Hadoop), Iterative MapReduce
(Twister) and NoSQL (HBase)), to Infrastructure as a Service
(low-level virtualization technologies).



\section{Class Summary}

In this course you will learn basic concepts in Cloud Computing,
including how to write your own software using key cloud programming
models and tools to support data mining and data analysis
applications.

\section{Course Organization}

This course is powered by Latex and ReStructuredText but is conducted
in a more structured fashion with a mix of recorded lectures,
programming labs and forum discussions. Each week we will post on
Canvas the instructions as to work to be done. Note all grading and
forum discussion will use Canvas and Piazza.

\section{What Should I Know?}

Cloud Computing online is a programming intensive course. It has
similar requirements to the CS graduate level residential
version. Students are expected to have weekly (or biweekly)
programming homework. General programming experience with Windows or
Linux using Java (2-3 years) and scripts is required. A background in
parallel and cluster computing is a plus, although not necessary.

\section{What Will I Learn?}

At the end of this course, you will have learned key concepts in cloud
computing and enough programming to be able to solve data analysis
problems on your own.

\section{Class Projects}

The class has several projects that will allow students to get
firsthand experience with the technologies taught here. Projects are
performed on VirtualBox Appliances or academic clouds like FutureSystems.

\section{Course Schedule}



\begin{itemize}
  \item Welcome Survey
  \item Unit 1 - Cloud Computing Fundamentals
  \item Unit 2 - How to Run VMs (IaaS)
  \item Unit 3 - How to Run MapReduce (PaaS)
  \item Unit 4 - How to Run Iterative MapReduce (PaaS)
  \item Mid-course assessment
  \item Unit 5 - How to Store Data (NoSQL)
  \item Unit 6 - Internet of Things
  \item Unit 7 - How to Build a Search Engine (SaaS)
  \item Post-course assessment
  \item Post-course Survey
  \item Test-course assessment
\end{itemize}

\section{Class Progress Distribution}

The course progress is the percentage of mandatory items completed for
the course.

\begin{itemize}
  \item Participation (10\%)
  \item Exams (50\%) - Midterm (20\%), Final (30\%)
  \item Written Assignments (10\%)
  \item Excersizes (30\%) - 8 Excersizes: 
    \begin{itemize}
      \item Hadoop Statistics (5)
      \item Hadoop PageRank (10)
      \item Hadoop Blast (10)
      \item HBase WordCount (5)
      \item Building an Inverted Index (10)
      \item Build a Search Engine (20)
      \item Harp PageRank (20)
      \item Harp Mini-batch K-Means (20)
    \end{itemize}
\end{itemize}

\section{Lesson Order}

Lessons in this course will be uploaded on a weekly basis. The
difficulty of the topics covered in these online lectures is
demonstrated by way of a color coding system. {\em Green} is basic,
{\em Yellow} is intermediate, and {\em Red} is advanced. We will build
up to more difficult concepts as the class progresses.

\section{Course Projects}

Take part in online assignments that will teach you the course
material in hands-on examples. You will apply actual code designed to
calculate website page ranking, word count, and even build your own
search engine. More than a dry tutorial, this is your chance to find
out how some of the most widely used applications on the Internet work
on a fundamental level.


