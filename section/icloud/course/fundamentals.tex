\chapter{Cloud Computing Fundamentals}

  6 Video lectures {48 minutes 20 seconds}

\section{Course Info}

Prof. Judy Qiu covers basic information about Unit 1, including lecture
schedule and homework submissions. Key cloud computing topics are
highlighted. A selection of recommended and required textbooks is given,
followed by an overview of the course structure.

\video{CLoud}{13:13}{Course Information}{https://www.youtube.com/watch?v=Kde5YVUwDTQ}

  Slide:
  \href{https://drive.google.com/open?id=0B88HKpainTSfYjU4QzdDSms0Nk0}{PDF}

\section{Introduction}

Changes in computing technology for the past five decades are discussed.
The rise of Big Data is shown in terms of its growth and significance. A
prediction is made that the paradigm which has held `til now of
individual researchers with personal computers will give way to
communities of researchers organizing through clouds. A more in-depth
look at Unit 1 follows, focusing on the chapters from Distributed and
Cloud Computing: From Parallel Processing to the Internet of Things.

\video{Cloud}  {8:31}{TBD}{https://www.youtube.com/watch?v=5lKj8_nqj9k}


\section{Data Center Model}

A look at what truly defines a `cloud'. Advantages like scalability and
cost-effectiveness have promulgated commercial cloud offerings such as
Amazon EC2. Cloud architecture as divided into three layers:
Infrastructure as a Service, Platform as a Service, and Software as a
Service. AzureBlast is used as an example of how to utilize the cloud
setup. Certain misconceptions about clouds are then presented for
further discussion.

\video{Cloud}  {8:08}{TBD}{https://www.youtube.com/watch?v=6Hq_LuLB-RU}


\section{Data Intensive Sciences}

Some time is spent analyzing the current age of vast data growth, where
business, science, and consumer activity has seen an explosion of stored
data measured in exobytes. In response to this, the way we conduct
scientific research has also undergone an upgrade. However, the average
scientist would rather focus on their own research rather than spend
time trying to learn different methods of cloud and supercomputing.

\video{Cloud}  {2:44}{TBD}{https://www.youtube.com/watch?v=Ptoj3BME_z4}


\section{IaaS, PaaS and SaaS}

Definitions and examples are given for Infrastructure as a Service,
Platform as a Service, and Software as a Service. A chart is shown
illustrating how use of clouds trades cost and control for efficiency.
Following this is an exploration of the MapReduce program, and an
illustration of its concepts through WordCount. Finally, four distinct
approaches to MapReduce are compared.

\video{Cloud}  {10:17}{TBD}{https://www.youtube.com/watch?v=_irz3v1gT-A}


\section{Challenges}

The demands of Big Data calls for advances in areas like distributed
computing, systems management, internet technology, and hardware. Clouds
have become more prominent in the last few decades, so much so that many
people today take advantage of them without even knowing it. Of course,
this has also led to increased concerns about security, price, tech
support, etc. In spite of this, clouds still have clear advantages over
traditional computing models. A quiz is offered at the end asking
students to correctly place software in a hierarchy of computing.

\video{Cloud}  {5:27}{TBD}{https://www.youtube.com/watch?v=VpDRGcBe4s8}

