Project 2: Hadoop PageRank

Cloud Computing

Spring 2017

Professor Judy Qiu

Goal

This assignment provides an illustration of PageRank algorithms and Hadoop. You will then blend these
applications by implementing a parallel version of PageRank using the programming interfaces of the Hadoop
MapReduce framework.

Deliverables

You are required to turn in the following items in a zip file (username HadoopPageRank.zip) in this assignment:
• The source code of Hadoop PageRank you implemented.
• Technical report (username HadoopPageRank report.docx) that contains:
• The description of the main steps and data flow in your program.
• The output file (username HadoopPageRank output.txt) which contains the first 10 urls along with
their ranks.
Evaluation
The point total for this project is 10, where the distribution is as follows:
• Completeness of your code and output (7 points)
• Correctness of written report (3 points)
1 What is PageRank?
The web search engine is a typical distributed system on the Internet. It is designed to search for information
on the World Wide Web. The search results are generally presented in a list of results and are often called
hits. PageRank is a well-known web graph ranking algorithm that helps Internet users sort hits by their
importance.
PageRank calculates a numerical value for each element of a hyperlinked set of webpages, which reflects
the probability that a random surfer will access that page. The process of PageRank can be understood as a
Markov Chain which requires iterative calculations to converge. An iteration of PageRank calculates the new
access probability for each webpage based on values calculated in the previous iteration. The process will
repeat until the number of current iterations is bigger than predefined maximum iterations, or the Euclidian
distance between rank values in two subsequent iterations is less than a predefined threshold that controls
the accuracy of the output results.
1
Figure 1: Mathematical PageRank for a simple network in Wikipedia
Figure 1 shows a web graph consisting of 11 vertices A, B, C, D, E, F, G1, G2, G3, G4, G5. Each vertex
refers to a unique webpage, and the directed edge means there is one link from the source webpage to the
target webpage. The percentage on each vertex represents the rank value of each webpage.
Notes
You can implement a sequential PageRank that can run on desktops or laptops. But when processing
larger input data, like web graphs containing more than a million webpages, you need to run the PageRank
application in parallel so that it can aggregate the computing power of multiple compute nodes. Currently, in
both industry and academia, the study of large-scale web or social graphs has become increasingly popular.
In one published paper, the job execution engines that claim to support large-scale PageRank include: MPI,
Hadoop, Dryad, Twister, Pregel.
Formula
Equation 1 is the formula to calculate the rank value for each webpage. We will learn this formula by
applying it to the case in Figure 1. There are 11 webpages in Figure 1, which include: A, B, C, D, E, F,
G1, G2, G3, G4, G5. Assuming the probability distribution for a web surfer accessing all these 11 pages in
current iteration is {PR(A), PR(B), PR(C), ... PR(G5)}, then the probability for the surfer to access Page
B in the next iteration is:

\[ PR(B) = P R(D)/2 + P R(E)/3 + P R(F)/2 + P R(C) + P R(G1)/2 + P R(G2)/2 + P R(G3)/2\]

In a general case, the PageRank value for any page u can be expressed as:
P R(u) = X
v∈Set
P R(V )
L(v)
(1)
The vertices seen in the right of the formula contain all the webpages that point to target webpage ’u’.
The L(v) refers to the out degree of each webpage in the vertices set. The initial rank values of each webpage,
like PR’(u), can be any double value. After several iteration calculations, the rank values converge to the
stationary distribution regardless of what their initial values are.
2
Damping factor
The PageRank theory holds that even an imaginary surfer who is randomly clicking on links will eventually
stop clicking. The probability, at any step, that the person will continue is a damping factor d. Various
studies have tested different damping factors, but it is generally assumed that the damping factor will be
around 0.85. The formula considering damping factor is shown in Equation 2. N refers to the total number
of unique urls.
P R(u) = 1 − d
N
+ d ∗
X
v∈Set
P R(V )
L(v)
(2)
Hadoop PageRank DataFlow
In this project, we have provided a sketch code which contains three MapReduce jobs for you to implement:
• CreateGraph (done): add one column, ’initial pagerank value’, to the input pagerank adjacency matrix
(AM). Then pass it to the PageRank program to calculate the pagerank values.
• PageRank (your implementation): take the transformed AM matrix and calculate pagerank values for
all pages.
• Cleanup Results: remove the targetUrls column and output (sourceUrl, pagerank value) as the
final result.
The detail dataflow can be seen in Figure 2. Part 1 and Part 3 are given as full solutions in this pipeline;
you will implement the 2nd part of the PageRank program.
Figure 2: Hadoop PageRank dataflow
3
Normally for any Hadoop MapReduce program, input data is uploaded and stored in the Hadoop Distributed
File System (HDFS) before computation in order to generate (key, value) pairs to the mapper.
Initially, the PageRank input data is stored in the format of adjacency matrix as a file(s) in the local file
system. Then it will be uploaded to the HDFS and distributed across the compute nodes. Hadoop framework
reads the application records from HDFS with the InputFormat interface and generates (key, value) pair
input streams. Each Map function produces zero or more intermediate (key, value) pairs by consuming
one input (key, value) pair. For this PageRank program, the map function applies the calculation P R(v)
L(V )
to
each (key, value) pair, where the key is the unique id or name of the webpage and the value contains the
current rank value of the webpage and its out link information. Map tasks then generate intermediate (key,
value) pairs, whose value is the partial rank value of every webpage. Each reduce task aggregates all the
partial values of specific webpages by applying the provided Equation 2. The aggregated global rank values
are written back to HDFS, which in turn is used as input in the next set of iterations, if any. ”Hadoop -
PageRank” in Figure 2 shows an example for the PageRank data processing.
Code for Hadoop PageRank
You need to complete two files in the provided pacakge inside ”indiana/cgl/hadoop/pagerank/”: PageRankMap.java
and PageRankReduce.java. Code snapshots are shown below.
1 /∗ f i l e : PageRankMap . j a v a ∗/
2 package i n di a n a . c g l . hadoop . pagerank ;
3
4 import j a v a . i o . B u f f e r e dW ri t e r ;
5 /∗ s e e d e t a i l i n s o u r c e code ∗/
6
7 p u bli c c l a s s PageRankMap e x t e n d s Mapper<LongWritable , Text , LongWritable , Text> {
8
9 // each map t a s k h a n dl e s one l i n e wi t hi n an a d j a c e n c y ma trix f i l e
10 // key : f i l e o f f s e t
11 // v al u e : <s o u r c e U rl PageRank#t a r g e t U rl s >
12 p u bli c v oid map( LongWritable key , Text value , Context c o n t e x t )
13 throws IOException , I n t e r r u p t e d E x c e p ti o n {
14
15 i n t numUrls = c o n t e x t . g e t C o n fi g u r a ti o n ( ) . g e t I n t ( ”numUrls” , 1 ) ;
16 S t ri n g l i n e = v al u e . t o S t ri n g ( ) ;
17 S t r i n g B u f f e r sb = new S t r i n g B u f f e r ( ) ;
18 // i n s t a n c e an o b j e c t t h a t r e c o r d s the i n f o rm a ti o n f o r one webpage
19 RankRecord r r d = new RankRecord ( l i n e ) ;
20 i n t s o u r c eU rl , t a r g e t U rl ;
21 // d ouble rankV alueO fS rcU rl ;
22
23 i f ( r r d . t a r g e t U r l s L i s t . s i z e ( ) <=0){
24 // t h e r e i s no out d e g r e e f o r t h i s webpage ;
25 // s c a t t e r i t s rank v al u e t o a l l o t h e r u r l s
26 d ouble rankValuePe rU rl = r r d . rankValue / ( d ouble ) numUrls ;
27 f o r ( i n t i =0; i<numUrls ; i++){
28 c o n t e x t . w ri t e (new LongWritable ( i ) , new Text ( S t ri n g . v alueO f ( rankValuePe rU rl ) ) ) ;
29 }
30 } e l s e {
31 /∗Write your code h e r e ∗/
32 } // f o r
33 c o n t e x t . w ri t e (new LongWritable ( r r d . s o u r c e U rl ) , new Text ( sb . t o S t ri n g ( ) ) ) ;
34 }//map
35 }
1 /∗ f i l e : PageRankReducer . j a v a ∗/
2 package i n di a n a . c g l . hadoop . pagerank ;
3
4 import j a v a . i o . B u f f e r e dW ri t e r ;
5 /∗ s e e d e t a i l i n s o u r c e code ∗/
6
7 p u bli c c l a s s PageRankReduce e x t e n d s Reducer<LongWritable , Text , LongWritable , Text>{
4
8 p u bli c v oid reduce ( LongWritable key , I t e r a b l e <Text> v al u e s ,
9 Context c o n t e x t ) throws IOException , I n t e r r u p t e d E x c e p ti o n {
10 d ouble sumOfRankValues = 0 . 0 ;
11 S t ri n g t a r g e t U r l s L i s t = ”” ;
12
13 i n t s o u r c e U rl = ( i n t ) key . g e t ( ) ;
14 i n t numUrls = c o n t e x t . g e t C o n fi g u r a ti o n ( ) . g e t I n t ( ”numUrls” , 1 ) ;
15
16 // hi n t : each t u pl e may i n cl u d e rank v al u e t u pl e o r l i n k r e l a t i o n t u pl e
17 f o r ( Text v al u e : v al u e s ) {
18 S t ri n g [ ] s t rA r r a y = v al u e . t o S t ri n g ( ) . s p l i t ( ”#” ) ;
19 /∗Write your code h e r e ∗/
20 }
21 // c a l c u l a t e u si n g the f o rmul a
22 sumOfRankValues = 0. 8 5 ∗ sumOfRankValues + 0. 1 5 ∗( 1. 0 ) / ( d ouble ) numUrls ;
23 c o n t e x t . w ri t e ( key , new Text ( sumOfRankValues+t a r g e t U r l s L i s t ) ) ;
24 }
25 }
Edit
The sketch code is stored within the provided VirtualBox image. Use Eclipse or linux text editor vi/vim to
add your code.
1 $ cd / r o o t /MoocHomeworks/HadoopPageRank/
2 $ vim s r c / i n di a n a / c g l / hadoop / pagerank /PageRankMap . j a v a
3 $ vim s r c / i n di a n a / c g l / hadoop / pagerank /PageRankReduce . j a v a
Compile and run your code
Use the one-click script compileAndExecHadoopPageRank.sh provided below. Standard error messages such
as compile errors, execution errors, etc. will be redirected on the screen. Debug them based on the returned
messages.
1 $ cd / r o o t /MoocHomeworks/HadoopPageRank/
2 # u s a ge : . / compileAndExecHadoopPageRank . sh [ PageRank Input F i l e ] [ Number o f U rl s ] [ Number Of
I t e r a t i o n s ]
3 $ . / compileAndExecHadoopPageRank . sh PageRankDataGenerator / pagerank5000g50 . i n p u t . 0 5000 1
View the result
The result is generated as /root/hbaseMoocAntProject/output/project2.txt.
1 $ cd / r o o t /MoocHomeworks/HadoopPageRank/
2 $ c a t output /∗
5