\FILENAME

\section{OUTDATED: Technology Training - Python and FutureSystems}

\TODO{warning:: this section will be updated and replaced.}

This section is meant to give an overview of the python tools needed for
doing for this course.

These are really powerful tools which every data scientist who wishes to
use python must know.

NumPy - It is popular library on top of which many other libraries (like
pandas, scipy) are built. It provides a way a vectorizing data. This
helps to organize in a more intuitive fashion and also helps us use the
various matrix operations which are popularly used by the machine
learning community. Matplotlib: This a data visualization package. It
allows you to create graphs charts and other such diagrams. It supports
Images in JPEG, GIF, TIFF format. SciPy: SciPy is a library built above
numpy and has a number of off the shelf algorithms / operations
implemented. These include algorithms from calculus(like integration),
statistics, linear algebra, image-processing, signal processing, machine
learning, etc.

\subsection{Python for Big Data: NumPy, SciPy,
MatPlotlib}\label{python-for-big-data-numpy-scipy-matplotlib}

This section is meant to give an overview of the python tools needed for
doing for this course. These are really powerful tools which every data
scientist who wishes to use python must know.

\subsubsection{Introduction}\label{introduction}

This section is meant to give an overview of the python tools needed for
doing for this course. These are really powerful tools which every data
scientist who wishes to use python must know. This section covers NumPy,
MatPlotLib, and Scipy.

\subsubsection{Pycharm}\label{pycharm}

is an Integrated Development Environment (IDE) used for programming in
Python. It provides code analysis, a graphical debugger, an integrated
unit tester, integration with git.

\video{Python}{8:56}{Pycharm}{https://youtu.be/X8ZpbZweJcw}

\subsubsection{Python in 45 minutes}\label{python-in-45-minutes}

Here is an introductory video about the Python programming language that
we found on the internet. Naturally there are many alternatives to this
video, but the video is probably a good start. It also uses PyCharm
which we recommend.

\video{Python}{??:??}{PyCharm}{https://www.youtube.com/watch?v=N4mEzFDjqtA}

How much you want to understand of python is actually a bit up to your,
while its goot to know classes and inheritance, you may be able for this
class to get away without using it. However, we do recommend that you
learn it.

\subsubsection{Numpy 1}\label{numpy-1}

NumPy - It is popular library on top of which many other libraries (like
pandas, scipy) are built. It provides a way a vectorizing data. This
helps to organize in a more intuitive fashion and also helps us use the
various matrix operations which are popularly used by the machine
learning community.

\video{Python}{11:53}{Numpy 1}{http://youtu.be/mN_JpGO9Y6s}

\video{Python}{11:26}{Numpy 2}{http://youtu.be/7QfW7AT7UNU}

\video{Python}{11:51}{Numpy 3}{http://youtu.be/Ccb67Q5gpsk}

\subsubsection{Matplotlib}\label{matplotlib}

Matplotlib is a data visualization package. It allows you to create
graphs charts and other such diagrams. It supports Images in JPEG, GIF,
TIFF format.

\video{Python}{11:21}{Matplotlib 1}{http://youtu.be/3UOvB5OmtYE}

\video{Python}{8:32}{Matplotlib 2}{http://youtu.be/9ONSnsN4hcg}

\subsubsection{Scipy 1}\label{scipy-1}

SciPy is a library built above numpy and has a number of off the shelf
algorithms / operations implemented. These include algorithms from
calculus(like integration), statistics, linear algebra,
image-processing, signal processing, machine learning, etc.

\video{Python}{6:57}{Scipy 1}{http://youtu.be/lpC6Mn-09jY}

\video{Python}{8:52}{Scipy 2}{http://youtu.be/-XKBz7qCUqw}

\subsection{OUTDATED: Using FutureSystems (Please do not do
yet)}\label{outdated-using-futuresystems-please-do-not-do-yet}

This section is meant to give an overview of the FutureSystems and how
to use for the Big Data Course. In addition to this creating
FutureSystems Account, Uploading OpenId and SSH Key and how to
instantiate and log into Virtual Machine and accessing Ipython are
covered. In the end we discuss about running Python and Java on Virtual
Machine.

\subsubsection{FutureSystems Overview}\label{futuresystems-overview}

In this video we introduce FutureSystems in terms of its services and
features.

FirstProgram.java: 
\url{http://openedx.scholargrid.org:18010/c4x/SoIC/INFO-I-523/asset/FirstProgram.java}

\video{Python}{12:12}{FutureGid}{http://youtu.be/RibpNSyd4qg}

\subsubsection{Creating Portal Account}\label{creating-portal-account}

This lesson explains how to create a portal account, which is the first
step in gaining access to FutureSystems.

See Lesson 4 and 7 for SSH key generation on Linux, OSX or Windows.

\video{Python}{11:50}{FutureGrid Introduction}{http://youtu.be/X6zeVEALzTk}

\subsubsection{OUTDATED: Upload an
OpenId}\label{outdated-upload-an-openid}

This lesson explains how to upload and use OpenID to easily log into the
FutureSystems portal.

\video{Python}{1:34}{Upload an OpenID}{http://youtu.be/rZzpCYWDEpI}

\subsubsection{SSH Key Generation using ssh-keygen
command}\label{ssh-key-generation-using-ssh-keygen-command}

SSH keys are used to identify user accounts in most systems including
FutureSystems. This lesson walks you through generating an SSH key via
ssh-keygen command line tool.

\video{Python}{4:06}{ssh-key gen}{http://youtu.be/pQb2VV1zNIc}

\subsubsection{Shell Access via SSH}\label{shell-access-via-ssh}

This lesson explains how to get access FutureSystems resources vis SSH
terminal with your registered SSH key.

\video{Python}{2:34}{Shell Access via SSH}{http://youtu.be/aJDXfvOrzRE}

\subsubsection{Advanced SSH}\label{advanced-ssh}

This lesson shows you how to write SSH `config' file in advanced
settings.

\video{Python}{2:47}{Advanced SSH}{http://youtu.be/eYanElmtqMo}

\subsubsection{SSH Key Generation via putty (Windows user
only)}\label{ssh-key-generation-via-putty-windows-user-only}

This lesson is for Windows users.

You will learn how to create an SSH key using PuTTYgen, add the public
key to you FutureSystems portal, and then login using the PuTTY SSH
client.

\video{Python}{3:51}{Windows users}{http://youtu.be/irmVJKwWQCU}

\subsubsection{Using FS - Creating VM using Cloudmesh and running
IPython}\label{using-fs---creating-vm-using-cloudmesh-and-running-ipython}

This lesson explains how to log into FutureSystems and our customized
shell and menu options that will simplify management of the VMs for this
upcoming lessons.

Instruction is at:
\url{http://cloudmesh.github.io/introduction_to_cloud_computing/class/cm-mooc/cm-mooc.html}

\video{Python}{19:28}{Using FS - Creating VM using Cloudmesh and running IPython}{http://youtu.be/nbZbJxheLwc}
