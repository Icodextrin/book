\FILENAME

\section{Overview of Data Science}\label{overview-of-data-science}

\emph{What is Big Data, Data Analytics and X-Informatics?}

The course introduction starts with X-Informatics and its rallying cry.
The growing number of jobs in data science is highlighted. The first
unit offers a look at the phenomenon described as the Data Deluge
starting with its broad features. Data science and the famous DIKW (Data
to Information to Knowledge to Wisdom) pipeline are covered. Then more
detail is given on the flood of data from Internet and Industry
applications with eBay and General Electric discussed in most detail.

In the next unit, we continue the discussion of the data deluge with a
focus on scientific research. He takes a first peek at data from the
Large Hadron Collider considered later as physics Informatics and gives
some biology examples. He discusses the implication of data for the
scientific method which is changing with the data-intensive methodology
joining observation, theory and simulation as basic methods. Two broad
classes of data are the long tail of sciences: many users with
individually modest data adding up to a lot; and a myriad of Internet
connected devices -- the Internet of
Things.

We give an initial technical overview of cloud computing as pioneered by
companies like Amazon, Google and Microsoft with new centers holding up
to a million servers. The benefits of Clouds in terms of power
consumption and the environment are also touched upon, followed by a
list of the most critical features of Cloud computing with a comparison
to supercomputing. Features of the data deluge are discussed with a
salutary example where more data did better than more thought. Then
comes Data science and one part of it \textasciitilde{}\textasciitilde{}
data analytics \textasciitilde{}\textasciitilde{} the large algorithms
that crunch the big data to give big wisdom. There are many ways to
describe data science and several are discussed to give a good composite
picture of this emerging field.

\subsection{Data Science generics and Commercial Data
Deluge}\label{data-science-generics-and-commercial-data-deluge}

We start with X-Informatics and its rallying cry. The growing number of
jobs in data science is highlighted. This unit offers a look at the
phenomenon described as the Data Deluge starting with its broad
features. Then he discusses data science and the famous DIKW (Data to
Information to Knowledge to Wisdom) pipeline. Then more detail is given
on the flood of data from Internet and Industry applications with eBay
and General Electric discussed in most detail.


\slides{Overview}{45}{TBD}{https://drive.google.com/open?id=0B88HKpainTSfenJ4dEZQOUxZSmM}


\subsubsection{What is X-Informatics and its
Motto}\label{what-is-x-informatics-and-its-motto}

This discusses trends that are driven by and accompany Big data. We give
some key terms including data, information, knowledge, wisdom, data
analytics and data science. WE introduce the motto of the course: Use
Clouds running Data Analytics Collaboratively processing Big Data to
solve problems in X-Informatics. We list many values of X you can
defined in various activities across the world.


\video{Overview}{9:49}{TBD}{https://www.youtube.com/watch?v=8T0OtdR9Bp4}




\subsubsection{Jobs}\label{jobs}

Big data is especially important as there are some many related jobs. We
illustrate this for both cloud computing and data science from reports
by Microsoft and the McKinsey institute respectively. We show a plot
from LinkedIn showing rapid increase in the number of data science and
analytics jobs as a function of time.


\video{Overview}{2:58}{TBD}{http://youtu.be/pRlfEigUJAc}

\subsubsection{Data Deluge: General
Structure}\label{data-deluge-general-structure}

We look at some broad features of the data deluge starting with the size
of data in various areas especially in science research. We give
examples from real world of the importance of big data and illustrate
how it is integrated into an enterprise IT architecture. We give some
views as to what characterizes Big data and why data science is a
science that is needed to interpret all the data.


\video{Overview}{13:04}{TBD}{http://youtu.be/mPJ9twAFRQU}

\subsubsection{Data Science: Process}\label{data-science-process}

We stress the DIKW pipeline: Data becomes information that becomes
knowledge and then wisdom, policy and decisions. This pipeline is
illustrated with Google maps and we show how complex the ecosystem of
data, transformations (filters) and its derived forms is.


\video{Overview}{4:27}{TBD}{http://youtu.be/ydH34L-z0Rk}

\subsubsection{Data Deluge: Internet}\label{data-deluge-internet}

We give examples of Big data from the Internet with Tweets, uploaded
photos and an illustration of the vitality and size of many commodity
applications.


\video{Overview}{3:42}{TBD}{http://youtu.be/rtuq5y2Bx2g}

\subsubsection{Data Deluge: Business}\label{data-deluge-business}

We give examples including the Big data that enables wind farms, city
transportation, telephone operations, machines with health monitors, the
banking, manufacturing and retail industries both online and offline in
shopping malls. We give examples from ebay showing how analytics
allowing them to refine and improve the customer experiences.


\video{Overview}{6:00}{TBD}{http://youtu.be/PJz38t6yn_s}

\video{Overview}{7:34}{TBD}{http://youtu.be/fESm-2Vox9M}

\video{Overview}{9:37}{TBD}{http://youtu.be/fcvn-IxPO00}

\subsubsection{Resources}\label{resources}

\begin{itemize}
\item
  \url{http://www.microsoft.com/en-us/news/features/2012/mar12/03-05CloudComputingJobs.aspx}
\item
  \url{http://www.mckinsey.com/mgi/publications/big_data/index.asp}
\item
  Tom Davenport
  \url{http://fisheritcenter.haas.berkeley.edu/Big_Data/index.html}
\item
  Anjul Bhambhri
  \url{http://fisheritcenter.haas.berkeley.edu/Big_Data/index.html}
\item
  Jeff Hammerbacher
  \url{http://berkeleydatascience.files.wordpress.com/2012/01/20120117berkeley1.pdf}
\item
  \url{http://www.economist.com/node/15579717}
\item
  \url{http://cs.metrostate.edu/~sbd/slides/Sun.pdf}
\item
  \url{http://jess3.com/geosocial-universe-2/}
\item
  Bill Ruh \url{http://fisheritcenter.haas.berkeley.edu/Big\_Data/index.html}
\item
  \url{http://www.hsph.harvard.edu/ncb2011/files/ncb2011-z03-rodriguez.pptx}
\item
  Hugh Williams
  \url{http://fisheritcenter.haas.berkeley.edu/Big_Data/index.html}
\end{itemize}

\subsection{Data Deluge and Scientific Applications and
Methodology}\label{data-deluge-and-scientific-applications-and-methodology}

\subsubsection{Overview}\label{overview}

We continue the discussion of the data deluge with a focus on scientific
research. He takes a first peek at data from the Large Hadron Collider
considered later as physics Informatics and gives some biology examples.
He discusses the implication of data for the scientific method which is
changing with the data-intensive methodology joining observation, theory
and simulation as basic methods. We discuss the long tail of sciences;
many users with individually modest data adding up to a lot. The last
lesson emphasizes how everyday devices
\textasciitilde{}\textasciitilde{} the Internet of Things
\textasciitilde{}\textasciitilde{} are being used to create a wealth of
data.

\slides{Overview}{22}{TBD}{https://drive.google.com/open?id=0B88HKpainTSfZzhqZHVKbllZcTA}{PDF}


\subsubsection{Science \& Research}\label{science-research}

We look into more big data examples with a focus on science and
research. We give astronomy, genomics, radiology, particle physics and
discovery of Higgs particle (Covered in more detail in later lessons),
European Bioinformatics Institute and contrast to Facebook and Walmart.


\video{Overview}{11:27}{TBD}{http://youtu.be/u1h6bAkuWQ8}

\video{Overview}{11:49}{TBD}{http://youtu.be/_JfcUg2cheg}


\subsubsection{Implications for Scientific
Method}\label{implications-for-scientific-method}

We discuss the emergences of a new fourth methodology for scientific
research based on data driven inquiry. We contrast this with third
\textasciitilde{}\textasciitilde{} computation or simulation based
discovery - methodology which emerged itself some 25 years ago.


\video{Overview}{5:07}{TBD}{http://youtu.be/srEbOAmU_g8}

\subsubsection{Long Tail of Science}\label{long-tail-of-science}

There is big science such as particle physics where a single experiment
has 3000 people collaborate!.Then there are individual investigators who
don't generate a lot of data each but together they add up to Big data.


\video{Overview}{2:10}{TBD}{http://youtu.be/dwzEKEGYhqE}

\subsubsection{Internet of Things}\label{internet-of-things}

A final category of Big data comes from the Internet of Things where
lots of small devices \textasciitilde{}\textasciitilde{} smart phones,
web cams, video games collect and disseminate data and are controlled
and coordinated in the cloud.


\video{Overview}{5:45}{TBD}{http://youtu.be/K2anbyxX48w}


\subsubsection{Resources}\label{resources-1}

\begin{itemize}
\tightlist
\item
  \url{http://www.economist.com/node/15579717}
\item
  Geoffrey Fox and Dennis Gannon Using Clouds for Technical Computing To
  be published in Proceedings of HPC 2012 Conference at Cetraro, Italy
  June 28 2012
\item
  \url{http://grids.ucs.indiana.edu/ptliupages/publications/Clouds_Technical_Computing_FoxGannonv2.pdf}
\item
  \url{http://grids.ucs.indiana.edu/ptliupages/publications/Where\%20does\%20all\%20the\%20data\%20come\%20from\%20v7.pdf}
\item
  \url{http://www.genome.gov/sequencingcosts/}
\item
  \url{http://www.quantumdiaries.org/2012/09/07/why-particle-detectors-need-a-trigger/atlasmgg}
\item
  \url{http://salsahpc.indiana.edu/dlib/articles/00001935/}
\item
  \url{http://en.wikipedia.org/wiki/Simple_linear_regression}
\item
  \url{http://www.ebi.ac.uk/Information/Brochures/}
\item
  \url{http://www.wired.com/wired/issue/16-07}
\item
  \url{http://research.microsoft.com/en-us/collaboration/fourthparadigm/}
\item
  CSTI General Assembly 2012, Washington, D.C., USA Technical Activities
  Coordinating Committee (TACC) Meeting, Data Management, Cloud
  Computing and the Long Tail of Science October 2012 Dennis Gannon
  \url{https://sites.google.com/site/opensourceiotcloud/}
\end{itemize}

\subsection{Clouds and Big Data Processing; Data Science Process and
Analytics}\label{clouds-and-big-data-processing-data-science-process-and-analytics}

\subsubsection{Overview}\label{overview-1}

We give an initial technical overview of cloud computing as pioneered by
companies like Amazon, Google and Microsoft with new centers holding up
to a million servers. The benefits of Clouds in terms of power
consumption and the environment are also touched upon, followed by a
list of the most critical features of Cloud computing with a comparison
to supercomputing.

He discusses features of the data deluge with a salutary example where
more data did better than more thought. He introduces data science and
one part of it \textasciitilde{}\textasciitilde{} data analytics
\textasciitilde{}\textasciitilde{} the large algorithms that crunch the
big data to give big wisdom. There are many ways to describe data
science and several are discussed to give a good composite picture of
this emerging field.


  \slides{Overview}{35}{TBD}{https://drive.google.com/open?id=0B88HKpainTSfV1FwdktnbTl3T1k}{PDF}


\subsection{Clouds}\label{clouds}

We describe cloud data centers with their staggering size with up to a
million servers in a single data center and centers built modularly from
shipping containers full of racks. The benefits of Clouds in terms of
power consumption and the environment are also touched upon, followed by
a list of the most critical features of Cloud computing and a comparison
to supercomputing.


\video{Overview}{16:04}{TBD}{https://www.youtube.com/watch?v=trIFW-rucgM}{MP4}



\subsubsection{Features of Data Deluge I}\label{features-of-data-deluge-i}

Data, Information, intelligence algorithms, infrastructure, data
structure, semantics and knowledge are related. The semantic web and Big
data are compared. We give an example where ``More data usually beats
better algorithms''. We discuss examples of intelligent big data and
list 8 different types of data deluge


\video{Overview}{8:02}{TBD}{http://youtu.be/FMktnTQGyrw}

\video{Overview}{6:24}{TBD}{http://youtu.be/QNVZobXHiZw}


\subsubsection{Data Science Process}\label{data-science-process-1}

We describe and critique one view of the work of a data scientists. Then
we discuss and contrast 7 views of the process needed to speed data
through the DIKW pipeline.


\video{Overview}{11:28}{TBD}{http://youtu.be/lpQ-Q9ZidR4}


\subsubsection{Data Analytics}\label{data-analytics}

\slides{Overview}{30}{TBD}{http://archive2.cra.org/ccc/files/docs/nitrdsymposium/keyes.pdf}


We stress the importance of data analytics giving examples from several
fields. We note that better analytics is as important as better
computing and storage capability. In the second video we look at High
Performance Computing in Science and Engineering: the Tree and the
Fruit.


\video{Overview}{7:28}{TBD}{http://youtu.be/RPVojR8jrb8}

\video{Overview}{6:51}{TBD}{http://youtu.be/wOSgywqdJDY}


\subsubsection{Resources}\label{resources-2}

\begin{itemize}
\tightlist
\item
  CSTI General Assembly 2012, Washington, D.C., USA Technical Activities
  Coordinating Committee (TACC) Meeting, Data Management, Cloud
  Computing and the Long Tail of Science October 2012 Dennis Gannon
\item
  Dan Reed Roger Barga Dennis Gannon Rich
  Wolskihttp://research.microsoft.com/en-us/people/barga/sc09\_cloudcomp\_tutorial.pdf
\item
  \url{http://www.datacenterknowledge.com/archives/2011/05/10/uptime-institute-the-average-pue-is-1-8/}
\item
  \url{http://loosebolts.wordpress.com/2008/12/02/our-vision-for-generation-4-modular-data-centers-one-way-of-getting-it-just-right/}
\item
  \url{http://www.mediafire.com/file/zzqna34282frr2f/koomeydatacenterelectuse2011finalversion.pdf}
\item
  Bina Ramamurthy
  \url{http://www.cse.buffalo.edu/~bina/cse487/fall2011/}
\item
  Jeff Hammerbacher
  \url{http://berkeleydatascience.files.wordpress.com/2012/01/20120117berkeley1.pdf}
\item
  Jeff Hammerbacher
  \url{http://berkeleydatascience.files.wordpress.com/2012/01/20120119berkeley.pdf}
\item
  Anjul Bhambhri
  \url{http://fisheritcenter.haas.berkeley.edu/Big_Data/index.html}
\item
  \url{http://cs.metrostate.edu/~sbd/slides/Sun.pdf}
\item
  Hugh Williams
  \url{http://fisheritcenter.haas.berkeley.edu/Big_Data/index.html}
\item
  Tom Davenport
  \url{http://fisheritcenter.haas.berkeley.edu/Big_Data/index.html}
\item
  \url{http://www.mckinsey.com/mgi/publications/big_data/index.asp}
\item
  \url{http://cra.org/ccc/docs/nitrdsymposium/pdfs/keyes.pdf}
\end{itemize}
