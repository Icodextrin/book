

\section{Introduction}\label{introduction}

\FILENAME

\begin{IU}
  You may find that some videos may have a different lesson, section
  or unit numbers. Please ignore this as we have significantly
  restructured the material and. In case the content does not
  correspond to the title, please let us know.
\end{IU}

This Part of the Handbook includes broad overview for the motivation
on why we study Big Data Applications. The information has been
consolidated from the Web page hosted at \URL{https://cloudmesh.github.io/classes/}.

The overview covers it's content and structure. It presents an
introduction to general field of Big Data and Analytics. We are
especially analysing the many different application areas in which Big
Data can be applied. As Big Datais typically not just used in isolation
but is part of a larger Informatics issue for a particular field we also
use the term X-Informatics, where X defines a usecase or area of
specialization in which Big Data is applied to. As such we organize the
material around the the \emph{Rallying Cry}: Use Clouds running
Data Analytics Collaboratively processing Big Data to solve problems in
X-Informatics.

This part is set up as a number of lessons that are typically between
20 minutes to an hour. The lessons are either provided as written
documents or as video lectures. They are enhanced by an in person
meeting that takes place either in a lecture room for residential
students or as online meeting for online students.

The part covers a mix of applications (the X in X-Informatics) and
technologies needed to support the field electronically i.e. to process
the application data. The overview ends with a discussion of the 
content at highest level. The material starts with a motivation
summarizing clouds and data science, then units describing applications
in areas such as Physics, e-Commerce, Web Search and Text mining,
Health, Sensors and Remote Sensing). These are interspersed with
discussions of infrastructure (clouds) and data analytics (algorithms
like clustering and collaborative filtering used in applications). We 
use Python as primary programming language. We will be
introducing practical use of cloud resources so that you have the
oportunity to explore example analytics applications on smaller data
sets that you define.

We start with striking examples of the data deluge
with examples from research, business and the consumer. The growing
number of jobs in data science is highlighted. He describes industry
trend in both clouds and big data. Then the cloud computing model
developed at amazing speed by industry is introduced. The 4 paradigms of
scientific research are described with growing importance of data
oriented version.He covers 3 major X-informatics areas: Physics,
e-Commerce and Web Search followed by a broad discussion of cloud
applications. Parallel computing in general and particular features of
MapReduce are described.

We discuss the following topics which may be adapted as we see fit. 

Writing Track:

\begin{itemize}
\item  Writing a short review article
\item  Writing a porject or term report
\end{itemize}

Theory Track:

\begin{itemize}
\item  Motivation: Big Data and the Cloud; Centerpieces of the Future Economy
\item  Introduction: What is Big Data, Data Analytics
\item  Use Cases: Big Data Use Cases Survey

  \begin{itemize}
  \item    Use Case, Physics Discovery of Higgs Particle
  \item    Use Case: e-Commerce and Lifestyle with recommender systems
  \item    Use Case: Web Search and Text Mining and their technologies
  \item    Use Case: Sports
  \item    Use Case: Health
  \item    Use Case: Sensors
  \item    Use Case: Radar for Remote Sensing.
  \end{itemize}

\item Parallel Computing Overview and familiar examples
\item Cloud Technology for Big Data Applications \& Analytics
\end{itemize}

Practice Track:

\begin{itemize}
\item  Python for Big Data Applications and Analytics: NumPy, SciPy,
  MatPlotlib
\item  Using FutureGrid for Big Data Applications and Analytics 
\item  Using Chameleon Cloud for Big Data Applications and Analytics 
\item (optional) Using Plotviz Software for Displaying Point
  Distributions in 3D
\item  Recommender Systems - K-Nearest Neighbors, Clustering and heuristic
  methods
\item  PageRank
\item  Kmeans
\item  MapReduce
\item  Kmeans and MapReduce Parallelism
\end{itemize}

\subsection{Motivation}

We motivate the study of X-informatics by describing data science and
clouds. He starts with striking examples of the data deluge with
examples from research, business and the consumer. The growing number of
jobs in data science is highlighted. He describes industry trend in both
clouds and big data.

He introduces the cloud computing model developed at amazing speed by
industry. The 4 paradigms of scientific research are described with
growing importance of data oriented version. He covers 3 major
X-informatics areas: Physics, e-Commerce and Web Search followed by a
broad discussion of cloud applications. Parallel computing in general
and particular features of MapReduce are described. He comments on a
data science education and the benefits of using MOOC's.

\subsubsection{Emerging Technologies}\label{emerging-technologies}

This presents the overview of talk, some trends in computing and data
and jobs. Gartner's emerging technology hype cycle shows many areas of
Clouds and Big Data. We highlight 6 issues of importance: economic
imperative, computing model, research model, Opportunities in advancing
computing, Opportunities in X-Informatics, Data Science Education


\video{Introduction}{40:14}{Motivation}  {https://drive.google.com/file/d/0B1Of61fJF7WsV2RvMlFzSDNPZEU/view?usp=sharing}
  
\slides{Introduction}{30}  {Motivation}{https://drive.google.com/file/d/0B8936_ytjfjmOUZraHc4M1ptczA/view?usp=sharing}


\subsubsection{Data Deluge}\label{data-deluge}

We give some amazing statistics for total storage; uploaded video and
uploaded photos; the social media interactions every minute; aspects of
the business big data tidal wave; monitors of aircraft engines; the
science research data sizes from particle physics to astronomy and earth
science; genes sequenced; and finally the long tail of science. The next
slide emphasizes applications using algorithms on clouds. This leads to
the rallying cry ``Use Clouds running Data Analytics Collaboratively
processing Big Data to solve problems in X-Informatics educated in data
science'`with a catalog of the many values of X''Astronomy, Biology,
Biomedicine, Business, Chemistry, Climate, Crisis, Earth Science,
Energy, Environment, Finance, Health, Intelligence, Lifestyle,
Marketing, Medicine, Pathology, Policy, Radar, Security, Sensor, Social,
Sustainability, Wealth and Wellness''


\video{Introduction}{30:38}  {Data Deluge}{https://www.youtube.com/watch?v=7VHPXJv3DN4}


\slides{Introduction}{20}  {Data  Deluge}{https://drive.google.com/open?id=0B8936_ytjfjmUXY3anBaeU9lLVU}

\subsubsection{Jobs}\label{jobs}

Jobs abound in clouds and data science. There are documented shortages
in data science, computer science and the major tech companies advertise
for new talent.


\video{Introduction}{9:39}  {Jobs}{https://www.youtube.com/watch?v=KsjiQS8uXDA}


\slides{Introduction}{8}  {Jobs}{https://drive.google.com/open?id=0B8936_ytjfjmaG50YW9TeWdvUTg}


\subsubsection{Industrial Trends}\label{industrial-trends}

Trends include the growing importance of mobile devices and comparative
decrease in desktop access, the export of internet content, the change
in dominant client operating systems, use of social media, thriving
Chinese internet companies.


\video{Introduction}{19:25} 
  {Industrial Trends}{https://www.youtube.com/watch?v=32vD7uN7fqY}


\slides{Introduction}{16}
  {Industrial
  Trends}{https://drive.google.com/open?id=0B8936_ytjfjmWW1SdXgxWkRLYjg}



\video{Introduction}{16:54}   {Industrial Trends  II}{https://www.youtube.com/watch?v=O8fgXAQcnvw}

\slides{Introduction}{16}
  {Indusrial
  Trends II}{https://drive.google.com/open?id=0B8936_ytjfjmeEV2R19ORzhBQVE}



\video{Introduction}{30:13} 
  {Indusrial Trends
  III}{https://www.youtube.com/watch?v=kW38MG7ukzs}

\slides{Introduction}{21}
  {Industrial
  Trends III}{https://drive.google.com/open?id=0B8936_ytjfjmNDZKcE1MSU45ZG8}


\subsubsection{Digital Disruption of Old
Favorites}\label{digital-disruption-of-old-favorites}

Not everything goes up. The rise of the Internet has led to declines in
some traditional areas including Shopping malls and Postal Services.

\video{Introduction}{32:54} 
{Digital Distruption
and transformation}{https://www.youtube.com/watch?v=bw9yYXwe7Bs} 



\slides{Introduction}{28}
  {Digital
  Distruption and transformation}{https://drive.google.com/open?id=0B8936_ytjfjmdW5CYnBtME9FVTQ}


\subsubsection{Computing Model}\label{computing-model}

\emph{Industry adopted clouds which are attractive for data analytics}

Clouds and Big Data are transformational on a 2-5 year time scale.
Already Amazon AWS is a lucrative business with almost a \$4B revenue.
We describe the nature of cloud centers with economies of scale and
gives examples of importance of virtualization in server consolidation.
Then key characteristics of clouds are reviewed with expected high
growth in Infrastructure, Platform and Software as a Service.


\video{Introduction}{24:03} 
  {Computing Model I}{https://www.youtube.com/watch?v=oYKTCKFGTco}


\slides{Introduction}{14}
  {Computing
  Model I}{https://drive.google.com/open?id=0B8936_ytjfjmTU9nNml2bUlsUHM}



\video{Introduction}{28:18} 
  {Computing Model II}{https://www.youtube.com/watch?v=km_eXHq7m3o}


\slides{Introduction}{27}
  {Computing
  Model II}{https://drive.google.com/open?id=0B8936_ytjfjmNHhLYnI0X0YxdFE}

\subsubsection{Research Model}\label{research-model}

\emph{4th Paradigm; From Theory to Data driven science?}

We introduce the 4 paradigms of scientific research with the focus on
the new fourth data driven methodology.


\video{Introduction}{7:33}  {Research Model}{https://www.youtube.com/watch?v=xkeECe3mmjI}


\slides{Introduction}{4}  {Research  Model}{https//drive.google.com/open?id=0B8936_ytjfjma0pMbHJnek02dDA}


\subsubsection{Data Science Process}\label{data-science-process}

We introduce the DIKW data to information to knowledge to wisdom
paradigm. Data flows through cloud services transforming itself and
emerging as new information to input into other transformations.


\video{Introduction}{15:42} {Data Science Process}{https://www.youtube.com/watch?v=KstIH2aQ60Y}


\slides{Introduction}{10}
  {Data  Science Process}{https://drive.google.com/open?id=0B8936_ytjfjmVDVZa01keW0wQmc}


\subsubsection{Physics-Informatics}\label{physics-informatics}

\emph{Looking for Higgs Particle with Large Hadron Collider LHC}

We look at important particle physics example where the Large hadron
Collider has observed the Higgs Boson. He shows this discovery as a bump
in a histogram; something that so amazed him 50 years ago that he got a
PhD in this field. He left field partly due to the incredible size of
author lists on papers.


\video{Introduction}{13:27} 
  {Physics-informatics}{https://www.youtube.com/watch?v=2A7Z741FCHs}

\slides{Introduction}{6}
  {Physics-inforamtics}{https://drive.google.com/open?id=0B8936_ytjfjmc2J2TWgwWGRwaFk}


\subsubsection{Recommender Systems}\label{recommender-systems}

Many important applications involve matching users, web pages, jobs,
movies, books, events etc. These are all optimization problems with
recommender systems one important way of performing this optimization.
We go through the example of Netflix \textasciitilde{}\textasciitilde{}
everything is a recommendation and muses about the power of viewing all
sorts of things as items in a bag or more abstractly some space with
funny properties.


\video{Introduction}{12:21}
  {Recommender Systems  I}{https://www.youtube.com/watch?v=LXhng3fcG9o}



\slides{Introduction}{9}
  {Recommender  Systems I}{https://drive.google.com/open?id=0B8936_ytjfjmOXlVd2FsSUkwekk}



\video{Introduction}{9:44} 
  {Recommender Systems
  II}{https://www.youtube.com/watch?v=Y4S0jY0yfEE}

\slides{Introduction}{6}
  {Recommender
  Systems II}{https://drive.google.com/open?id=0B8936_ytjfjmMzM2M3RhMEJ4bjQ}


\subsubsection{Web Search and Information
Retrieval}\label{web-search-and-information-retrieval}

We look at Web Search and here we give an overview of the
data analytics for web search, Pagerank as a method of ranking web pages
returned and uses material from Yahoo on the subtle algorithms for
dynamic personalized choice of material for web pages.


\video{Introduction}{12:05}   {Web Search and  Information Retrieval}{https://www.youtube.com/watch?v=p-0NtNTzoh8}


\slides{Introduction}{6}  {Web  Search and Information Retrieval}{https://drive.google.com/open?id=0B8936_ytjfjmSm8zNmZ5VFJxRms}


\subsubsection{Cloud Application in
Research}\label{cloud-application-in-research}

We describe scientific applications and how they map onto clouds,
supercomputers, grids and high throughput systems. He likes the cloud
use of the Internet of Things and gives examples.


\video{Introduction}{33:51}{Cloud Applications  in Research}{https://www.youtube.com/watch?v=U3ZG2qOFpxE}


\slides{Introduction}{20}  {Cloud  Applications in Research}{https://drive.google.com/open?id=0B8936_ytjfjma0RhdU0zdkxmczA}

\subsubsection{Parallel Computing and
MapReduce}\label{parallel-computing-and-mapreduce}

We define MapReduce and gives a homely example from fruit blending.


\video{Introduction}{14:02}  {Computing and  MapReduce}{https://www.youtube.com/watch?v=aQ8NMxe9IsU}


\slides{Introduction}{9}  {Computing  and MapReduce}{https://drive.google.com/open?id=0B8936_ytjfjmeTl4NWhHRjJMOGc}

\subsubsection{Data Science Education}\label{data-science-education}

We discuss one reasons for
\textasciitilde{}\textasciitilde{} Data Science as an educational
initiative and aspects of its Indiana University implementation. Then
general; features of online education are discussed with clear growth
spearheaded by MOOC's where we use this material and others as an example.
He stresses the choice between one class to 100,000 students or 2,000
classes to 50 students and an online library of MOOC lessons. In olden
days he suggested `'hermit's cage virtual university''
\textasciitilde{}\textasciitilde{} gurus in isolated caves putting
together exciting curricula outside the traditional university model.
Grading and mentoring models and important online tools are discussed.
Clouds have MOOC's describing them and MOOC's are stored in clouds; a
pleasing symmetry.


\video{Introduction}{28:08}   {Data Science  Education}{https://www.youtube.com/watch?v=bA_eNjJTmRQ}


\slides{Introduction}{19}  {Data  Science Education}{https://drive.google.com/open?id=0B8936_ytjfjmT0J1RjYwY1VwZ1k}


\subsubsection{Conclusions}\label{conclusions}

The conclusions highlight clouds, data-intensive methodology,
employment, data science, MOOC's and never forget the Big Data ecosystem
in one sentence ``Use Clouds running Data Analytics Collaboratively
processing Big Data to solve problems in X-Informatics educated in data
science''


\video{Introduction}{4:59}  {Conclusions}{https://www.youtube.com/watch?v=FmcR5mrhYvk}

\slides{Introduction}{4}  {Conclusions}{https://drive.google.com/open?id=0B8936_ytjfjmVjRNeG1pdUNnMlE}


\subsubsection{Resources}\label{resources}

\begin{itemize}
\item
  \url{http://www.gartner.com/technology/home.jsp} and many web links
\item
  Meeker/Wu May 29 2013 Internet Trends D11 Conference
  \url{http://www.slideshare.net/kleinerperkins/kpcb-internet-trends-2013}
\item
  \url{http://cs.metrostate.edu/~sbd/slides/Sun.pdf}
\item
  Taming The Big Data Tidal Wave: Finding Opportunities in Huge Data
  Streams with Advanced Analytics, Bill Franks Wiley ISBN:
  978-1-118-20878-6
\item
  Bill Ruh
  \url{http://fisheritcenter.haas.berkeley.edu/Big_Data/index.html}
\item
  \url{http://www.genome.gov/sequencingcosts/}
\item
  CSTI General Assembly 2012, Washington, D.C., USA Technical Activities
  Coordinating Committee (TACC) Meeting, Data Management, Cloud
  Computing and the Long Tail of Science October 2012 Dennis Gannon
\item
  \url{http://www.microsoft.com/en-us/news/features/2012/mar12/03-05CloudComputingJobs.aspx}
\item
  \url{http://www.mckinsey.com/mgi/publications/big_data/index.asp}
\item
  Tom Davenport
  \url{http://fisheritcenter.haas.berkeley.edu/Big_Data/index.html}
\item
  \url{http://research.microsoft.com/en-us/people/barga/sc09_cloudcomp_tutorial.pdf}
\item
  \url{http://research.microsoft.com/pubs/78813/AJ18_EN.pdf}
\item
  \url{http://www.google.com/green/pdfs/google-green-computing.pdf}
\item
  \url{http://www.wired.com/wired/issue/16-07}
\item
  \url{http://research.microsoft.com/en-us/collaboration/fourthparadigm/}
\item
  Jeff Hammerbacher
  \url{http://berkeleydatascience.files.wordpress.com/2012/01/20120117berkeley1.pdf}
\item
  \url{http://grids.ucs.indiana.edu/ptliupages/publications/Where\%20does\%20all\%20the\%20data\%20come\%20from\%20v7.pdf}
\item
  \url{http://www.interactions.org/cms/?pid=1032811}
\item
  \url{http://www.quantumdiaries.org/2012/09/07/why-particle-detectors-need-a-trigger/atlasmgg/}
\item
  \url{http://www.sciencedirect.com/science/article/pii/S037026931200857X}
\item
  \url{http://www.slideshare.net/xamat/building-largescale-realworld-recommender-systems-recsys2012-tutorial}
\item
  \url{http://www.ifi.uzh.ch/ce/teaching/spring2012/16-Recommender-Systems_Slides.pdf}
\item
  \url{http://en.wikipedia.org/wiki/PageRank}
\item
  \url{http://pages.cs.wisc.edu/~beechung/icml11-tutorial/}
\item
  \url{https://sites.google.com/site/opensourceiotcloud/}
\item
  \url{http://datascience101.wordpress.com/2013/04/13/new-york-times-data-science-articles/}
\item
  \url{http://blog.coursera.org/post/49750392396/on-the-topic-of-boredom}
\item
  \url{http://x-informatics.appspot.com/course}
\item
  \url{http://iucloudsummerschool.appspot.com/preview}
\item
  \url{https://www.youtube.com/watch?v=M3jcSCA9_hM}
\end{itemize}


