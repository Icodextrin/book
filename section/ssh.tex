\section{Using SSH Keys}

If you do not know what ssh is we recommend that you
\href{http://openssh.com/manual.html}{read up on it} . However, the
simple material presented here will help you etting started quickly. It
can however not replace the more comprehensive documentation.

To access remote resources this is often achieved via SSH. You need to
provide a public ssh key to FutureSystem. We explain how to generate a
ssh key, upload it to the FutureSystem portal and log onto the
resources. This manual covers UNIX, Mac OS X.

\subsection{Using SSH from Windows}\label{using-ssh-from-windows}

For Linux users, please skip to the section s-ssh-generate

For Mac users, please skip to the section s-ssh-osx

\begin{description}
\item[For this class we recommend that you use a virtual]
machine via virtual box and use the Linux ssh instructions. The
information here is just provided for completness and no support will be
offered for native windows support.
\end{description}

Windows users need to have some special software to be able to use the
SSH commands. If you have one that you are comfortable with and know how
to setup key pairs and access the contents of your public key, please
feel free to use it.

The most popular software making ssh clients available to Windows users
include

\begin{itemize}
\tightlist
\item
  \href{http://cygwin.com/install.html}{cygwin}
\item
  \href{http://the.earth.li/~sgtatham/putty/0.62/htmldoc/}{putty}
\item
  or installing a \href{http://cygwin.com/install.html}{virtualiztion
  software} and running Linux virtual machine on your Windows OS.
\item
  using chocolatey
\item
  using bash ubuntu under WIndows 10 (we need a contribution on this)
\end{itemize}

We will be discussing here how to use it in Powershell with the help of
chopolatey. Other options may be better suited for you and we leave it
up to you to make this decission. In general we recommend that you use
an ubuntu OS either on bare hardware or a virtual machine. Naturally
your computer must support this. It will be up to you to find such a
computer.

However if you want a unix like environments with ssh you can use
Chocolatey.

Chocolatey is a software management tool that mimics the install
experience that you have on Linux and OSX. It has a repository with many
packages. Before using and installing a package be aware of the
consequences when installing software on your computer. Please be aware
that there could be malicious code offered in the chocolatey repository
although the distributors try to remove them.

The installation is sufficently explained at

  \URL{https://chocolatey.org/install}

Once installed you have a command choco and you should make sure you
have the newest version with :

\begin{verbatim}
choco upgrade chocolatey
\end{verbatim}

Now you can browse packages at

\URL{https://chocolatey.org/packages}

Search for openssh and see the results. You may find different versions.
Select the one that most suits you and satisfies your security
requirements as well as your architecture. Lets assume you chose the
Microsoft port, than you can install it with:

\begin{verbatim}
choco install win32-openssh
\end{verbatim}

\begin{description}
\item[If you have a different version such as a 64 bit version]
please find teh appropriate commands
\end{description}

Other packages of interest include

\begin{itemize}
\item LaTeX:: choco install miktex
\item jabref: choco install jabref
\item pycharm: choco install pycharm-community
\item python 2.7.11: choco install python2
\item pip: choco install pip
\item virtual box: choco install virtualbox
\item emacs: choco install emacs
\item lyx: choco install lyx
\item vagrant: choco install vagrant
\end{itemize}

Before installing any of them evaluate if you need them.

\subsection{Using SSH on Mac OS X}\label{using-ssh-on-mac-os-x}

Mac OS X comes with an ssh client. In order to use it you need to open
the \texttt{Terminal.app} application. Go to \texttt{Finder}, then click
\texttt{Go} in the menu bar at the top of the screen. Now click
\texttt{Utilities} and then open the \texttt{Terminal} application.

\subsection{Generate a SSH key}\label{generate-a-ssh-key}

First we must generate a ssh key with the tool
\href{http://linux.die.net/man/1/ssh-keygen}{ssh-keygen}. This program
is commonly available on most UNIX systems (this includes Cygwin if you
installed the ssh module or use our pre-generated cygwin executable). It
will ask you for the location and name of the new key. It will also ask
you for a passphrase, which you \textbf{MUST} provide. Some teachers and
teaching assistants advice you to not use passphrases. This is
\textbf{WRONG} as it allows someone that gains access to your computer
to also gain access to all resources that have the public key. Also,
please use a strong passphrase to protect it appropriately.

In case you already have a ssh key in your machine, you can reuse it and
skip this whole section.

To generate the key, please type:

Example:

\begin{verbatim}
ssh-keygen -t rsa -C localname@indiana.edu
\end{verbatim}

This command requires the interaction of the user. The first question
is:

\begin{verbatim}
Enter file in which to save the key (/home/localname/.ssh/id_rsa): 
\end{verbatim}

We recommend using the default location \textasciitilde{}/.ssh/ and the
default name id\_rsa. To do so, just press the enter key.

Your \emph{localname} is the username on your computer.

The second and third question is to protect your ssh key with a
passphrase. This passphrase will protect your key because you need to
type it when you want to use it. Thus, you can either type a passphrase
or press enter to leave it without passphrase. To avoid security
problems, you \textbf{MUST} chose a passphrase. Make sure to not just
type return for an empty passphrase:

\begin{verbatim}
Enter passphrase (empty for no passphrase):
\end{verbatim}

and:

\begin{verbatim}
Enter same passphrase again:
\end{verbatim}

If executed correctly, you will see some output similar to:

\begin{verbatim}
Generating public/private rsa key pair.
Enter file in which to save the key (/home/localname/.ssh/id_rsa): 
Enter passphrase (empty for no passphrase):
Enter same passphrase again:
Your identification has been saved in /home/localname/.ssh/id_rsa.
Your public key has been saved in /home/localname/.ssh/id_rsa.pub.
The key fingerprint is:
34:87:67:ea:c2:49:ee:c2:81:d2:10:84:b1:3e:05:59 localname@indiana.edu
The key's random art image  File "/Users/grey/.pyenv/versions/2.7.13/envs/ENV2/lib/python2.7/site-packages/traitlets/config/application.py", line 445, in initialize_subcommand
subapp = import_item(subapp)
\end{verbatim}

\begin{quote}
\begin{description}
\item[File
``/Users/grey/.pyenv/versions/2.7.13/envs/ENV2/lib/python2.7/site-packages/ipython\_genutils/importstring.py'',
line 31, in import\_item]
module = \_\_import\_\_(package, fromlist={[}obj{]})
\end{description}
\end{quote}

\begin{description}
\item[ImportError: No module named nbconvert.nbconvertapp]
is:

\begin{verbatim}
+--[ RSA 2048]----+
|.+...Eo= .       |
| ..=.o + o +o    |
|O.  o o +.o      |
| = .   . .       |
+-----------------+
\end{verbatim}
\end{description}

Once, you have generated your key, you should have them in the .ssh
directory. You can check it by :

\begin{verbatim}
$ cat ~/.ssh/id_rsa.pub
\end{verbatim}

If everything is normal, you will see something like:

\begin{verbatim}
ssh-rsa AAAAB3NzaC1yc2EAAAADAQABAAABAQCXJH2iG2FMHqC6T/U7uB8kt
6KlRh4kUOjgw9sc4Uu+Uwe/EwD0wk6CBQMB+HKb9upvCRW/851UyRUagtlhgy
thkoamyi0VvhTVZhj61pTdhyl1t8hlkoL19JVnVBPP5kIN3wVyNAJjYBrAUNW
4dXKXtmfkXp98T3OW4mxAtTH434MaT+QcPTcxims/hwsUeDAVKZY7UgZhEbiE
xxkejtnRBHTipi0W03W05TOUGRW7EuKf/4ftNVPilCO4DpfY44NFG1xPwHeim
Uk+t9h48pBQj16FrUCp0rS02Pj+4/9dNeS1kmNJu5ZYS8HVRhvuoTXuAY/UVc
ynEPUegkp+qYnR user@myemail.edu
\end{verbatim}

\subsection{Add or Replace Passphrase for an Already Generated
Key}\label{add-or-replace-passphrase-for-an-already-generated-key}

In case you need to change your change passphrase, you can simply run
``ssh-keygen -p'' command. Then specify the location of your current
key, and input (old and) new passphrases. There is no need to
re-generate keys:

\begin{verbatim}
ssh-keygen -p
\end{verbatim}

You will see the following output once you have completed that step:

\begin{verbatim}
Enter file in which the key is (/home/localname/.ssh/id_rsa):
Enter old passphrase:
Key has comment '/home/localname/.ssh/id_rsa'
Enter new passphrase (empty for no passphrase):
Enter same passphrase again:
Your identification has been saved with the new passphrase.  
\end{verbatim}

\subsection{Upload the key to gitlab}\label{upload-the-key-to-gitlab}

Follow the instructions provided here:

  \URL{http://docs.gitlab.com/ce/ssh/README.html}

\subsection{Exercise}\label{exercise}

\begin{description}
\item[SSH.1:]
create an SSH key pair
\item[SSH.2:]
upload the key to github and/or gitlab. Create a fork in git and use
your ssh key to clone and commit to it
\item[SSH.3:]
Get an account on futuresystems.org (if you are authorized to do so).
Upload your key to futuresystems.org. Login to india.futuresystems.org
Note. that this could take some time as administrators need to approve
you. Be patient.
\end{description}
