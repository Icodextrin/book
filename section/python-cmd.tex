\section{cmd Module}\label{cmd-module}
\index{cmd}

\subsection{Introduction}\label{introduction}

The Python cmd module is useful for any more involved command-line
application. It is used in the
\href{http://cloudmesh.github.io/}{Cloudmesh Project}, for example, and
students in I524
\textless{}../../i524/index\textgreater{} have found it helpful in their
projects. The Python cmd module contains a public class, Cmd, designed
to be used as a base class for command processors such as interactive
shells and other command interpreters.

\subsection{\texorpdfstring{\emph{Hello, World} with
cmd}{Hello, World with cmd}}\label{hello-world-with-cmd}

This example shows a very simple command interpreter that simply
responds to the greet command.

In order to demonstrate commands provided by cmd, let's save the
following program in a file called helloworld.py.

\begin{lstlisting}
from __future__ import print_function, division
import cmd


class HelloWorld(cmd.Cmd):
    '''Simple command processor example.'''

    def do_greet(self, line):
        if line is not None and len(line.strip()) > 0:
            print('Hello, %s!' % line.strip().title())
        else:
            print('Hello!')

    def do_EOF(self, line):
        print('bye, bye')
        return True


if __name__ == '__main__':
    HelloWorld().cmdloop()
\end{lstlisting}

A session with this program might look like this:

\begin{lstlisting}
$ python helloworld.py

(Cmd) help

Documented commands (type help <topic>):
========================================
help

Undocumented commands:
======================
EOF  greet

(Cmd) greet
Hello!
(Cmd) greet albert
Hello, Albert!
<CTRL-D pressed>
(Cmd) bye, bye
\end{lstlisting}

The Cmd class can be used to customize a subclass that becomes a
user-defined command prompt. After you have executed your program,
commands defined in your class can be used. Take note of the following
in this example:

\begin{itemize}

\item
  The methods of the class of the form do\_xxx implement the shell
  commands, with xxx being the name of the command. For example, in the
  HelloWorld class, the function do\_greet maps to the greet on the
  command line.
\item
  The EOF command is a special command that is executed when you press
  CTRL-D on your keyboard.
\item
  As soon as any command method returns True the shell application
  exits. Thus, in this example the shell is exited by pressing CTRL-D,
  since the do\_EOF method is the only one that returns True.
\item
  The shell application is started by calling the cmdloop method of the
  class.
\end{itemize}

\subsection{A More Involved Example}\label{a-more-involved-example}

Let's look at a little more involved example. Save the following code in
a file called calculator.py.

\begin{lstlisting}
from __future__ import print_function, division
import cmd


class Calculator(cmd.Cmd):
 prompt = 'calc >>> '
 intro = 'Simple calculator that can do addition, subtraction, multiplication and division.'

 def do_add(self, line):
     args = line.split()
     total = 0
     for arg in args:
         total += float(arg.strip())
     print(total)

 def do_subtract(self, line):
     args = line.split()
     total = 0
     if len(args) > 0:
         total = float(args[0])
     for arg in args[1:]:
         total -= float(arg.strip())
     print(total)

 def do_EOF(self, line):
     print('bye, bye')
     return True


if __name__ == '__main__':
 Calculator().cmdloop()
\end{lstlisting}

A session with this program might look like this:

\begin{lstlisting}
$ python calculator.py
Simple calculator that can do addition, subtraction, multiplication and division.
calc >>> help

Documented commands (type help <topic>):
========================================
help

Undocumented commands:
======================
EOF  add  subtract

calc >>> add
0
calc >>> add 4 5 6
15.0
calc >>> subtract
0
calc >>> subtract 10 2
8.0
calc >>> subtract 10 2 20
-12.0
calc >>> bye, bye
\end{lstlisting}

\begin{itemize}

\item
  In this case we are using the prompt and intro class variables to
  define what the default prompt looks like and a welcome message when
  the command interpreter is invoked.
\item
  In the add and subtract commands we are using the strip and split
  methods to parse all arguments. If you want to get fancy, you can use
  Python modules like getopts or argparse for this, but this is not
  necessary in this simple example.
\end{itemize}

\subsection{Help Messages}\label{help-messages}

Notice that all commands presently show up as undocumented. To remedy
this, we can define help\_ methods for each command:

\begin{lstlisting}
from __future__ import print_function, division
import cmd


class Calculator(cmd.Cmd):
  prompt = 'calc >>> '
  intro = 'Simple calculator that can do addition, subtraction, multiplication and division.'

  def do_add(self, line):
      args = line.split()
      total = 0
      for arg in args:
          total += float(arg.strip())
      print(total)

  def help_add(self):
      print('\n'.join([
          'add [number,]',
          'Add the arguments together and display the total.'
      ]))

  def do_subtract(self, line):
      args = line.split()
      total = 0
      if len(args) > 0:
          total = float(args[0])
      for arg in args[1:]:
          total -= float(arg.strip())
      print(total)

  def help_subtract(self):
      print('\n'.join([
          'subtract [number,]',
          'Subtract all following arguments from the first argument.'
      ]))

  def do_EOF(self, line):
      print('bye, bye')
      return True


if __name__ == '__main__':
  Calculator().cmdloop()
\end{lstlisting}

Now, we can obtain help for the add and subtract commands:

\begin{lstlisting}
$ python calculator.py
Simple calculator that can do addition, subtraction, multiplication and division.
calc >>> help

Documented commands (type help <topic>):
========================================
add  help  subtract

Undocumented commands:
======================
EOF

calc >>> help add
add [number,]
Add the arguments together and display the total.
calc >>> help subtract
subtract [number,]
Subtract all following arguments from the first argument.
calc >>> bye, bye
\end{lstlisting}
%$

\subsection{Useful Links}\label{useful-links}

\begin{itemize}

\item
  \href{https://docs.python.org/2/library/cmd.html}{Python Docs}
\item
  \href{https://pymotw.com/2/cmd/}{Python Module of the Week: cmd --
  Create line-oriented command processors}
\end{itemize}
