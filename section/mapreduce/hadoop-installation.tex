\section{HADOOP}\label{hadoop}

Hadoop is an open source framework for storage and processing of large
datasets on commodity clusters. Hadoop internally uses its own file
system called HDFS (Hadoop Distributed File System). In this tutorial
our main focus is on using Hadoop 2.6.0 version along with the Yarn
resource manager. We will be focusing more on the Hadoop installation in
this chapter.

\subsection{Prerequisits}\label{prerequisits}

We assume that you have emacs, ssh, and rsync installed

\begin{Shaded}
\begin{Highlighting}[]
\NormalTok{$ }\KeywordTok{sudo} \NormalTok{apt-get install emacs}
\NormalTok{$ }\KeywordTok{sudo} \NormalTok{apt-get install ssh}
\NormalTok{$ }\KeywordTok{sudo} \NormalTok{apt-get install rsync}
\end{Highlighting}
\end{Shaded}

\subsubsection{User and User Group
Creation}\label{user-and-user-group-creation}

In installing hadoop, for the security reasons we will stick to a
specific user for hadoop and a specific user group.

\begin{Shaded}
\begin{Highlighting}[]
\NormalTok{$ }\KeywordTok{sudo} \NormalTok{addgroup hadoop_group}
\NormalTok{$ }\KeywordTok{sudo} \NormalTok{adduser --ingroup hadoop_group hduser1}
\NormalTok{$ }\KeywordTok{sudo} \NormalTok{adduser hduser1 sudo}
\end{Highlighting}
\end{Shaded}

After this point we are completed giving sudo privileges to the created
user in the specific user group.

\subsubsection{Configuring SSH}\label{configuring-ssh}

Here we have to configure ssh for the created user for security in the
hadoop installation. Run the following commands to configure ssh.

\begin{Shaded}
\begin{Highlighting}[]
\NormalTok{$ }\KeywordTok{su} \NormalTok{– hduser1}
\end{Highlighting}
\end{Shaded}

\begin{Shaded}
\begin{Highlighting}[]
\NormalTok{$ }\KeywordTok{ssh-keygen} \NormalTok{-t rsa}
\end{Highlighting}
\end{Shaded}

After this step you will see some console screen as shown below. Follow
the instructions in the command line.

When you see the following console input, press ENTER.

\begin{Shaded}
\begin{Highlighting}[]
\KeywordTok{Enter} \NormalTok{file in which to save the key (/home/hduser1/.ssh/id_rsa)}\KeywordTok{:}
\end{Highlighting}
\end{Shaded}

Next you will be asked to enter a password for ssh configuration,

\begin{Shaded}
\begin{Highlighting}[]
\KeywordTok{Enter} \NormalTok{passphrase (empty for no passphrase)}\KeywordTok{:}
\end{Highlighting}
\end{Shaded}

Here enter the same password

\begin{Shaded}
\begin{Highlighting}[]
\KeywordTok{Enter} \NormalTok{same passphrase again:}
\end{Highlighting}
\end{Shaded}

Finally you will see something like this after these steps are finished.

\begin{Shaded}
\begin{Highlighting}[]
\KeywordTok{Generating} \NormalTok{public/private rsa key pair.}
\KeywordTok{Enter} \NormalTok{file in which to save the key (/home/hduser1/.ssh/id_rsa)}\KeywordTok{:}
\KeywordTok{Created} \NormalTok{directory }\StringTok{'/home/hduser1/.ssh'}\NormalTok{.}
\KeywordTok{Enter} \NormalTok{passphrase (empty for no passphrase)}\KeywordTok{:}
\KeywordTok{Enter} \NormalTok{same passphrase again:}
\KeywordTok{Your} \NormalTok{identification has been saved in /home/hduser1/.ssh/id_rsa.}
\KeywordTok{Your} \NormalTok{public key has been saved in /home/hduser1/.ssh/id_rsa.pub.}
\KeywordTok{The} \NormalTok{key fingerprint is:}
\KeywordTok{SHA256}\NormalTok{:0UBCPd6oYp7MEzCpOhMhNiJyQo6PaPCDuOT48xUDDc0 hduser1@messi}
\KeywordTok{The} \NormalTok{key}\StringTok{'s randomart image is:}
\StringTok{+---[RSA 2048]----+}
\StringTok{|    .+ooo        |}
\StringTok{| .   oE.oo       |}
\StringTok{|+  .. ...+.      |}
\StringTok{|X+=  .  o..      |}
\StringTok{|XX.o  o.S        |}
\StringTok{|Bo+ + .o         |}
\StringTok{|*o * +.          |}
\StringTok{|*.. *.           |}
\StringTok{| +.o..           |}
\StringTok{+----[SHA256]-----+}
\end{Highlighting}
\end{Shaded}

You have successfully configured ssh.

\subsubsection{Install Java}\label{install-java}

First log on to the newly created user :

\begin{Shaded}
\begin{Highlighting}[]
\NormalTok{$ }\KeywordTok{su} \NormalTok{- hduser1}
\end{Highlighting}
\end{Shaded}

\begin{Shaded}
\begin{Highlighting}[]
\NormalTok{$ }\KeywordTok{mkdir} \NormalTok{-p ~/cloudmesh/bin}
\NormalTok{$ }\KeywordTok{cd} \NormalTok{~/cloudmesh/bin}
\NormalTok{$ }\KeywordTok{wget} \NormalTok{-c --header }\StringTok{"Cookie: oraclelicense=accept-securebackup-cookie"} \StringTok{"http://download.oracle.com/otn-pub/java/jdk/8u161-b12/2f38c3b165be4555a1fa6e98c45e0808/jdk-8u161-linux-x64.tar.gz"}
\NormalTok{$ }\KeywordTok{tar} \NormalTok{xvzf jdk-8u161-linux-x64.tar.gz}
\end{Highlighting}
\end{Shaded}

\subsection{Installation Of Hadoop}\label{installation-of-hadoop}

First we will take a look on how to install Hadoop 2.6.0 on Ubuntu
16.04. We may need a prior folder structure to do the installation
properly.

\begin{Shaded}
\begin{Highlighting}[]
\NormalTok{$ }\KeywordTok{cd} \NormalTok{~/cloudmesh/bin/}
\NormalTok{$ }\KeywordTok{wget} \NormalTok{https://archive.apache.org/dist/hadoop/core/hadoop-2.6.0/hadoop-2.6.0.tar.gz}
\NormalTok{$ }\KeywordTok{tar} \NormalTok{-xzvf hadoop-2.6.0.tar.gz}
\end{Highlighting}
\end{Shaded}

\subsubsection{Add Environmental
Variables}\label{add-environmental-variables}

In Ubuntu the environmental variables are setup in a file called bashrc
at it can be accessed the following way

\begin{Shaded}
\begin{Highlighting}[]
\NormalTok{$ }\KeywordTok{emacs} \NormalTok{~/.bashrc}
\end{Highlighting}
\end{Shaded}

where is hadoop comming from

\begin{Shaded}
\begin{Highlighting}[]
\KeywordTok{export} \OtherTok{JAVA_HOME=}\NormalTok{~/cloudmesh/bin/jdk1.8.0_161}
\KeywordTok{export} \OtherTok{HADOOP_HOME=}\NormalTok{~/cloudmesh/bin/hadoop-2.6.0}
\KeywordTok{export} \OtherTok{YARN_HOME=$HADOOP_HOME}
\KeywordTok{export} \OtherTok{HADOOP_CONF_DIR=$HADOOP_HOME}\NormalTok{/etc/hadoop}
\KeywordTok{export} \OtherTok{PATH=$HADOOP_HOME}\NormalTok{/bin:}\OtherTok{$JAVA_HOME}\NormalTok{/bin:}\OtherTok{$PATH}
\end{Highlighting}
\end{Shaded}

In Emacs to save the file Ctrl+X+S and Ctrl+X+C to exit. After editing
you must update the variables in the system.

\begin{Shaded}
\begin{Highlighting}[]
\NormalTok{$ }\KeywordTok{source} \NormalTok{~/.bashrc}
\NormalTok{$ }\KeywordTok{java} \NormalTok{-version}
\end{Highlighting}
\end{Shaded}

If you have installed things properly there will be no errors. It will
show the version as follows,

\begin{Shaded}
\begin{Highlighting}[]
\KeywordTok{java} \NormalTok{version }\StringTok{"1.8.0_161"}
\KeywordTok{Java}\NormalTok{(TM) }\KeywordTok{SE} \NormalTok{Runtime Environment (build 1.8.0_161-b12)}
\KeywordTok{Java} \NormalTok{HotSpot(TM) }\KeywordTok{64-Bit} \NormalTok{Server VM (build 25.161-b12, mixed mode)}
\end{Highlighting}
\end{Shaded}

And verifying the hadoop installation,

\begin{Shaded}
\begin{Highlighting}[]
\NormalTok{$ }\KeywordTok{hadoop}
\end{Highlighting}
\end{Shaded}

If you have successfully installed this, there must be a message shown
as below.

\begin{Shaded}
\begin{Highlighting}[]
\KeywordTok{Usage}\NormalTok{: hadoop [--config confdir] COMMAND}
       \KeywordTok{where} \NormalTok{COMMAND is one of:}
  \KeywordTok{fs}                   \NormalTok{run a generic filesystem user client}
  \KeywordTok{version}              \NormalTok{print the version}
  \KeywordTok{jar} \KeywordTok{<}\NormalTok{jar}\KeywordTok{>}            \NormalTok{run a jar file}
  \KeywordTok{checknative} \NormalTok{[-a}\KeywordTok{|-h}\NormalTok{]  check native hadoop and compression libraries availability}
  \KeywordTok{distcp} \KeywordTok{<}\NormalTok{srcurl}\KeywordTok{>} \KeywordTok{<}\NormalTok{desturl}\KeywordTok{>} \NormalTok{copy file or directories recursively}
  \KeywordTok{archive} \NormalTok{-archiveName NAME -p }\KeywordTok{<}\NormalTok{parent path}\KeywordTok{>} \KeywordTok{<}\NormalTok{src}\KeywordTok{>}\NormalTok{* }\KeywordTok{<}\NormalTok{dest}\KeywordTok{>} \NormalTok{create a hadoop archive}
  \KeywordTok{classpath}            \NormalTok{prints the class path needed to get the}
  \KeywordTok{credential}           \NormalTok{interact with credential providers}
                       \KeywordTok{Hadoop} \NormalTok{jar and the required libraries}
  \KeywordTok{daemonlog}            \NormalTok{get/set the log level for each daemon}
  \KeywordTok{trace}                \NormalTok{view and modify Hadoop tracing settings}
 \KeywordTok{or}
  \KeywordTok{CLASSNAME}            \NormalTok{run the class named CLASSNAME}

\KeywordTok{Most} \NormalTok{commands print help when invoked w/o parameters.}
\end{Highlighting}
\end{Shaded}

