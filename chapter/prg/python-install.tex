\FILENAME

\section{Python Installation}\label{python-installation}

Python is easy to install and very good instructions for most platforms
can be found on the python.org Web page. We will be using Python 2.7.13
and/or Python 3 in our activities.

To manage python modules, it is useful to have
\href{https://pypi.python.org/pypi/pip}{pip} package installation tool
on your system.

In the tutorial, we assume that you have a computer with python
installed. However, we also recommend that for the class you use
Python's virtualenv (see below) to isolate your development Python from
the system installed Python.

\subsection{Managing custom Python
installs}\label{managing-custom-python-installs}

Often you have your own computer and you do not like to change its
environment to keep it in pristine condition. Python comes with mnay
libraries that could for example conflict with libraries that you have
installed. To avoid this it is bets to work in an isolated python we can
use tools such as virtualenv, pyenv or pyvenv for 3.6.2. Which you use
depends on you, but we highly recommend pyenv if you can.

\subsubsection{Managing Multiple Python Versions with
Pyenv}\label{managing-multiple-python-versions-with-pyenv}

Python has several versions that are used by the community. This
includes Python 2 and Python 3, but alls different management of the
python libraries. As each OS may have their own version of python
installed. It is not recommended that you modify that version. Instead
you may want to create a localized python installation that you as a
user can modify. To do that we recommend \emph{pyenv}. Pyenv allows
users to switch between multiple versions of Python
(\url{https://github.com/yyuu/pyenv}). To summarize:

\begin{itemize}
\tightlist
\item
  users to change the global Python version on a per-user basis;
\item
  users to enable support for per-project Python versions;
\item
  easy version changes without complex environment variable management;
\item
  to search installed commands across different python versions;
\item
  integrate with tox (\url{https://tox.readthedocs.io/}).
\end{itemize}

\paragraph{Instalation without pyenv}\label{instalation-without-pyenv}

If you need to have more than one python version installed and do not
want or can use pyenv, we recommend you download and install python
2.7.13 and 3.6.2 from python.org
(\url{https://www.python.org/downloads/})

\paragraph{Disabeling wrong python installs on
OSX}\label{disabeling-wrong-python-installs-on-osx}

While working with students we have seen at times that they take other
classes either at universities or online that teach them how to program
in python. Unfortuanatley, although they seem to do that they often
ignore to teach you how to properly install python. I just reachentl had
a students that had installed python 7 times on his OSX machine, while
another student had 3 different instalations, all of which confliced
with each other as they were not set up properly.

We recommend that you inspect if you have a files such as
\textasciitilde{}/.bashrc or \textasciitilde{}/.bashrc\_profile in your
ehome directory and identify if it activates various versions of python
on your computer. If so you could try to deactivate them while
outcommenting the various versions with the \# character at the
beginning of the line, start a new terminal and see if the terminal
shell still works. Than you can follow our instructions here while using
an install on pyenv.

\paragraph{Install pyenv on OSX from
git}\label{install-pyenv-on-osx-from-git}

This is our recommended way to install pyenv on OSX:

\begin{verbatim}
$ git clone https://github.com/pyenv/pyenv.git ~/.pyenv
$ git clone https://github.com/pyenv/pyenv-virtualenv.git ~/.pyenv/plugins/pyenv-virtualenv
$ git clone https://github.com/yyuu/pyenv-virtualenvwrapper.git ~/.pyenv/plugins/pyenv-virtualenvwrapper
$ echo 'export PYENV_ROOT="$HOME/.pyenv"' >> ~/.bash_profile
$ echo 'export PATH="$PYENV_ROOT/bin:$PATH"' >> ~/.bash_profile
\end{verbatim}

\paragraph{Instalation of Homebrew}\label{instalation-of-homebrew}

Before installing anything on your computer make sure you have enough
space. Use in the terminal the command:

\begin{verbatim}
$ df -h
\end{verbatim}

which gives your an overview of your file system. If you do not have
enough space, please make sure you free up unused files from your drive.

In many occasions it is beneficial to use readline as it provides nice
editing features for the terminal and xz for completion. First, make
sure you have xcode installed:

\begin{verbatim}
$ xcode-select --install
\end{verbatim}

Next install homebrew, pyenv, pyenv-virtualenv and pyenv-virtualwrapper.
Additionally install readline and some compression tools:

\begin{verbatim}
/usr/bin/ruby -e "$(curl -fsSL https://raw.githubusercontent.com/Homebrew/install/master/install)"
brew update
brew install readline xz
\end{verbatim}

\paragraph{Install pyenv on OSX with
Homebrew}\label{install-pyenv-on-osx-with-homebrew}

We describe here a mechanism of installing pyenv with homebrew. Other
mechanisms can be found on the pyenv documentation page
(\url{https://github.com/yyuu/pyenv-installer}). You must have homebrew
installed as discussed in the previous section.

To install pyenv with homebrew execute in the terminal:

\begin{verbatim}
brew install pyenv pyenv-virtualenv pyenv-virtualenvwrapper
\end{verbatim}

\paragraph{Install pyenv on Ubuntu}\label{install-pyenv-on-ubuntu}

The following steps will install pyenv in a new ubuntu 16.04
distribution.

Start up a terminal and execute in the terminal the following commands.
We recommend that you do it one command at a time so you can observe if
the command succeeds:

\begin{verbatim}
$ sudo apt-get update
$ sudo apt-get install git python-pip make build-essential libssl-dev
$ sudo apt-get install zlib1g-dev libbz2-dev libreadline-dev libsqlite3-dev
$ sudo pip install virtualenvwrapper

$ git clone https://github.com/yyuu/pyenv.git ~/.pyenv
$ git clone https://github.com/pyenv/pyenv-virtualenv.git ~/.pyenv/plugins/pyenv-virtualenv   
$ git clone https://github.com/yyuu/pyenv-virtualenvwrapper.git ~/.pyenv/plugins/pyenv-virtualenvwrapper

$ echo 'export PYENV_ROOT="$HOME/.pyenv"' >> ~/.bashrc
$ echo 'export PATH="$PYENV_ROOT/bin:$PATH"' >> ~/.bashrc
\end{verbatim}

Now that you have installed pyenv it is not yet activated in your
current terminal. The easiest thing to do is to start a new terminal and
typ in:

\begin{verbatim}
which pyenv
\end{verbatim}

If you see a response pyenv is installed and you can proceed with the
next steps.

\begin{description}
\item[Please remember whenever you modify .bashrc or]
.bash\_profile you need to start a new terminal.
\end{description}

\paragraph{Install Different Python
Versions}\label{install-different-python-versions}

Pyenv provides a large list of different python versions. To see the
entire list please use the command:

\begin{verbatim}
$ pyenv install -l
\end{verbatim}

However, for us we only need to worry about python 2.7.13 and python
3.6.2 (once 3.6.2 becomes available we will use that). You can now
install different versions of python into your local environment with
the following commands:

\begin{verbatim}
$ pyenv install 2.7.13
$ pyenv install 3.6.2
\end{verbatim}

You can set the global python default version with:

\begin{verbatim}
$ pyenv global 2.7.13
\end{verbatim}

Type the following to determine which version you activated:

\begin{verbatim}
$ pyenv version
\end{verbatim}

Type the following to determine which versions you have available:

\begin{verbatim}
$ pyenv versions
\end{verbatim}

Associate a specific environment name with a certain python version, use
the following commands:

\begin{verbatim}
$ pyenv virtualenv 2.7.13 ENV2
$ pyenv virtualenv 3.6.2 ENV3
\end{verbatim}

In the example above, ENV2 would represent python 2.7.13 while ENV3
would represent python 3.6.2. Often it is easier to type the alias
rather than the explicit version.

\paragraph{Set up the Shell}\label{set-up-the-shell}

To make all work smoothly from your terminal, you can include the
following in your .bashrc files:

\begin{verbatim}
export PYENV_VIRTUALENV_DISABLE_PROMPT=1
eval "$(pyenv init -)"
eval "$(pyenv virtualenv-init -)"

__pyenv_version_ps1() {
  local ret=$?;
  output=$(pyenv version-name)
  if [[ ! -z $output ]]; then
    echo -n "($output)"
  fi
  return $ret;
}

PS1="\$(__pyenv_version_ps1) ${PS1}"
\end{verbatim}

We recommend that you do this towards the end of your file.

\paragraph{Switching Environments}\label{switching-environments}

After setting up the different environments, switching between them is
now easy. Simply use the following commands:

\begin{verbatim}
(2.7.13) $ pyenv activate ENV2
(ENV2) $ pyenv activate ENV3
(ENV3) $ pyenv activate ENV2
(ENV2) $ pyenv deactivate ENV2
(2.7.13) $ 
\end{verbatim}

To make it even easier, you can add the following lines to your
.bash\_profile file:

\begin{verbatim}
alias ENV2="pyenv activate ENV2"
alias ENV3="pyenv activate ENV3"
\end{verbatim}

If you start a new terminal, you can switch between the different
versions of python simply by typing:

\begin{verbatim}
$ ENV2
$ ENV3
\end{verbatim}

\subsection{Instalation without
pyenv}\label{instalation-without-pyenv-1}

If you need to have more than one python version installed and do not
want or can use pyenv, we recommend you download and install python
2.7.13 and 3.6.2 from python.org
(\url{https://www.python.org/downloads/})

\subsubsection{Make sure pip is up to
date}\label{make-sure-pip-is-up-to-date}

As you will want to install other packages, make sure pip is up to date:

\begin{verbatim}
pip install pip -U
\end{verbatim}

pyenv virtualenv anaconda3-4.3.1 ANA3 pyenv activate ANA3

\subsection{Anaconda and Miniconda}\label{anaconda-and-miniconda}

\begin{description}
\item[We do not recommend that you use anaconda or miniconda as it may]
interfere with your default python interpreters and setup.
\end{description}

Please note that beginners to pyton should always use anaconda or
miniconda only afterthey have installed pyenv and use it. For this class
neither anaconda nor miniconda is required. In fact we do not recommend
it. We keep this section as we know that other classes at IU may use
anaconda. We are not aware if these classes teach you the right way to
install it, with \emph{pyenv}.

\subsubsection{Miniconda}\label{miniconda}

\begin{description}
\item[This section about miniconda is experimental and has not]
been tested. We are looking for contributors that help completing it. If
you use anaconda or miniconda we recommend to manage it via pyenv.
\end{description}

To install mini conda you can use the following commands:

\begin{verbatim}
$ mkdir ana
$ cd ana
$ pyenv install miniconda3-latest
$ pyenv local miniconda3-latest
$ pyenv activate miniconda3-latest
$ conda create -n ana anaconda
\end{verbatim}

To activate use:

\begin{verbatim}
$ source activate ana
\end{verbatim}

To deactivate use:

\begin{verbatim}
$ source deactivate
\end{verbatim}

To install cloudmesh cmd5 please use:

\begin{verbatim}
$ pip install cloudmesh.cmd5
$ pip install cloudmesh.sys
\end{verbatim}

\subsubsection{Anaconda}\label{anaconda}

\begin{description}
\item[This section about anaconda is experimental and has not]
been tested. We are looking for contributors that help completing it.
\end{description}

You can add anaconda to your pyenv with the following commands:

\begin{verbatim}
pyenv install anaconda3-4.3.1
\end{verbatim}

To switch more easily we recommend that you use the following in your
.bash\_profile file:

\begin{verbatim}
alias ANA="pyenv activate anaconda3-4.3.1"
\end{verbatim}

Once you have done this you can easily switch to anaconda with the
command:

\begin{verbatim}
$ ANA
\end{verbatim}

Terminology in annaconda could lead to confusion. Thus we like to point
out that the version number of anaconda is unrelated to the python
version. Furthermore, anaconda uses the term root not for the root user,
but for the originating directory in which the anaconda program is
installed.

In case you like to build your own conda packages at a later time we
recommend that you install the conda-build package:

\begin{verbatim}
$ conda install conda-build
\end{verbatim}

When executing:

\begin{verbatim}
pyenv versions
\end{verbatim}

you will see after the install completed the anaconda versions
installed:

\begin{verbatim}
pyenv versions
system
2.7.13
2.7.13/envs/ENV2
3.6.2
3.6.2/envs/ENV3
ENV2 
ENV3
* anaconda3-4.3.1 (set by PYENV_VERSION environment variable)
\end{verbatim}

Let us now create virtualenv for anaconda:

\begin{verbatim}
$ pyenv virtualenv anaconda3-4.3.1 ANA
\end{verbatim}

To activate it you can now use:

\begin{verbatim}
$ pyenv ANA
\end{verbatim}

However, anaconda may modify your .bashrc or .bash\_profile files and ,
may result in incompatibilities with other python versions. For this
reason we recommend not to use it. If you find ways to get it to work
reliably with other versions, please let us know and we update this
tutorial.

To install cloudmesh cmd5 please use:

\begin{verbatim}
$ pip install cloudmesh.cmd5
$ pip install cloudmesh.sys
\end{verbatim}

\paragraph{Exercise}\label{exercise}

\begin{description}
\item[Epyenv.1:]
Write installation instructions for an operating system of your choice
and add to this documentation.
\item[Epyenv.2:]
Replicate the steps above, so you can type in ENV2 and ENV3 in your
terminals to switch between python 2 and 3.
\end{description}

\subsubsection{virtualenv}\label{virtualenv}

environment while using virtualenv,. Documentation about it can be found
at:

\begin{verbatim}
* https://virtualenv.pypa.io
\end{verbatim}

The installation is simple once you have pip installed. If it is not
installed you can say:

\begin{verbatim}
$ easy_install pip
\end{verbatim}

After that you can install the virtual env with:

\begin{verbatim}
$ pip install virtualenv
\end{verbatim}

To setup an isolated environment for example in the directory
\textasciitilde{}/ENV please use:

\begin{verbatim}
$ virtualenv ~/ENV
\end{verbatim}

To activate it you can use the command:

\begin{verbatim}
$ source ~/ENV/bin/activate
\end{verbatim}

you can put this command in your .bashrc or .bash\_profile files so you
do not forget to activate it. Instructions for this can be
found in our lesson on Linux \textless{}bashrc\textgreater{}.
