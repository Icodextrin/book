\FILENAME

\section{Statements and Strings}\label{statements-and-strings}
\index{Python!statements}
\index{Python!strings}

Let us explore the syntax of Python. Type into the interactive loop and
press Enter:

\begin{verbatim}
>>> print("Hello world from Python!")
Hello world from Python!
\end{verbatim}

What happened: the print function was given a \textbf{string} to
process. A string is a sequence of characters. A \textbf{character} can
be a alphabetic (A through Z, lower and upper case), numeric (any of the
digits), white space (spaces, tabs, newlines, etc), syntactic directives
(comma, colon, quotation, exclamation, etc), and so forth. A string is
just a sequence of the character and typically indicated by surrounding
the characters in double quotes.

Standard output is discussed in the ../../lesson/linux/shell lesson.

So, what happened when you pressed Enter? The interactive Python program
read the line print "Hello world from Python!", split it into the print
statement and the "Hello world from Python!" string, and then executed
the line, showing you the output.

\section{Variables}\label{variables}
\index{Python!variables}

You can store data into a \textbf{variable} to access it later. For
instance, instead of:

\begin{verbatim}
>>> print('Hello world from Python!')
\end{verbatim}

which is a lot to type if you need to do it multiple times, you can
store the string in a variable for convenient access:

\begin{verbatim}
>>> hello = 'Hello world from Python!'
>>> print(hello)
Hello world from Python!
\end{verbatim}

\section{Data Types}\label{data-types}
\index{Python!data types}

\subsection{Booleans}\label{booleans}

A \textbf{boolean} is a value that indicates \emph{truthness} of
something. You can think of it as a toggle: either ``on'' or ``off'',
``one'' or ``zero'', ``true'' or ``false''. In fact, the only possible
values of the \textbf{boolean} (or bool) type in Python are:

\begin{itemize}
\tightlist
\item
  True
\item
  False
\end{itemize}

You can combine booleans with \textbf{boolean operators}:

\begin{itemize}
\tightlist
\item
  and
\item
  or
\end{itemize}

\begin{verbatim}
>>> print(True and True)
True
>>> print(True and False)
False
>>> print(False and False)
False
>>> print(True or True)
True
>>> print(True or False)
True
>>> print(False or False)
False
\end{verbatim}

\subsection{Numbers}\label{numbers}
\index{Python!numbers}

The interactive interpreter can also be used as a calculator. For
instance, say we wanted to compute a multiple of 21:

\begin{verbatim}
>>> print(21 * 2)
42
\end{verbatim}

We saw here the print statement again. We passed in the result of the
operation 21 * 2. An \textbf{integer} (or \textbf{int}) in Python is a
numeric value without a fractional component (those are called
\textbf{floating point} numbers, or \textbf{float} for short).

The mathematical operators compute the related mathematical operation to
the provided numbers. Some operators are:

\begin{itemize}
\tightlist
\item
  * --- multiplication
\item
  / --- division
\item
  + --- addition
\item
  - --- subtraction
\item
  ** --- exponent
\end{itemize}

Exponentiation is read as x**y is x to the yth power:

\[x^y\]

You can combine \textbf{float}s and \textbf{int}s:

\begin{verbatim}
>>> print(3.14 * 42 / 11 + 4 - 2)
13.9890909091
>>> print(2**3)
8
\end{verbatim}

Note that \textbf{operator precedence} is important. Using parenthesis
to indicate affect the order of operations gives a difference results,
as expected:

\begin{verbatim}
>>> print(3.14 * (42 / 11) + 4 - 2)
11.42
>>> print(1 + 2 * 3 - 4 / 5.0)
6.2
>>> print( (1 + 2) * (3 - 4) / 5.0 )
-0.6
\end{verbatim}



\section{Module Management}\label{module-management}

A module allows you to logically organize your Python code. Grouping
related code into a module makes the code easier to understand and use.
A module is a Python object with arbitrarily named attributes that you
can bind and reference. A module is a file consisting of Python code. A
module can define functions, classes and variables. A module can also
include runnable code.

\subsection{Import Statement}\label{import-statement}

\begin{quote}
When the interpreter encounters an import statement, it imports the
module if the module is present in the search path. A search path is a
list of directories that the interpreter searches before importing a
module. The from\ldots{}import Statement Python's from statement lets
you import specific attributes from a module into the current namespace.
The from\ldots{}import has the following syntax - from modname:
\end{quote}

import name1{[}, name2{[}, \ldots{} nameN{]}{]}

When the interpreter encounters an import statement, it imports the
module if the module is present in the search path. A search path is a
list of directories that the interpreter searches before importing a
module.

\subsection{The from \ldots{} import
Statement}\label{the-from-import-statement}

Python's from statement lets you import specific attributes from a
module into the current namespace. The from \ldots{} import has the
following syntax:

\begin{verbatim}
::
\end{verbatim}

\begin{quote}
from module1 import name1{[}, name2{[}, \ldots{} nameN{]}{]}
\end{quote}

\section{Date Time in Python}\label{date-time-in-python}

The datetime module supplies classes for manipulating dates and times in
both simple and complex ways. While date and time arithmetic is
supported, the focus of the implementation is on efficient attribute
extraction for output formatting and manipulation. For related
functionality, see also the time and calendar modules.

The import Statement You can use any Python source file as a module by
executing an import statement in some other Python source file.

\begin{verbatim}
>>>from datetime import datetime
\end{verbatim}

This module offers a generic date/time string parser which is able to
parse most known formats to represent a date and/or time.

\begin{verbatim}
>>>from dateutil.parser import parse
\end{verbatim}

pandas is an open source Python library for data analysis that needs to
be imported.

\begin{verbatim}
>>>import pandas as pd
\end{verbatim}

Create a string variable with the class start time

\begin{verbatim}
>>>fall_start = '08-21-2017'
\end{verbatim}

Convert the string to datetime format

\begin{verbatim}
>>>datetime.strptime(fall_start, '%m-%d-%Y')
datetime.datetime(2017, 8, 21, 0, 0)
\end{verbatim}

Creating a list of strings as dates

\begin{verbatim}
>>>class_dates = ['8/25/2017', '9/1/2017', '9/8/2017', '9/15/2017', '9/22/2017', '9/29/2017']
\end{verbatim}

Convert Class\_dates strings into datetime format and save the list into
variable a

\begin{verbatim}
>>>a = [datetime.strptime(x, '%m/%d/%Y') for x in class_dates]
\end{verbatim}

Use parse() to attempt to auto-convert common string formats. Parser
must be a string or character stream, not list.

\begin{verbatim}
>>>parse(fall_start)
datetime.datetime(2017, 8, 21, 0, 0)
\end{verbatim}

Use parse() on every element of the Class\_dates string.

\begin{verbatim}
>>>[parse(x) for x in class_dates] 
[datetime.datetime(2017, 8, 25, 0, 0),
 datetime.datetime(2017, 9, 1, 0, 0),
 datetime.datetime(2017, 9, 8, 0, 0),
 datetime.datetime(2017, 9, 15, 0, 0),
 datetime.datetime(2017, 9, 22, 0, 0),
 datetime.datetime(2017, 9, 29, 0, 0)]  
\end{verbatim}

Use parse, but designate that the day is first.

\begin{verbatim}
>>>parse (fall_start, dayfirst=True)
datetime.datetime(2017, 8, 21, 0, 0)
\end{verbatim}

Create a dataframe.A DataFrame is a tablular data structure comprised of
rows and columns, akin to a spreadsheet, database table. DataFrame as a
group of Series objects that share an index (the column names).

\begin{verbatim}
>>>import pandas as pd
>>>data = {'class_dates': ['8/25/2017 18:47:05.069722', '9/1/2017 18:47:05.119994', 
                        '9/8/2017 18:47:05.178768', '9/15/2017 18:47:05.230071', 
                        '9/22/2017 18:47:05.230071', '9/29/2017 18:47:05.280592'], 
        'complete': [1, 0, 1, 1, 0, 1]} 
>>>df = pd.DataFrame(data, columns = ['class_dates', 'complete'])
>>>print(df)
                 class_dates  complete
0  8/25/2017 18:47:05.069722         1
1   9/1/2017 18:47:05.119994         0
2   9/8/2017 18:47:05.178768         1
3  9/15/2017 18:47:05.230071         1
4  9/22/2017 18:47:05.230071         0
5  9/29/2017 18:47:05.280592         1
\end{verbatim}

Convert df{[}`date'{]} from string to datetime

\begin{verbatim}
>>>import pandas as pd
>>>pd.to_datetime(df['class_dates'])
0   2017-08-25 18:47:05.069722
1   2017-09-01 18:47:05.119994
2   2017-09-08 18:47:05.178768
3   2017-09-15 18:47:05.230071
4   2017-09-22 18:47:05.230071
5   2017-09-29 18:47:05.280592
Name: class_dates, dtype: datetime64[ns]
\end{verbatim}

\section{Control Statements}\label{control-statements}

\subsection{Comparision}\label{comparision}

Computer programs do not only execute instructions. Occasionally, a
choice needs to be made. Such as a choice is based on a condition.
Python has several conditional operators:

\begin{verbatim}
>   greater than
<   smaller than
==  equals
!=  is not
\end{verbatim}

Conditions are always combined with variables. A program can make a
choice using the if keyword. For example:

\begin{verbatim}
>>> x = int(input("Guess x:"))
>>> if x == 4:
...    print('You guessed correctly!')
...    <ENTER>
\end{verbatim}

In this example, \emph{You guessed correctly!} will only be printed if
the variable x equals to four (see table above). Python can also execute
multiple conditions using the elif and else keywords.

\begin{verbatim}
>>> x = int(input("Guess x:"))
>>> if x == 4:
...     print('You guessed correctly!')
... elif abs(4 - x) == 1:
...     print('Wrong guess, but you are close!')
... else:
...     print('Wrong guess')
... <ENTER>
\end{verbatim}

\subsection{Iteration}\label{iteration}

To repeat code, the for keyword can be used. For example, to display the
numbers from 1 to 10, we could write something like this:

\begin{verbatim}
>>> for i in range(1, 11):
...    print('Hello!')
\end{verbatim}

The second argument to range, \emph{11}, is not inclusive, meaning that
the loop will only get to \emph{10} before it finishes. Python itself
starts counting from 0, so this code will also work:

\begin{verbatim}
>>> for i in range(0, 10):
...    print(i + 1)
\end{verbatim}

In fact, the range function defaults to starting value of \emph{0}, so
the above is equivalent to:

\begin{verbatim}
>>> for i in range(10):
...    print(i + 1)
\end{verbatim}

We can also nest loops inside each other:

\begin{verbatim}
>>> for i in range(0,10):
...     for j in range(0,10):
...         print(i,' ',j)
... <ENTER>
\end{verbatim}

In this case we have two nested loops. The code will iterate over the
entire coordinate range (0,0) to (9,9)

\section{Datatypes}\label{datatypes}

\subsection{Lists}\label{lists}

see: \url{https://www.tutorialspoint.com/python/python_lists.htm}

Lists in Python are ordered sequences of elements, where each element
can be accessed using a 0-based index.

To define a list, you simply list its elements between square brackest
`{[}{]}`:

\begin{verbatim}
>>> >>> names = ['Albert', 'Jane', 'Liz', 'John', 'Abby']
>>> names[0] # access the first element of the list
'Albert'
>>> names[2] # access the third element of the list
'Liz'
\end{verbatim}

You can also use a negative index if you want to start counting elements
from the end of the list. Thus, the last element has index \emph{-1},
the second before last element has index \emph{-2} and so on:

\begin{verbatim}
>>> names[-1] # access the last element of the list
'Abby'
>>> names[-2] # access the second last element of the list
'John'
\end{verbatim}

Python also allows you to take whole slices of the list by specifing a
beginning and end of the slice separated by a colon `::

::

  \textgreater{}\textgreater{}\textgreater{} names{[}1:-1{]} \# the middle elements, excluding first and last
  {[}'Jane', 'Liz', 'John'{]}

As you can see from the example above, the starting index in the slice
is inclusive and the ending one, exclusive.

Python provides a variety of methods for manipulating the members of a
list.

You can add elements with append`:

\begin{verbatim}
>>> names.append('Liz')
>>> names
['Albert', 'Jane', 'Liz', 'John', 'Abby', 'Liz']
\end{verbatim}

As you can see, the elements in a list need not be unique.

Merge two lists with `extend`:

\begin{verbatim}
>>> names.extend(['Lindsay', 'Connor'])
>>> names
['Albert', 'Jane', 'Liz', 'John', 'Abby', 'Liz', 'Lindsay', 'Connor']
\end{verbatim}

Find the index of the first occurrence of an element with `index`:

\begin{verbatim}
>>> names.index('Liz')
2
\end{verbatim}

Remove elements by value with `remove`:

\begin{verbatim}
>>> names.remove('Abby')
>>> names
['Albert', 'Jane', 'Liz', 'John', 'Liz', 'Lindsay', 'Connor']
\end{verbatim}

Remove elements by index with `pop`:

\begin{verbatim}
>>> names.pop(1)
'Jane'
>>> names
['Albert', 'Liz', 'John', 'Liz', 'Lindsay', 'Connor']
\end{verbatim}

Notice that pop returns the element being removed, while remove does
not.

If you are familiar with stacks from other programming languages, you
can use insert and `pop`:

\begin{verbatim}
>>> names.insert(0, 'Lincoln')
>>> names
['Lincoln', 'Albert', 'Liz', 'John', 'Liz', 'Lindsay', 'Connor']
>>> names.pop()
'Connor'
>>> names
['Lincoln', 'Albert', 'Liz', 'John', 'Liz', 'Lindsay']
\end{verbatim}

The Python documentation contains a \href{}{full list of list
operations}.

To go back to the range function you used earlier, it simply creates a
list of numbers:

\begin{verbatim}
>>> range(10)
[0, 1, 2, 3, 4, 5, 6, 7, 8, 9]
>>> range(2, 10, 2)
[2, 4, 6, 8]
\end{verbatim}

\subsection{Sets}\label{sets}

Python lists can contain duplicates as you saw above:

\begin{verbatim}
>>> names = ['Albert', 'Jane', 'Liz', 'John', 'Abby', 'Liz']
\end{verbatim}

When we don't want this to be the case, we can use a
\href{https://docs.python.org/2/library/stdtypes.html\#set}{set}:

\begin{verbatim}
>>> unique_names = set(names)
>>> unique_names
set(['Lincoln', 'John', 'Albert', 'Liz', 'Lindsay'])
\end{verbatim}

Keep in mind that the \emph{set} is an unordered collection of objects,
thus we can not access them by index:

\begin{verbatim}
>>> unique_names[0]
Traceback (most recent call last):
  File "<stdin>", line 1, in <module>
  TypeError: 'set' object does not support indexing
\end{verbatim}

However, we can convert a set to a list easily:

\textgreater{}\textgreater{}\textgreater{} unique\_names =
list(unique\_names) \textgreater{}\textgreater{}\textgreater{}
unique\_names {[}`Lincoln', `John', `Albert', `Liz', `Lindsay'{]}
\textgreater{}\textgreater{}\textgreater{} unique\_names{[}0{]}
`Lincoln'

Notice that in this case, the order of elements in the new list matches
the order in which the elements were displayed when we create the set
(we had set({[}'Lincoln', 'John', 'Albert', 'Liz',
'Lindsay'{]}) and now we have {[}'Lincoln', 'John', 'Albert', 'Liz',
'Lindsay'{]}). You should not assume this is the case in general. That
is, don't make any assumptions about the order of elements in a set when
it is converted to any type of sequential data structure.

You can change a set's contents using the add, remove and update methods
which correspond to the append, remove and extend methods in a list. In
addition to these, \emph{set} objects support the operations you may be
familiar with from mathematical sets: \emph{union}, \emph{intersection},
\emph{difference}, as well as operations to check containment. You can
read about this in the
\href{https://docs.python.org/2/library/stdtypes.html\#set}{Python
documentation for sets}.

\subsection{Removal and Testing for Membership in
Sets}\label{removal-and-testing-for-membership-in-sets}

One important advantage of a \emph{set} over a \emph{list} is that
\textbf{access to elements is fast}. If you are familiar with different
data structures from a Computer Science class, the Python list is
implemented by an array, while the set is implemented by a hash table.

We will demonstrate this with an example. Let's say we have a list and a
set of the same number of elements (approximately 100 thousand):

\begin{verbatim}
>>> import sys, random, timeit
>>> nums_set = set([random.randint(0, sys.maxint) for _ in range(10**5)])
>>> nums_list = list(nums_set)
>>> len(nums_set)
100000
\end{verbatim}

We will use the
\href{https://docs.python.org/2/library/timeit.html}{timeit} Python
module to time 100 operations that test for the existence of a member in
either the list or set:

\begin{verbatim}
>>> timeit.timeit('random.randint(0, sys.maxint) in nums', setup='import random; nums=%s' % str(nums_set), number=100)
0.0004038810729980469
>>> timeit.timeit('random.randint(0, sys.maxint) in nums', setup='import random; nums=%s' % str(nums_list), number=100)
0.3980541229248047
\end{verbatim}

The exact duration of the operations on your system will be different,
but the take away will be the same: searching for an element in a set is
orders of magnitude faster than in a list. This is important to keep in
mind when you work with large amounts of data.

\subsection{Dictionaries}\label{dictionaries}

One of the very important data structures in python is a dictionary also
referred to as \emph{dict}.

A dictionary represents a key value store:

\begin{verbatim}
>>> person = {'Name': 'Albert', 'Age': 100, 'Class': 'Scientist'}
>>> print("person['Name']: ", person['Name'])
person['Name']:  Albert
>>> print("person['Age']: ", person['Age'])
person['Age']:  100
\end{verbatim}

You can delete elements with the following commands:

\begin{verbatim}
>>> del person['Name'] # remove entry with key 'Name'
>>> person
{'Age': 100, 'Class': 'Scientist'}
>>> person.clear()     # remove all entries in dict
>>> person
{}
>>> del person         # delete entire dictionary
>>> person
Traceback (most recent call last):
  File "<stdin>", line 1, in <module>
  NameError: name 'person' is not defined
\end{verbatim}

You can iterate over a dict:

\begin{verbatim}
>>> person = {'Name': 'Albert', 'Age': 100, 'Class': 'Scientist'}
>>> for item in person:
...   print(item, person[item])
...   <ENTER>
Age 100
Name Albert
Class Scientist
\end{verbatim}

\subsection{Dictionary Keys and
Values}\label{dictionary-keys-and-values}

You can retrieve both the keys and values of a dictionary using the
keys() and values() methods of the dictionary, respectively:

\begin{verbatim}
>>> person.keys()
['Age', 'Name', 'Class']
>>> person.values()
[100, 'Albert', 'Scientist']
\end{verbatim}

Both methods return lists. Notice, however, that the order in which the
elements appear in the returned lists (Age, Name, Class) is different
from the order in which we listed the elements when we declared the
dictionary initially (Name, Age, Class). It is important to keep this in
mind: \textbf{you can't make any assumptions about the order in which
the elements of a dictionary will be returned by the keys() and values()
methods}.

However, you can assume that if you call keys() and values() in
sequence, the order of elements will at least correspond in both
methods. In the above example Age corresponds to 100, Name to 'Albert,
and Class to Scientist, and you will observe the same correspondence in
general as long as \textbf{keys() and values() are called one right
after the other}.

\subsection{Counting with
Dictionaries}\label{counting-with-dictionaries}

One application of dictionaries that frequently comes up is counting the
elements in a sequence. For example, say we have a sequence of coin
flips:

\begin{verbatim}
>>> import random
>>> die_rolls = [random.choice(['heads', 'tails']) for _ in range(10)]
>>> die_rolls
['heads', 'tails', 'heads', 'tails', 'heads', 'heads', 'tails', 'heads', 'heads', 'heads']
\end{verbatim}

The actual list die\_rolls will likely be different when you execute
this on your computer since the outcomes of the die rolls are random.

To compute the probabilities of heads and tails, we could count how many
heads and tails we have in the list:

\begin{verbatim}
>>> counts = {'heads': 0, 'tails': 0}
>>> for outcome in coin_flips:
...   assert outcome in counts
...   counts[outcome] += 1
...   <ENTER>
>>> print('Probability of heads: %.2f' % (counts['heads'] / len(coin_flips)))
Probability of heads: 0.70
>>> print('Probability of tails: %.2f' % (counts['tails'] / sum(counts.values())))
Probability of tails: 0.30
\end{verbatim}

In addition to how we use the dictionary counts to count the elements of
coin\_flips, notice a couple things about this example:

\begin{enumerate}
\tightlist
\item
  We used the assert outcome in counts statement. The assert statement
  in Python allows you to easily insert debugging statements in your
  code to help you discover errors more quickly. assert statements are
  executed whenever the internal Python \_\_debug\_\_ variable is set to
  True, which is always the case unless you start Python with the -O
  option which allows you to run \emph{optimized} Python.
\item
  When we computed the probability of tails, we used the built-in sum
  function, which allowed us to quickly find the total number of coin
  flips. sum is one of many built-in function you can
  \href{https://docs.python.org/2/library/functions.html}{read about
  here}.
\end{enumerate}

\section{Functions}\label{functions}

You can reuse code by putting it inside a function that you can call in
other parts of your programs. Functions are also a good way of grouping
code that logically belongs together in one coherent whole. A function
has a unique name in the program. Once you call a function, it will
execute its body which consists of one or more lines of code:

\begin{verbatim}
def check_triangle(a, b, c):
return \
    a < b + c and a > abs(b - c) and \
    b < a + c and b > abs(a - c) and \
    c < a + b and c > abs(a - b)

print(check_triangle(4, 5, 6))
\end{verbatim}

The def keyword tells Python we are defining a function. As part of the
definition, we have the function name, check\_triangle, and the
parameters of the function -- variables that will be populated when the
function is called.

We call the function with arguments 4, 5 and 6, which are passed in
order into the parameters a, b and c. A function can be called several
times with varying parameters. There is no limit to the number of
function calls.

It is also possible to store the output of a function in a variable, so
it can be reused.

\begin{verbatim}
def check_triangle(a, b, c):
 return \
     a < b + c and a > abs(b - c) and \
     b < a + c and b > abs(a - c) and \
     c < a + b and c > abs(a - b)

result = check_triangle(4, 5, 6)
print(result)
\end{verbatim}

\section{Classes}\label{classes}

A class is an encapsulation of data and the processes that work on them.
The data is represented in member variables, and the processes are
defined in the methods of the class (methods are functions inside the
class). For example, let's see how to define a Triangle class:

\begin{verbatim}
class Triangle(object):

 def __init__(self, length, width, height, angle1, angle2, angle3):
     if not self._sides_ok(length, width, height):
         print('The sides of the triangle are invalid.')
     elif not self._angles_ok(angle1, angle2, angle3):
         print('The angles of the triangle are invalid.')

     self._length = length
     self._width = width
     self._height = height

     self._angle1 = angle1
     self._angle2 = angle2
     self._angle3 = angle3

 def _sides_ok(self, a, b, c):
     return \
         a < b + c and a > abs(b - c) and \
         b < a + c and b > abs(a - c) and \
         c < a + b and c > abs(a - b)

 def _angles_ok(self, a, b, c):
     return a + b + c == 180

triangle = Triangle(4, 5, 6, 35, 65, 80)
\end{verbatim}

Python has full Aobject-oriented programming (OOP) capabilities, however
we can not cover all of them in a quick tutorial, so please refer to the
\href{https://docs.python.org/2.7/tutorial/classes.html}{Python docs on
classes and OOP}.

\section{Database Access}\label{database-access}

see:
\url{https://www.tutorialspoint.com/python/python_database_access.htm}

\section{Modules}\label{modules}

Make sure you are no longer in the interactive interpreter. If you are
you can type quit() and press Enter to exit.

You can save your programs to files which the interpreter can then
execute. This has the benefit of allowing you to track changes made to
your programs and sharing them with other people.

Start by opening a new file hello.py in the Python editor of your
choice. If you don't have a preferred editor, we recommend
\href{https://www.jetbrains.com/pycharm/}{PyCharm}.

Now write this simple program and save it:

\begin{verbatim}
from __future__ import print_statement, division
print("Hello world!")
\end{verbatim}

As a check, make sure the file contains the expected contents on the
command line:

\begin{verbatim}
$ cat hello.py
from __future__ import print_statement, division
print("Hello world!")
\end{verbatim}

To execute your program pass the file as a parameter to the python
command:

\begin{verbatim}
$ python hello.py
Hello world!
\end{verbatim}

Files in which Python code is stored are called \textbf{module}s. You
can execute a Python module form the command line like you just did, or
you can import it in other Python code using the import statement.

Let's write a more involved Python program that will receive as input
the lengths of the three sides of a triangle, and will output whether
they define a valid triangle. A triangle is valid if the length of each
side is less than the sum of the lengths of the other two sides and
greater than the difference of the lengths of the other two sides.:

\begin{verbatim}
"""Usage: check_triangle.py [-h] LENGTH WIDTH HEIGHT

Check if a triangle is valid.

Arguments:
  LENGTH     The length of the triangle.
  WIDTH      The width of the traingle.
  HEIGHT     The height of the triangle.

Options:
-h --help
"""
from __future__ import print_function, division
from docopt import docopt

if __name__ == '__main__':
  args = docopt(__doc__)
  a, b, c = int(args['LENGTH']), int(args['WIDTH']), int(args['HEIGHT'])
  valid_triangle = \
      a < b + c and a > abs(b - c) and \
      b < a + c and b > abs(a - c) and \
      c < a + b and c > abs(a - b)
  print('Triangle with sides %d, %d and %d is valid: %r' % (
      a, b, c, valid_triangle
  ))
\end{verbatim}

Assuming we save the program in a file called check\_triangle.py, we can
run it like so:

\begin{verbatim}
$ python check_triangle.py 4 5 6
Triangle with sides 4, 5 and 6 is valid: True
\end{verbatim}

Let break this down a bit.

\begin{enumerate}
\tightlist
\item
  We are importing the print\_function and division modules from Python
  3 like we did earlier in this tutorial. It's a good idea to always
  include these in your programs.
\item
  We've defined a boolean expression that tells us if the sides that
  were input define a valid triangle. The result of the expression is
  stored in the valid\_triangle variable. inside are true, and False
  otherwise.
\item
  We've used the backslash symbol \textbackslash{} to format are code
  nicely. The backslash simply indicates that the current line is being
  continued on the next line.
\item
  When we run the program, we do the check if \_\_name\_\_ ==
  '\_\_main\_\_'. \_\_name\_\_ is an internal Python variable that
  allows us to tell whether the current file is being run from the
  command line (value \_\_name\_\_), or is being imported by a module
  (the value will be the name of the module). Thus, with this statement
  we're just making sure the program is being run by the command line.
\item
  We are using the docopt module to handle command line arguments. The
  advantage of using this module is that it generates a usage help
  statement for the program and enforces command line arguments
  automatically. All of this is done by parsing the docstring at the top
  of the file.
\item
  In the print function, we are using
  \href{https://docs.python.org/2/library/string.html\#format-string-syntax}{Python's
  string formatting capabilities} to insert values into the string we
  are displaying.
\end{enumerate}

\section{Installing Libraries}\label{installing-libraries}

Often you may need functionality that is not present in Python's
standard library. In this case you have two option:

\begin{itemize}
\tightlist
\item
  implement the features yourself
\item
  use a third-party library that has the desired features.
\end{itemize}

Often you can find a previous implementation of what you need. Since
this is a common situation, there is a service supporting it: the
\href{https://pypi.python.org/pypi}{Python Package Index} (or PyPi for
short).

Our task here is to install the \href{}{autopep8} tool from PyPi. This
will allow us to illustrate the use if virtual environments using the
pyenv or virtualenv command, and installing and uninstalling PyPi
packages using pip.

\section{Using pip to Install
Packages}\label{using-pip-to-install-packages}

Let's now look at another important tool for Python development: the
Python Package Index, or PyPI for short. PyPI provides a large set of
third-party python packages. If you want to do something in python,
first check pypi, as odd are someone already ran into the problem and
created a package solving it.

In order to install package from PyPI, use the pip command. We can
search for PyPI for packages:

\begin{verbatim}
$ pip search --trusted-host pypi.python.org autopep8 pylint
\end{verbatim}

It appears that the top two results are what we want so install them:

\begin{verbatim}
$ pip install --trusted-host pypi.python.org autopep8 pylint
\end{verbatim}

This will cause pip to download the packages from PyPI, extract them,
check their dependencies and install those as needed, then install the
requested packages.

\begin{description}
\item[You can skip `--trusted-host pypi.python.org' option if you have]
patched urllib3 on Python 2.7.9.
\end{description}

\section{GUI}\label{gui}

\subsection{GUIZero}\label{guizero}

Install guizero with the following command:

\begin{verbatim}
sudo pip3 install guizero
\end{verbatim}

For a comprehensive tutorial on guizero,
\href{https://lawsie.github.io/guizero/howto/}{click here}.

\subsection{Kivy}\label{kivy}

You can install Kivy on OSX as followes:

\begin{verbatim}
brew install pkg-config sdl2 sdl2_image sdl2_ttf sdl2_mixer gstreamer
pip install -U Cython
pip install kivy
pip install pygame
\end{verbatim}

A hello world program for kivy is included in the cloudmesh.robot
repository. Which you can fine here

\begin{itemize}
\tightlist
\item
  \url{https://github.com/cloudmesh/cloudmesh.robot/tree/master/projects/kivy}
\end{itemize}

To run the program, please download it or execute it in cloudmesh.robot
as follows:

\begin{verbatim}
cd cloudmesh.robot/projects/kivy
python swim.py
\end{verbatim}

To create stand alone packages with kivy, please see:

\begin{verbatim}
-  https://kivy.org/docs/guide/packaging-osx.html
\end{verbatim}

\section{Formatting and Checking Python
Code}\label{formatting-and-checking-python-code}

First, get the bad code:

\begin{verbatim}
$ wget --no-check-certificate http://git.io/pXqb -O bad_code_example.py
\end{verbatim}

Examine the code:

\begin{verbatim}
$ emacs bad_code_example.py
\end{verbatim}

As you can see, this is very dense and hard to read. Cleaning it up by
hand would be a time-consuming and error-prone process. Luckily, this is
a common problem so there exist a couple packages to help in this
situation.

\section{Using autopep8}\label{using-autopep8}

We can now run the bad code through autopep8 to fix formatting problems:

\begin{verbatim}
$ autopep8 bad_code_example.py >code_example_autopep8.py
\end{verbatim}

Let us look at the result. This is considerably better than before. It
is easy to tell what the example1 and example2 functions are doing.

It is a good idea to develop a habit of using autopep8 in your
python-development workflow. For instance: use autopep8 to check a file,
and if it passes, make any changes in place using the -i flag:

\begin{verbatim}
$ autopep8 file.py    # check output to see of passes
$ autopep8 -i file.py # update in place
\end{verbatim}

If you use pyCharm you have the ability to use a similar function while
p;ressing on Inspect Code.

\section{Further Learning}\label{further-learning}

There is much more to python than what we have covered here:

\begin{itemize}
\tightlist
\item
  conditional expression (if, if...then,`if..elif..then`)
\item
  function definition(def)
\item
  class definition (class)
\item
  function positional arguments and keyword arguments
\item
  lambda expression
\item
  iterators
\item
  generators
\item
  loops
\item
  docopts
\item
  humanize
\end{itemize}

\section{Writing Python 3 Compatible
Code}\label{writing-python-3-compatible-code}

To write python 2 and 3 compatib;e code we recommend that you take a
look at: \url{http://python-future.org/compatible_idioms.html}

\section{Using Python on
FutureSystems}\label{using-python-on-futuresystems}

This is only important if you use Futuresystems resources.

In order to use Python you must log into your FutureSystems account.
Then at the shell prompt execute the following command:

\begin{verbatim}
$ module load python
\end{verbatim}

This will make the python and virtualenv commands available to you.

The details of what the module load command does are described in the
future lesson modules.

\section{Ecosystem}\label{ecosystem}

\subsection{pypi}\label{pypi}

Link: \href{https://pypi.python.org/pypi}{pypi}

The Python Package Index is a large repository of software for the
Python programming language containing a large number of packages
{[}link{]}. The nice think about pipy is that many packages can be
installed with the program `pip'.

To do so you have to locate the \textless{}package\_name\textgreater{}
for example with the search function in pypi and say on the commandline:

\begin{verbatim}
pip install <package_name>
\end{verbatim}

where pagage\_name is the string name of the package. an example would
be the package called cloudmesh\_client which you can install with:

\begin{verbatim}
pip install cloudmesh_client
\end{verbatim}

If all goes well the package will be installed.

\subsection{Alternative Installations}\label{alternative-installations}

The basic installation of python is provided by python.org. However
others claim to have alternative environments that allow you to install
python. This includes

\begin{itemize}
\tightlist
\item
  \href{https://store.enthought.com/downloads/\#default}{Canopy}
\item
  \href{https://www.continuum.io/downloads}{Anaconda}
\item
  \href{http://ironpython.net/}{IronPython}
\end{itemize}

Typically they include not only the python compiler but also several
useful packages. It is fine to use such environments for the class, but
it should be noted that in both cases not every python library may be
available for install in the given environment. For example if you need
to use cloudmesh client, it may not be available as conda or Canopy
package. This is also the case for many other cloud related and useful
python libraries. Hence, we do recommend that if you are new to python
to use the distribution form python.org, and use pip and virtualenv.

Additionally some python version have platform specific libraries or
dependencies. For example coca libraries, .NET or other frameworks are
examples. For the assignments and the projects such platform dependent
libraries are not to be used.

If however you can write a platform independent code that works on
Linux, OSX and Windows while using the python.org version but develop it
with any of the other tools that is just fine. However it is up to you
to guarantee that this independence is maintained and implemented. You
do have to write requirements.txt files that will install the necessary
python libraries in a platform independent fashion. The homework
assignment PRG1 has even a requirement to do so.

In order to provide platform independence we have given in the class a
``minimal'' python version that we have tested with hundreds of
students: python.org. If you use any other version, that is your
decision. Additionally some students not only use python.org but have
used iPython which is fine too. However this class is not only about
python, but also about how to have your code run on any platform. The
homework is designed so that you can identify a setup that works for
you.

However we have concerns if you for example wanted to use chameleon
cloud which we require you to access with cloudmesh. cloudmesh is not
available as conda, canopy, or other framework package. Cloudmesh client
is available form pypi which is standard and should be supported by the
frameworks. We have not tested cloudmesh on any other python version
then python.org which is the open source community standard. None of the
other versions are standard.

In fact we had students over the summer using canopy on their machines
and they got confused as they now had multiple python versions and did
not know how to switch between them and activate the correct version.
Certainly if you know how to do that, than feel free to use canopy, and
if you want to use canopy all this is up to you. However the homework
and project requires you to make your program portable to python.org. If
you know how to do that even if you use canopy, anaconda, or any other
python version that is fine. Graders will test your programs on a
python.org installation and not canpoy, anaconda, ironpython while using
virtualenv. It is obvious why. If you do not know that answer you may
want to think about that every time they test a program they need to do
a new virtualenv and run vanilla python in it. If we were to run two
instals in the same system, this will not work as we do not know if one
student will cause a side effect for another. Thus we as instructors do
not just have to look at your code but code of hundreds of students with
different setups. This is a non scalable solution as every time we test
out code from a student we would have to wipe out the OS, install it
new, install an new version of whatever python you have elected, become
familiar with that version and so on and on. This is the reason why the
open source community is using python.org. We follow best practices.
Using other versions is not a community best practice, but may work for
an individual.

We have however in regards to using other python version additional
bonus projects such as

\begin{itemize}
\tightlist
\item
  deploy run and document cloudmesh on ironpython
\item
  deploy run and document cloudmesh on anaconde, develop script to
  generate a conda packge form github
\item
  deploy run and document cloudmesh on canopy, develop script to
  generate a conda packge form github
\item
  deploy run and document cloudmesh on ironpython
\item
  other documentation that would be useful
\end{itemize}

\subsection{Autoenv: Directory-based
Environments}\label{autoenv-directory-based-environments}

\begin{description}
\item[We do not recommend that you use autoenv. Instead we]
recommend that you use pyenv.
\end{description}

Link:
Autoenv \textless{}https://pypi.python.org/pypi/autoenv/0.2.0\textgreater{}

If a directory contains a .env file, it will automatically be executed
when you cd into it. It's easy to use and install.

This is useful for

\begin{itemize}
\tightlist
\item
  auto-activating virtualenvs
\item
  project-specific environment variables
\end{itemize}

To use it add the ENV you created with virtualenv into .env file within
your project directory:

\begin{verbatim}
$ echo "source ~/ENV/bin/activate" > yourproject/.env
$ echo "echo 'whoa'" > yourproject/.env
$ cd project
whoa
\end{verbatim}

To install it on Mac OS X use Homebrew:

\begin{verbatim}
$ brew install autoenv
$ echo "source $(brew --prefix autoenv)/activate.sh" >> ~/.bash_profile
\end{verbatim}

To install it using pip use:

\begin{verbatim}
$ pip install autoenv
$ echo "source `which activate.sh`" >> ~/.bashrc
\end{verbatim}

To install it using git use:

\begin{verbatim}
$ git clone git://github.com/kennethreitz/autoenv.git ~/.autoenv
$ echo 'source ~/.autoenv/activate.sh' >> ~/.bashrc
\end{verbatim}

Before sourcing activate.sh, you can set the following variables:

\begin{itemize}
\tightlist
\item
  `AUTOENV\_AUTH\_FILE`: Authorized env files, defaults to
  \textasciitilde{}/.autoenv\_authorized
\item
  `AUTOENV\_ENV\_FILENAME`: Name of the .env file, defaults to .env
\item
  `AUTOENV\_LOWER\_FIRST`: Set this variable to flip the order of .env
  files executed
\end{itemize}

Autoenv overrides cd. If you already do this, invoke autoenv\_init
within your custom cd after sourcing activate.sh.

\begin{description}
\item[Autoenv can be disabled via unset cd if you experience I/O issues]
with certain file systems, particularly those that are FUSE-based (such
as smbnetfs).
\end{description}

\section{Resources}\label{resources}

If you are unfamiliar with programming in Python, we also refer you to
some of the numerous online resources. You may wish to start with
\href{https://www.learnpython.org}{Learn Python} or the book
\href{http://learnpythonthehardway.org/book/}{Learn Python the Hard
Way}. Other options include
\href{http://www.tutorialspoint.com/python/}{Tutorials Point} or
\href{http://www.codecademy.com/en/tracks/python}{Code Academy}, and the
Python wiki page contains a long list of
\href{https://wiki.python.org/moin/BeginnersGuide/Programmers}{references
for learning} as well. Additional resources include:

\begin{itemize}
\tightlist
\item
  \url{https://virtualenvwrapper.readthedocs.io}
\item
  \url{https://github.com/yyuu/pyenv}
\item
  \url{https://amaral.northwestern.edu/resources/guides/pyenv-tutorial}
\item
  \url{https://godjango.com/96-django-and-python-3-how-to-setup-pyenv-for-multiple-pythons/}
\item
  \url{https://www.accelebrate.com/blog/the-many-faces-of-python-and-how-to-manage-them/}
\item
  \url{http://ivory.idyll.org/articles/advanced-swc/}
\item
  \url{http://python.net/~goodger/projects/pycon/2007/idiomatic/handout.html}
\item
  \url{http://www.youtube.com/watch?v=0vJJlVBVTFg}
\item
  \url{http://www.korokithakis.net/tutorials/python/}
\item
  \url{http://www.afterhoursprogramming.com/tutorial/Python/Introduction/}
\item
  \url{http://www.greenteapress.com/thinkpython/thinkCSpy.pdf}
\item
  \url{https://docs.python.org/3.3/tutorial/modules.html}
\item
  \url{https://www.learnpython.org/en/Modules/_and/_Packages}
\item
  \url{https://docs.python.org/2/library/datetime.html}
\item
  \url{https://chrisalbon.com/python/strings/_to/_datetime.html}
\end{itemize}

A very long list of useful information are also available from

\begin{itemize}
\tightlist
\item
  \url{https://github.com/vinta/awesome-python}
\item
  \url{https://github.com/rasbt/python_reference}
\end{itemize}

This list may be useful as it also contains links to data visualization
and manipulation libraries, and AI tools and libraries. Please note that
for this class you can reuse such libraries if not otherwise stated.

\section{Jupyter Notebook Tutorials}\label{jupyter-notebook-tutorials}

A Short Introduction to Jupyter Notebooks and NumPy To view the
notebook, open this link in a background tab
\textless{}\url{https://nbviewer.jupyter.org/}\textgreater{} and copy
and paste the following link in the URL input area
\textless{}\url{https://cloudmesh.github.io/classes/lesson/prg/Jupyter-NumPy-tutorial-I523-F2017.ipynb}\textgreater{}
Then hit Go!

\section{Exercises}\label{exercises}

\begin{description}
\item[EPython.1:]
Write a python program called iterate.py that accepts an integer n from
the command line. Pass this integer to a function called iterate.

The iterate function should then iterate from 1 to n. If the ith number
is a multiple of three, print ``multiple of 3'', if a multiple of 5
print ``multiple of 5'', if a multiple of both print ``multiple of 3 and
5'', else print the value.
\item[EPython.2:]
\begin{enumerate}
\tightlist
\item
  Create a pyenv or virtualenv \textasciitilde{}/ENV
\item
  Modify your \textasciitilde{}/.bashrc shell file to activate your
  environment upon login.
\item
  Install the docopt python package using pip
\item
  Write a program that uses docopt to define a commandline program.
  Hint: modify the iterate program.
\item
  Demonstrate the program works and submit the code and output.
\end{enumerate}
\end{description}
