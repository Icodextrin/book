\begin{fileremark}i523/2017/course/web.rst\end{fileremark}
\section{Web Search and Text Mining}\label{web-search-and-text-mining}

This section starts with an overview of data mining and puts our study
of classification, clustering and exploration methods in context. We
examine the problem to be solved in web and text search and note the
relevance of history with libraries, catalogs and concordances. An
overview of web search is given describing the continued evolution of
search engines and the relation to the field of Information.

The importance of recall, precision and diversity is discussed. The
important Bag of Words model is introduced and both Boolean queries and
the more general fuzzy indices. The important vector space model and
revisiting the Cosine Similarity as a distance in this bag follows. The
basic TF-IDF approach is dis cussed. Relevance is discussed with a
probabilistic model while the distinction between Bayesian and frequency
views of probability distribution completes this unit.

We start with an overview of the different steps (data analytics) in web
search and then goes key steps in detail starting with document
preparation. An inverted index is described and then how it is prepared
for web search. The Boolean and Vector Space approach to query
processing follow. This is followed by Link Structure Analysis including
Hubs, Authorities and PageRank. The application of PageRank ideas as
reputation outside web search is covered. The web graph structure,
crawling it and issues in web advertising and search follow. The use of
clustering and topic models completes the section.

\subsection{Web Search and Text Mining I}\label{web-search-and-text-mining-i}

The unit starts with the web with its size, shape (coming from the
mutual linkage of pages by URL's) and universal power laws for number of
pages with particular number of URL's linking out or in to page.
Information retrieval is introduced and compared to web search. A
comparison is given between semantic searches as in databases and the
full text search that is base of Web search. The origin of web search in
libraries, catalogs and concordances is summarized. DIKW -- Data
Information Knowledge Wisdom -- model for web search is discussed. Then
features of documents, collections and the important Bag of Words
representation. Queries are presented in context of an Information
Retrieval architecture. The method of judging quality of results
including recall, precision and diversity is described. A time line for
evolution of search engines is given.

Boolean and Vector Space models for query including the cosine
similarity are introduced. Web Crawlers are discussed and then the steps
needed to analyze data from Web and produce a set of terms. Building and
accessing an inverted index is followed by the importance of term
specificity and how it is captured in TF-IDF. We note how frequencies
are converted into belief and relevance.

\slides{Web Search and Text Mining}{56}{https://drive.google.com/open?id=0B8936_ytjfjmeWVSYk9RVXcyOFk}

\subsection{Web and Document/Text Search: The
Problem}\label{web-and-documenttext-search-the-problem}

\video{9:56}{Text Mining}{https://www.youtube.com/watch?v=RFBeAWBkUsI}

This lesson starts with the web with its size, shape (coming from the
mutual linkage of pages by URL's) and universal power laws for number of
pages with particular number of URL's linking out or in to page.



\subsection{Information Retrieval leading to Web
Search}\label{information-retrieval-leading-to-web-search}

\video{6:06}{Information Retrival}{https://youtu.be/KtWhk2cdRa4}

Information retrieval is introduced A comparison is given between
semantic searches as in databases and the full text search that is base
of Web search. The ACM classification illustrates potential complexity
of ontologies. Some differences between web search and information
retrieval are given.



\subsection{History behind Web Search}\label{history-behind-web-search}

\video{5:48}{Web Search History}{https://youtu.be/J7D61uH5gVM}

The origin of web search in libraries, catalogs and concordances is
summarized.




\subsection{Key Fundamental Principles behind Web
Search}\label{key-fundamental-principles-behind-web-search}

\video{9:30}{Principles}{https://youtu.be/yPFi6xFnDHE}

This lesson describes the DIKW -- Data Information Knowledge Wisdom --
model for web search. Then it discusses documents, collections and the
important Bag of Words representation.



\subsection{Information Retrieval (Web Search)
Components}\label{information-retrieval-web-search-components}

\video{5:06}{Fundametal Principals of Web
  Search}{https://youtu.be/EGsnonXgb3Y}

This describes queries in context of an Information Retrieval
architecture. The method of judging quality of results including recall,
precision and diversity is described.



\subsection{Search Engines}\label{search-engines}

\video{3:08}{Search Engines}{https://youtu.be/kBV-99N6f7k}

This short lesson describes a time line for evolution of search engines.
The first web search approaches were directly built on Information
retrieval but in 1998 the field was changed when Google was founded and
showed the importance of URL structure as exemplified by PageRank.



\subsection{Boolean and Vector Space
Models}\label{boolean-and-vector-space-models}

\video{6:17}{Boolean and Vector Space
  Model}{https://youtu.be/JzGBA0OhsIk}

This lesson describes the Boolean and Vector Space models for query
including the cosine similarity.



\subsection{Web crawling and Document
Preparation}\label{web-crawling-and-document-preparation}

\video{4:55}{Web crawling and Document
  Preparation}{https://youtu.be/Wv-r-PJ9lro}

This describes a Web Crawler and then the steps needed to analyze data
from Web and produce a set of terms.



\subsection{Indices}\label{indices}

\video{5:44}{Indices}{https://youtu.be/NY2SmrHoBVM}

This lesson describes both building and accessing an inverted index. It
describes how phrases are treated and gives details of query structure
from some early logs.



\subsection{TF-IDF and Probabilistic
Models}\label{tf-idf-and-probabilistic-models}

\video{3:57}{TF-IDF and Probabilistic
  Models}{https://youtu.be/9P_HUmpselU}

It describes the importance of term specificity and how it is captured
in TF-IDF. It notes how frequencies are converted into belief and
relevance.



\subsection{Resources}\label{resources}

\begin{itemize}
\tightlist
\item
  \url{http://saedsayad.com/data_mining_map.htm}
\item
  \url{http://webcourse.cs.technion.ac.il/236621/Winter2011-2012/en/ho_Lectures.html}
\item
  The Web Graph: an
  Overviews://www.youtube.com/watch?v=yPFi6xFnDHE\&feature=youtu.be
  Jean-Loup Guillaume and Matthieu Latapy
  \url{https://hal.archives-ouvertes.fr/file/index/docid/54458/filename/webgraph.pdf}
\item
  Constructing a reliable Web graph with information on browsing
  behavior, Yiqun Liu, Yufei Xue, Danqing Xu, Rongwei Cen, Min Zhang,
  Shaoping Ma, Liyun Ru
  \url{http://www.sciencedirect.com/science/article/pii/S0167923612001844}
\item
  \url{http://www.ifis.cs.tu-bs.de/teaching/ss-11/irws}
\end{itemize}

\subsection{Web Search and Text Mining
II}\label{web-search-and-text-mining-ii}

\slides{Text
  Mining}{33}{https://drive.google.com/open?id=0B6wqDMIyK2P7YmpLbzQ0X2xpbDg}{PDF}

We start with an overview of the different steps (data analytics) in web
search. This is followed by Link Structure Analysis including Hubs,
Authorities and PageRank. The application of PageRank ideas as
reputation outside web search is covered. Issues in web advertising and
search follow. his leads to emerging field of computational advertising.
The use of clustering and topic models completes unit with Google News
as an example.



\subsection{Data Analytics for Web Search}\label{data-analytics-for-web-search}

\video{6:11}{Web Search and Text Mining
  II}{https://www.youtube.com/watch?v=kHEFxhWwhx0}

This short lesson describes the different steps needed in web search
including: Get the digital data (from web or from scanning); Crawl web;
Preprocess data to get searchable things (words, positions); Form
Inverted Index mapping words to documents; Rank relevance of documents
with potentially sophisticated techniques; and integrate technology to
support advertising and ways to allow or stop pages artificially
enhancing relevance.




\subsection{Link Structure Analysis including PageRank}\label{link-structure-analysis-including-pagerank}

\video{17:24}{Realated
  Applications}{https://www.youtube.com/watch?v=ApDu-7_1LYk}

The value of links and the concepts of Hubs and Authorities are
discussed. This leads to definition of PageRank with examples.
Extensions of PageRank viewed as a reputation are discussed with journal
rankings and university department rankings as examples. There are many
extension of these ideas which are not discussed here although topic
models are covered briefly in a later lesson.



\subsection{Web Advertising and
Search}\label{web-advertising-and-search}

\video{9:02}{Web Advertising and
  Search}{https://www.youtube.com/watch?v=375sY1YMk5U}

Internet and mobile advertising is growing fast and can be personalized
more than for traditional media. There are several advertising types
Sponsored search, Contextual ads, Display ads and different models: Cost
per viewing, cost per clicking and cost per action. This leads to
emerging field of computational advertising.




\subsection{Clustering and Topic Models}\label{clustering-and-topic-models}

\video{6:21}{Clustering and Topic
  Models}{https://youtu.be/95cHMyZ-TUs}

We discuss briefly approaches to defining groups of documents. We
illustrate this for Google News and give an example that this can give
different answers from word-based analyses. We mention some work at
Indiana University on a Latent Semantic Indexing model.



\subsection{Resources}\label{resources-1}

\begin{itemize}
\item
  \url{http://www.ifis.cs.tu-bs.de/teaching/ss-11/irws}
\item
  \url{https://en.wikipedia.org/wiki/PageRank}
\item
  \url{http://webcourse.cs.technion.ac.il/236621/Winter2011-2012/en/ho_Lectures.html}
\item
  Meeker/Wu May 29 2013 Internet Trends D11 Conference
  \url{http://www.slideshare.net/kleinerperkins/kpcb-internet-trends-2013}
\end{itemize}
