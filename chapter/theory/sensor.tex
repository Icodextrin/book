\begin{fileremark}\url{https://github.com/cloudmesh/classes/blob/master/docs/source/i523/2017/course/sensor.rst}\end{fileremark}
\section{Sensors Case Study}\label{sensors-case-study}

We start with the Internet of Things IoT giving examples like monitors
of machine operation, QR codes, surveillance cameras, scientific
sensors, drones and self driving cars and more generally transportation
systems. We give examples of robots and drones. We introduce the
Industrial Internet of Things IIoT and summarize surveys and
expectations Industry wide. We give examples from General Electric.
Sensor clouds control the many small distributed devices of IoT and
IIoT. More detail is given for radar data gathered by sensors;
ubiquitous or smart cities and homes including U-Korea; and finally the
smart electric grid.

\slides{Sensor I}{31}{https://drive.google.com/open?id=0B8936_ytjfjmVXZCUnR3TnVMMFk}
\slides{Sensor II}{44}{https://drive.google.com/open?id=0B8936_ytjfjmelMwSUl6Q1lLV1k}


\subsection{Internet of Things}\label{internet-of-things}

There are predicted to be 24-50 Billion devices on the Internet by 2020;
these are typically some sort of sensor defined as any source or sink of
time series data. Sensors include smartphones, webcams, monitors of
machine operation, barcodes, surveillance cameras, scientific sensors
(especially in earth and environmental science), drones and self driving
cars and more generally transportation systems. The lesson gives many
examples of distributed sensors, which form a Grid that is controlled by
a cloud.

\video{12:36}{Internet of Things}{https://www.youtube.com/watch?v=0O0-mz-CWtQ} 


\subsection{Robotics and IOT
Expectations}\label{robotics-and-iot-expectations}

Examples of Robots and Drones.

\video {8:05}{Robotics and IoT Expectations}{https://www.youtube.com/watch?v=ABP0Yygw2Zg}


\subsection{Industrial Internet of
Things}\label{industrial-internet-of-things}

We summarize surveys and expectations Industry wide.

\video{1:24:02}{Industrial Internet of Things}{https://www.youtube.com/watch?v=kxKzBfd62Og}


\subsection{Sensor Clouds}\label{sensor-clouds}

We describe the architecture of a Sensor Cloud control environment and
gives example of interface to an older version of it. The performance of
system is measured in terms of processing latency as a function of
number of involved sensors with each delivering data at 1.8 Mbps rate.

\video{4:40}{Sensor Clouds}{https://youtu.be/0egT1FsVGrU}


\subsection{Earth/Environment/Polar Science data gathered by
Sensors}\label{earthenvironmentpolar-science-data-gathered-by-sensors}

This lesson gives examples of some sensors in the
Earth/Environment/Polar Science field. It starts with material from the
CReSIS polar remote sensing project and then looks at the NSF Ocean
Observing Initiative and NASA's MODIS or Moderate Resolution Imaging
Spectroradiometer instrument on a satellite.

\video{4:58}{Earth/Environment/Polar Science data gathered by Sensors}{https://youtu.be/CS2gX7axWfI}


\subsection{Ubiquitous/Smart Cities}\label{ubiquitoussmart-cities}

For Ubiquitous/Smart cities we give two examples: Iniquitous Korea and
smart electrical grids.

\video{1:44}{Ubiquitous/Smart Cities}{https://youtu.be/MFFIItQ3SOo}


\subsection{U-Korea (U=Ubiquitous)}\label{u-korea-uubiquitous}

Korea has an interesting positioning where it is first worldwide in
broadband access per capita, e-government, scientific literacy and total
working hours. However it is far down in measures like quality of life
and GDP. U-Korea aims to improve the latter by Pervasive computing,
everywhere, anytime i.e. by spreading sensors everywhere. The example of
a `High-Tech Utopia' New Songdo is given.

\video{2:49}{U-Korea (U=Ubiquitous)}{https://www.youtube.com/watch?v=U38zWbSI2n4} 


\subsection{Smart Grid}\label{smart-grid}

The electrical Smart Grid aims to enhance USA's aging electrical
infrastructure by pervasive deployment of sensors and the integration of
their measurement in a cloud or equivalent server infrastructure. A
variety of new instruments include smart meters, power monitors, and
measures of solar irradiance, wind speed, and temperature. One goal is
autonomous local power units where good use is made of waste heat.

\video{6:04}{Smart Grid}{https://www.youtube.com/watch?v=UfEiIzaZzI8} 



\subsection{Resources}\label{resources}

\begin{itemize}
\item
  \url{https://www.gesoftware.com/minds-and-machines}
\item
  \url{https://www.gesoftware.com/predix}
\item
  \url{https://www.gesoftware.com/sites/default/files/the-industrial-internet/index.html}
\item
  \url{https://developer.cisco.com/site/eiot/discover/overview/}
\item
  \url{http://www.accenture.com/SiteCollectionDocuments/PDF/Accenture-Industrial-Internet-Changing-Competitive-Landscape-Industries.pdf}
\item
  \url{http://www.gesoftware.com/ge-predictivity-infographic}
\item
  \url{http://www.getransportation.com/railconnect360/rail-landscape}
\item
  \url{http://www.gesoftware.com/sites/default/files/GE-Software-Modernizing-Machine-to-Machine-Interactions.pdf}
\end{itemize}
