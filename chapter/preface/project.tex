\begin{fileremark}\url{https://github.com/cloudmesh/classes/blob/master/docs/source/i523/2017/project.rst}\end{fileremark}
\section{Term Paper or Project}\label{term-paper-or-project}

You have a choice to write a term paper or do a software project. This
will constitute to \textbf{60\%} of your class grade. You have plenty of
time to make this choice and if you find you struggle with programming
you may want to consider a term paper instead of a project.

In case you chose a project your maximum grade for the entire class
could be an A+. However, an A+ project must be truly outstanding and
include an exceptional project report. Such a project and report will
have the potential quality of being able to be published in a
conference.

In case you chose a Term Paper your maximum grade for the entire class
will be an A-.

Please note that a project includes also writing a project report/paper.
However the length is a bit lower than for a term paper.

\subsection{Team}\label{team}

Software projects and term papers can be conducted with one, two or
three class members. We do not allow more than three members in a
project. It will be up to you to determine a team, but we recommend that
you chose wisely. Naturally if a team member does not contribute to the
project you need to address this early on. Please do not come to us a
week before the deadline is due and say a team member has not
contributed, this is far to late to do any adjustment to the team. It is
in your responsibility to manage the team.

\subsection{Common Deleiverables}\label{common-deleiverables}

Both Projects and Term paper have the following common deliverables

\begin{description}
\item[Work Breakdown:]
This is an appendix to the document that describes in detail who did
what in the project. This section comes in a new page after the
references. It does not count towards the page length of the document.
It also includes explicit URLs to the the git history that documents the
statistics to demonstrate not only one student has worked on the
project. If you can not provide such a statistic or all checkins have
been made by a single student, the project has shown that they have not
properly used git. Thus points will be deducted from the project.
Furthermore, if we detect that a student has not contributed to a
project we may invite the student to give a detailed presentation of the
project.
\item[Bibliography:]
All bibliography has to be provided in a jabref/bibtex file. This is
regardless if you use LaTeX or Word. Ther is \textbf{NO EXCEPTION} to
this rule. PLease be advised doing references right takes some time so
you want to do this early. Please note that exports of Endnote or other
bibliography management tools do not lead to properly formatted bibtex
files, despite they claiming to do so. You will have to clean them up
and we recommend to do it the other way around. Manage your bibliography
with jabref, and if you like to use it import them to endnote or other
tools. Naturally you may have to do some cleanup to. If you use LaTeX
and jabref, you have naturally much less work to do. What you chose is
up to you.
\item[Report Format:]
All reports will be using the our common format. This format is not the
same as the ACM format, so if you use systems such as overleaf or
sharelatex, you need to upload it and use it there.

The format for LaTeX and Word found here:

\begin{itemize}
\tightlist
\item
  \href{https://github.com/cloudmesh/classes/raw/master/docs/source/format/report.zip}{report.zip}
\item
  \href{https://github.com/cloudmesh/classes/raw/master/docs/source/format/report.tar.gz}{report.tar.gz}
\end{itemize}
\end{description}

There will be \textbf{NO EXCEPTION} to this format. In case you are in a
team, you can use either github while collaboratively developing the
LaTeX document or use MicrosoftOne Drive which allows collaborative
editing features. All bibliographical entries must be put into a
bibliography manager such as jabref, endnote, or Mendeley. This will
guarantee that you follow proper citation styles. You can use either ACM
or IEEE reference styles. Your final submission will include the
bibliography file as a separate document.

Documents that do not follow the ACM format and are not accompanied by
references managed with jabref or endnote or are not spell checked will
be returned without review.

More details about the format can be found at

\begin{itemize}
\tightlist
\item
  \url{https://cloudmesh.github.io/classes/lesson/doc/report.html}
\end{itemize}

\subsection{Software Project}\label{software-project}

\subsubsection{Systems Usage}\label{systems-usage}

Projects may be executed on your local computer, a cloud or other
resources you may have access to. This may include:

\begin{itemize}
\tightlist
\item
  chameleoncloud.org
\item
  furturesystems.org
\item
  AWS (you will be responsible for charges)
\item
  Azure (you will be responsible for charges)
\item
  virtualbox if you have a powerful computer and like to prototype
\item
  other clouds
\end{itemize}

\subsubsection{Deliverables}\label{deliverables}

The following artifacts are part of the deliverables for a project

\begin{description}
\item[Code:]
You must deliver the \textbf{source code} in github. The code must be
compilable and a TA may try to replicate to run your code. You MUST
avoid lengthy install descriptions and everything must be installable
from the command line. We will check submission. All team members must
be responsible for one or all parts of the project.

Code repositories are for code, if you have additional libraries that
are needed you need to develop a script or use a DevOps framework to
install such software. Thus zip files and .class, .o files are not
permissible in the project. Each project must be reproducible with a
simple script. An example is:

\begin{verbatim}
git clone ....
make install
make run
make view
\end{verbatim}

Which would use a simple make file to install, run, and view the
results. Naturally you can use cmd5 (we teach this in class), ansible or
shell scripts. It is not permissible to use GUI based DevOps
preinstalled frameworks. Everything must be installable form the command
line.
\item[Data:]
If the data is small it can be added into a data directory in github. If
you have larger data, it should be downloaded from the internet. IT is
in your responsibility to develop a download program,
\item[Project Report:]
A report must be produced while using the format discussed in the Report
Format section. The following length is required:

\begin{itemize}
\tightlist
\item
  6 pages, one student in the project
\item
  8 pages, two students in the project
\item
  10 pages, three students in the project
\end{itemize}
\item[License:]
All projects are developed under an open source license such as Apache
2.0 License, or similar. You will be required to add a LICENCE.txt file
and if you use other software identify how it can be reused in your
project. If your project uses different licenses, please add in a
README.rst file which packages are used and which license these packages
have.
\end{description}

\subsection{Term Paper}\label{term-paper}

In case you chose the term paper, you or your team will pick a topic
relevant for the class. You will write a high quality scholarly paper
about this topic. The following artifacts are part of the deliverables
for a term paper. A report must be produced while using the format
discussed in the Report Format section. The following length is
required:

\begin{itemize}
\tightlist
\item
  9 pages, one student in the project
\item
  12 pages, two student in the project
\item
  15 pages, three student in the project
\end{itemize}

\section{Grading}\label{grading}

Grading for homework will be done within two weeks of the submission on
the due date. For homework that were submitted beyond the due date, the
grading will be done a month after the submission. A 10\% grade
reduction will be given for residential students if the project is late.
Some homework can not be delivered late (which will be clearly marked
and 0 points will be given if late; these are mostly related to setting
up your account and communicating to us your account names.)

It is the student's responsibility to upload submissions well ahead of
the deadline to avoid last minute problems with network connectivity,
browser crashes, cloud issues, etc. It is a very good idea to make early
submissions and then upload updates as the deadline approaches; we will
grade the last submission received before the deadline.

Note that paper and project will take a considerable amount of time and
doing proper time management is a must for this class. Avoid starting
your project late. Procrastination does not pay off. Late Projects or
term papers will receive a 10\% grade reduction.

\begin{itemize}
\tightlist
\item
  15\% Paper 1
\item
  15\% Paper 2
\item
  60\% Term Paper or Project
\item
  10\% Participation/Discussion
\end{itemize}

All other assignments are pass/fail assignments and are geared towards
you being able to explore rather than you being bound by small
assignments.
