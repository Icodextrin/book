\chapter{Numpy}\label{c:numpy}

NumPy is a popular library on that is used by many other python
librariessuch as pandas, and SciPy. It provides simple to use array
operations for data. This helps to accass arrays in a more intuitive
fashion and introduces various matrix operations.

We provide a short introduction to Numpy.

First we import the modules needed for this introduction and abreviate
them with the \texttt{as} feature of the import statement

\begin{lstlisting}
import numpy as np
import matplotlib as mpl
import matplotlib.pyplot as plt
\end{lstlisting}

Now we showcase some features of Numpy.

\section{Float Range}

\texttt{arange()} is like \texttt{range()}, but for floating-point
numbers.

\begin{lstlisting}
X = np.arange(0.2,1,.1)
\end{lstlisting}

\begin{lstlisting}
print (X)
\end{lstlisting}

We use this function to generate parameter space that can then be
iterated on.

\begin{lstlisting}
P = 10.0 ** np.arange(-7,1,1)

print (P)
\end{lstlisting}

\begin{lstlisting}
for x,p in zip(X,P):
    print ('%f, %f' % (x, p))
\end{lstlisting}

\section{Arrays}

To create one dimensional arrays you use

\begin{lstlisting}
a = np.array([1, 2, 3])   
\end{lstlisting}

To check some properties you can use the following

\begin{lstlisting}
print(type(a))            # Prints "<class 'numpy.ndarray'>"
\end{lstlisting}

\begin{lstlisting}
print(a.shape)            # Prints "(3,)"
\end{lstlisting}

The shape indicates that in the first dimension, there are 3 elements.
To print the actual values you can use

\begin{lstlisting}
print(a)                  # Prints the values of the array
print(a[0], a[1], a[2])   # Prints "1 2 3"
\end{lstlisting}

To change values you can use the index of the element or use any other
python method to do so. IN our example we change the first element to
\texttt{42}

\begin{lstlisting}
a[0] = 42                 
print(a)                  
\end{lstlisting}

To create more dimensional arrays you use

\begin{lstlisting}
b = np.array([[1,2,3],[4,5,6]])    # Create a 2 dimensional array
print(b.shape)                     # Prints "(2, 3)"
print(b[0, 0], b[0, 1], b[1, 0])   # Prints "1 2 4"
\end{lstlisting}

\section{Array Operations}\label{array-operations}

Let us first create some arrays with a predefined datatype

\begin{lstlisting}
x = np.array([[1,2],[3,4]], dtype=np.float64)
y = np.array([[5,6],[7,8]], dtype=np.float64)
\end{lstlisting}

\begin{lstlisting}
print (x)
\end{lstlisting}

\begin{lstlisting}
print(y)
\end{lstlisting}

To add the numbers use

\begin{lstlisting}
print(x+y)
\end{lstlisting}

Other functions such as \texttt{-}, \texttt{*}, \texttt{/} behave as
expected using elementwise oerations:

\begin{lstlisting}
print(x-y)
\end{lstlisting}

\begin{lstlisting}
print(x*y)
\end{lstlisting}

\begin{lstlisting}
print(x/y)
\end{lstlisting}

To apply functions such as `sin' make sure you use the function provided
by the numpy package such as \texttt{np.sin}. The list of functions is
included in the manual at *
\url{https://docs.scipy.org/doc/numpy/reference/routines.math.html}

\begin{lstlisting}
print (np.sin(x))
\end{lstlisting}

\begin{lstlisting}
print (np.sum(x))
\end{lstlisting}

Computations can also be applied to columns and rows

\begin{lstlisting}
print(np.sum(x, axis=0)) # sum of each column
print(np.sum(x, axis=1)) # sum of each row
\end{lstlisting}

\section{Linear Algebra}

Linear algebra methods are also provided.

\begin{lstlisting}
from numpy import linalg 

A = np.diag((1,2,3))

w,v = linalg.eig(A)

print ('w =', w)
print ('v =', v)
\end{lstlisting}

\section{Resources}

\begin{itemize}
\item \url{https://docs.scipy.org/doc/numpy/}
\item \url{http://cs231n.github.io/python-numpy-tutorial/\#numpy}
\end{itemize}
