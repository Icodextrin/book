%---------------------------------------------------------------
%	PART
%---------------------------------------------------------------

\part{Preface}



%---------------------------------------------------------------
%	CHAPTER
%---------------------------------------------------------------

\chapterimage{water.png} % Chapter heading image

\chapter{Introduction}

\section{Citation}

The citation for this document is 

\begin{quote}

Gregor von Laszewski, Geoffrey C. Fox, and Judy Qiu, Handbook of
Clouds and Big Data -- Theory and Practice, Indiana University,
Jan., 2018
Smith Research Center, Bloomington, IN 47408
\URL{https://github.com/laszewski/laszewski.github.io/raw/master/papers/vonLaszewski-bigdata.pdf}
\end{quote}

\begin{WARNING}
The url will change
\end{WARNING}

The bibtex entry for this document is

\begin{verbatim}
@TechReport{las17handbook,
  author =       {Gregor von Laszewski and Geoffrey C. Fox and Judy Qiu},
  title =        {Handbook of Clouds and Big Data -- Theory and Practice},
  institution =  {Indiana University},
  year =         {2018},
  OPTtype =      {Draft},
  address =      {Smith Research Center, Bloomington, IN 47408},
  month =        jan,
  url={https://github.com/laszewski/laszewski.github.io/raw/master/papers/vonLaszewski-bigdata.pdf},
}
\end{verbatim}



\section{Authors}

\FILENAME

\begin{description}

\item[Gregor von Laszewski] \index{von Laszewski, Gregor} Gregor von
  Laszewski is an Assistant Director DSC in the School of Informatics
  and Computing and Engeneering at Indiana University. He holds also a
  position as Adjunct Professor in the Intelligent Systems Engineering
  Department. Previously he held Adjunct Professor positions at the
  Computer Science Department at Indiana University and University of
  North Texas. He received a Ph.D. from Syracuse University in
  computer science.

  He served as the architect of the the FutureGrid project. He held a
  position at Argonne National Laboratory where he was last a
  scientist and a fellow of the Computationand part of the initial
  team working on Grid computing.

  At IU He served as the architect of the the FutureGrid project. His
  current interest and projects include cloud computing, big data, and
  scientific impact metrics, and edge computing.  He initiated the
  Cloudmesh project which is a toolkit to enable cloud computing
  across various Cloud and Container IaaS such as OpenStack, AWS,
  Azure, docker, docker swarm, and kubernetes.


\item[Geoffrey C. Fox] \index{Fox, Geoffrey C} Fox received a Ph.D. in
  Theoretical Physics from Cambridge University and is now
  distinguished professor of Informatics and Computing, and Physics at
  Indiana University where he is director of the Digital Science
  Center, Chair of Department of Intelligent Systems Engineering and
  Director of the Data Science program at the School of Informatics,
  Computing, and Engineering.

  He currently works in applying computer science from infrastructure
  to analytics in Biology, Pathology, Sensor Clouds, Earthquake and
  Ice-sheet Science, Image processing, Deep Learning, Manufacturing,
  Network Science and Particle Physics. The infrastructure work is
  built around Software Defined Systems on Clouds and Clusters. The
  analytics focuses on scalable parallelism.

\item [Dr. Judy Qiu] is an Associate Professor in the School of
  Informatics and Computing at Indiana University. Her research
  interests focus on data-intensive computing at the intersection of
  cloud and multicore technologies, with an emphasis on life science
  applications using MapReduce as well as traditional parallel and
  distributed computing approaches. Her contributions are focused on
  Hadoop and Introdcution into Cloud Computing. 

\end{description}

\section{Acknowledgements}

Part \ref{P:chameleon} focussing on Chameleon Cloud was contributed by
Kate Kaheay the lead of the Chameleon Cloud project sponsored by NSF
grant 1743358, Collaborative Research: Chameleon: A Large-Scale,
Reconfigurable Experimental Environment for Cloud Research. Some
content has been modified and added by Gregor von Laszewski to target
specifically classes and education material targeted for Indiana
University.

\section{About}

The material in this document is covering material that is used in the
following classes at Indiana University.

\begin{itemize}
\item Undergraduate: E222 Intelligent systems II
\item Graduate: E516 Introduction to Cloud Computing
\item Graduate: E616 Advanced Cloud Computing
\item Graduate: E534 Big Data Applications
\end{itemize}

Material from earlier classes known under the numbers I523 and I524
and Introduction to Cloud Computing have influenced this material. The
collection of this material is updated continiously and new versions
will be made available throughout the semester.

\section{Results}\label{S:results}

In addition to this document we list documents about BigData
technologies and Projects that have been delivered by students of past
classes. As we can not post any grades it is importan that you
yourself identify good examples from the published list. It is
important not to make the mistake to work towards minimal fulfillment
of the class requirements, but instead work towards achieving your
best. All documents produced by this class will be made available on
github.com and through this document. This helps other students to
learn from your experience and to counteract plagiarism.

\begin{NOTE}
  Examples provided here may not necessarily meet the requirements for
  your current class as the content and requirements have changed
  since the other classes were thought. This includes format of the
  paper, paper length, as well as the topic for projects.
\end{NOTE}

\subsection{Introduction to Big Data Applications and Analytics}

This class was known under the name I523.

% this document needs to be deleted as it is temporary
% \URL{https://github.com/laszewski/laszewski.github.io/raw/master/papers/vonLaszewski-i523.pdf}

Application and Technologies (Vol.1):
\URL{https://github.com/laszewski/laszewski.github.io/raw/master/papers/vonLaszewski-i523-v1.pdf}

Application and Technologies (Vol. 2):
\URL{https://github.com/laszewski/laszewski.github.io/raw/master/papers/vonLaszewski-i523-v2.pdf}

Project: First 500 pages:
\URL{https://github.com/laszewski/laszewski.github.io/raw/master/papers/vonLaszewski-i523-v3-1.pdf}

Project: Second part, starting past page number501:
\URL{https://github.com/laszewski/laszewski.github.io/raw/master/papers/vonLaszewski-i523-v3-2.pdf}

\subsection{Big Data Applications and Open Source Software}

This class was known under the name I524

Big Data Software Vol 1.:
\URL{https://github.com/cloudmesh/sp17-i524/blob/master/paper1/proceedings.pdf}

Big Data Software Vol 2.:
\URL{https://github.com/cloudmesh/sp17-i524/blob/master/paper2/proceedings.pdf}

Big Data Projects:
\URL{https://github.com/cloudmesh/sp17-i524/blob/master/project/projects.pdf}


\section{Contributing}

We encourage students of the classes to contribute to this material,
provide corrections, and additions.

This document is managed on \verb|github.com| at 

\URL{https://github.com/cloudmesh/book/}

The current release version is held in the master branch.
Development versions will be held under a number of branches:

\begin{description}
\item[e516] Branch with contributions from students of the e616
  class. Merges to and from the {\em latex} branch will be conducted
  on a daily bases by TA's.
\item[e616] Branch with contributions from students of the e616
  class. Merges to and from the {\em latex} branch will be conducted
  on a daily bases by TA's.
\item[dev] Branch managed by Gregor and the TA's
\item[master] Branch that contains the current released version. This
  version is updated once a week from the branch {\em latex}.
\end{description}

Contributions are to be conducted as pull requests. It is important to
keep the pull requests small and on a section or even paragraph
base. This helps avoiding conflicts at time of checkin and is a common
practice in large communities. It is not a good practice to work for
weeks on improvements and than issue the pull request. For sure things
will have changed and it will take you a long time to catch up.

The document is based on selected material published at the following
Web page

\URL{https://cloudmesh.github.io/classes/}

It is part of a classes taught at Indiana University. The class
communication takes place in piazza and you need to enroll in it via
CANVAS where you will find a panel for it.

The PDF version will be made in future available at 

\URL{https://github.com/laszewski/laszewski.github.io/raw/master/papers/vonLaszewski-bigdata.pdf}

This PDF document will be updated based on feedback from the students
and once we have now material available. 

\section{Contributing}
\index{Contributing}

\subsection{Pull requests}

It is easy to contribute to this document and we invite everyone to
improve the material. To do so you need to fork the repository from 

\URL{https://github.com/cloudmesh/book/}

and clone it. Then you can modify information in the different files
or add new sections. It is important that you make changes based on
sections and than for them create a new pull request. This simplifies
the review process. We will typically want only one file to be
changed. Aslo before you issue your pull request make sure that no one
else has already made changes. In that case we ask you to integrate
them into your document.



\subsection{Contributors}

We like to acknowledge the following contributors that helped on this
document. Please notify us with your name and a brief commend on what
you contributed:

Descriptions provided in Section \ref{s:} were contributed by the
following people that are either listed by full name or their
github.com id:

\begin{quotation}{\em
Abhijit Thakre, Abhishek Gupta, Abhishek Naik, Ajit Balaga, Anurag
Kumar Jain, Avadhoot Agasti, Badi' Abdul-Wahid, Cmbays, DIKSHA,
Dimitar Nikolov, Govind, Govind Mishra, Grace Li, Gregor von
Laszewski, Harshit Krishnakumar, Hyungro Lee, Jerome Mitchell, Jimmy
Ardiansyah, Jon, Jon Montgomery, Jordan Simmons, Juliette Zerick,
Karthik, Kumar Satyam, Mark McCombe, Matthew Lawson, Methkupalli
Vasanth, Miao Jiang, Miao Zhang, Milind Suryawanshi,
MilindSuryawanshi, Nandita Sathe, Naveen, Niteesh01, Piyush Rai,
Piyush Shinde, Prashanth, Pratik Jain, Rahul Raghatate, Rahul Singh,
Ribka Rufael, Ronak Parekh, Saber Sheybani, Sabyasachi Roy Choudhury,
Sagar Vora, Sahiti Korrapati, Scott McClary, Sean Shiverick,
SilviaKarim14, Sivaprasad Sushmita, Snehal Chemburkar, Sowmya Ravi,
Srikanth Ramanam, Sunanda Unni, SushmitaSivaprasad, Tony Liu, Vasanth
Methkupalli, Veera Marni, Vibhatha Abeykoon, Vibhatha Lakmal Abeykoon,
Vishwanath Kodre, William H Knapp III, acastrob, ak.15, alyez,
anveling, argetlam115, athakre, bhavesh37, cacoulte, cglmoocs,
elenadesigner, eunosm3, harkrish1, jemitchell, justbbusy, jzerick,
kartanba, karthick, karthick venkatesan, karthik-anba, kpvenkat,
ksrivatsav, lmundia, miaozhan, michaelsmith1983, mmccombe, nsathe,
piyurai, pratik11jain, ronak1182, sabyasachi087,
shah0112, sriramsitharaman, suunni, tifabi, tonythomascn, vasanth,
vibhatha, vkodre, vlabeyko, xl41, yatinsharma7
}\end{quotation}


\section{Conventions}
\index{Convention}

\subsection{Videos}

Videos to the class are referred to with embedded links into the PDF
document as follows: 

\video{About}{25:36}{Test Video}{https://www.youtube.com/watch?v=yC3PNkb_9mI}

An index will also be available in the index page
that lists on which page the video has been added.

\subsection{Slides}

Sides
\slides{About}{10}{Test slides}{PUT URL HERE}

\subsection{Images}

The video icon was copied from \url{http://www.freeiconspng.com/img/8039}.

\subsection{URLs}

The online version of this document contains a significant number of
links. The links are either embedded or are directly visible. The
color of the links is blue.

\begin{description}
\item[Direct URL:] This is an example for a
  \url{https://github.com/cloudmesh/book/}
\item[Embedded URL:] This is an example for an embedded URL that
  points to the \href{https://github.com/cloudmesh/book/}{source on github}
\end{description}

\subsection{Boxes}

\begin{NOTE}
Notes are written in blue boxes and indicate a clarification or some
important information that you do not want to overlook.
\end{NOTE}

\begin{WARNING}
Warnings are written in red  boxes and indicate a piece of information
that you must not ignore.
\end{WARNING}

\begin{IU}
Red boxes with Indiana University are information that relate to
students that use this material as part of the courses offered at
Indiana University.
\end{IU}

To dos are highlighted in boxes with the keyword TODO. The offer
opportunities for student to gather points for the discussion grade.

\TODO{An example todo}

\section{Exercise}

\begin{description}
\item[Exercise.Preface.1:] Inspect the PDF documents produced by previous
classes. Note the differences between technology and application
reviews and projects. 
\item[Exercise.Preface.2:] COntribute to this document while finding a
  single spelling error. Before you do the pull request, make sure the
  document compiles.
\end{description}


